\documentclass{article}
\usepackage[margin=1in]{geometry}
\usepackage{amsmath,amssymb,amsthm}
\usepackage[utf8]{inputenc}
\usepackage[english]{babel}

\newcommand{\Group}{\mathcal{G}}

\theoremstyle{definition}
\newtheorem{definition}{Definition}[section]

\newtheorem{theorem}{Theorem}[section]
\newtheorem{corollary}{Corollary}[theorem]
\newtheorem{lemma}[theorem]{Lemma}

\begin{document}
    \theoremstyle{definition}
    \begin{definition}
        A \underline{Binary Operation} on a set $\mathcal{S}$ is a mapping $*$ from
        $\mathcal{S}\times\mathcal{S}\to\mathcal{S}$ that associates elements in $\mathcal{S}\times\mathcal{S}$ to an
        element in $\mathcal{S}$.
    \end{definition}

    \theoremstyle{definition}
    \begin{definition}
        A \underline{Group} is a pairing of a set $\Group$ and a binary operator $*$ $(\Group,*)$ such that
        \begin{enumerate}
            \item $*$ is associative, i.e. $a*(b*c) = (a*b)*c$.
            \item $\exists e\in\Group \backepsilon x*e=e*x=x$. This is called the identity.
            \item $\forall x\in\Group, \exists y\in\Group \backepsilon x*y = y*x =e$. $y$ is called the inverse of $x$.
        \end{enumerate}
        If $*$ is commutative, then $\Group$ is called \underline{abelian}.
    \end{definition}

    \begin{theorem}
        Let $*$ be a binary operation on $\mathcal{S}$. If there is an identity, it is unique.
    \end{theorem}
    \begin{proof}
        Let $e,f$ be identities in $(\mathcal{S},*)$. Then $e*f=e$ and $e*f=f$, so $e=f$.
    \end{proof}

    \begin{theorem}
        If $\Group$ is a group, then the inverse of an element is unique.
    \end{theorem}
    \begin{proof}
        Let $a\in\Group$ with $a''$ and $a'$    
    \end{proof}

\end{document}
