\documentclass{article}
\usepackage{amssymb,amsmath}
\usepackage{mathtools}
\usepackage{graphicx}
\usepackage{multicol}
\topmargin=-0.3in
\textheight=9.2in
\textwidth=168mm
\oddsidemargin=-0.2in
\evensidemargin=-0.2in

\begin{document}
\section*{16 Nov 2015}
\hrule
\noindent\\
\subsection*{Chapter 5: Memory}
\begin{multicols}{2}
5 classic components of a computer
\begin{itemize}
\item Datapath
\item Control
\item Memory
\item Input
\item Output
\end{itemize}
Memory is composed of Cache, Main Memory, and Secondary Memory\\
\begin{itemize}
\item Cache memory is composed of Flip-Flops
\item Cache is physically larger than RAM memory (main memory)
\item RAM has one transistor and one capacitor; cache has two AND gates
\end{itemize}
Memory Types\\
\textbf{RAM}
\begin{itemize}
\item Main Memory
\item \textbf{DRAM}: High density, low power, cheap, slow
\item \textbf{SDRAM}: Faster than DRAM
\item \textbf{DDR RAM}: Twice the amount of data in a single bus clock cycle
\item \textbf{SRAM}: Does not refresh and stays as long as there is power.
\begin{itemize}
\item Three levels
\item More complex: 6 - 8 transistors
\item Fast: very close to the CPU
\item Used for cache memory
\end{itemize}
\end{itemize}
Larger capacity memory is usually slower\\\\
Performance
\end{multicols}
\end{document}