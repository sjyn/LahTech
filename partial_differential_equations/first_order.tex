\section{First Order Partial Differential Equations}
\hrule
\noindent\\\\
\noindent\textbf{\textit{Ex:}} Solve the following PDE
\[
u_{x} + u_{y} + u = x + y
\]
\indent \textbf{\textit{Solution:}} In this case, we have a non-zero term on the
right hand side of the equation. We will use substitution to solve this. For the
time being, we will let $T$ be $x + y$. To find our $S$, we can use a similar
method to the one that we used above. We know that the coefficients on the
$u_{x}$ and $u_{y}$ are $1$ in this case. Now we have:
\begin{align*}
\frac{dy}{dx} &= 1\\
dy &= dx\\
\int dy &= \int dx\\
x &= y + c\\
x - y &= c
\end{align*}
\noindent Now we can say that $S = x - y$; however, we need to check our choice
for $T$ to make sure this substitution will not just yield a trivial solution.
We can do this by Next we have to actually use the substitution. We need to
determine whether the coefficient matrix, or the Jacobian, of our substitutions
has a determinant of zero. If it does, then we need to choose a different $T$.
\[
\left|
\begin{array}{c c}
1 & 1\\
1 & -1
\end{array}
\right| = 2 \neq 0.
\]
Our choices of variables will work here, so now we need to express $u_{x}$ and
$u_{y}$ in terms of $S$ and $T$. The following equations come from the chain
rule; when the $S$ and $T$ terms are differentiated, we get $u_{S}$ and $u_{T}$
left behind.
\begin{alignat*}{3}
u_{x} &= u_{S}\frac{\partial S}{\partial x} + u_{T}\frac{\partial T}{\partial x}
\qquad\qquad &&u_{y} = u_{S}\frac{\partial S}{\partial y} + u_{T}\frac{\partial T}{\partial y}\\
u_{x} &= -u_{S} + u_{T}  \qquad\qquad &&u_{y} = u_{S} + u_{T}
\end{alignat*}
\noindent We can now plug in our values into the original equation. Note that we
didn't do anything with $u$; it gets left alone. In addition, if we needed
$u_{xx}$ or $u_{yy}$, we can just differentiate again. Plugging in yields:
\begin{gather*}
(-u_{S} + u_{T}) + (u_{S} + u_{t}) + u = T\\
2u_{T} + u = T
\end{gather*}
\noindent This is just a first order ODE. Since it's not homogeneous, we need to
solve for the general solution, and then the particular solution. Let's start
with the general.
\begin{gather*}
2y^{'} + y = 0\\
2r + 1 = 0\\
r = -\frac{1}{2}\\
y = ce^{\left(-\frac{1}{2}\right)x}
\end{gather*}
\noindent In this case, we know that our $x$ is really $T$. So we finally have
$y = ce^{\left(-\frac{1}{2}\right)T}$. We will back substitute soon, but let's
find the particular solution before we do that.
\noindent Recall from ODE that we have several cases for the particular
solution. In our case, we have a linear equation on the right hand side of our
ODE. This lets us say:
\begin{gather*}
y = AT + B\\
y^{'} = A\\
\end{gather*}
\noindent Now we can substitute again into our non-homogeneous ODE as follows:
\[
2y^{'} + y = T \equiv 2A + AT + B = T
\]
\noindent And we have that $A = 1$, and $B = -2$. Now we have a final solution of:
\begin{gather*}
y = ce^{\left(-\frac{1}{2}\right)T} + T - 2\\
y = ce^{\left(-\frac{1}{2}\right)(x+y)} + (x + y) - 2
\end{gather*}
\noindent We aren't quite done yet, as we have a solution to an ODE, not a PDE.
It isn't too difficult to switch back; The $c$ is really the only difference. In
our ODE, the $c$ is a constant term, but to our PDE, the $c$ is a function. The
$c$ now becomes $f(s)$. We will go ahead and back substitute, which will give us
$f(y-x)$. Now we have our final solution:
\[
u(x,y) = f(y-x)e^{\left(-\frac{1}{2}\right)(x+y)} + (x + y) - 2
\]
\noindent And we are done.
