\section{Bessel Functions}
\hrule
\noindent\\

The 2D heat equation provided us a way to describe heat flow through a 2D surface, such as a thin rod.
Now imagine that we have a surface that is shaped like a large cylinder. We've now moved into 3D space,
and as such we will have to come up with a way to describe the heat transfer through the cylinder. Let's
start by recalling Laplace's equation,
\[
\nabla^{2}u = 0,
\]
which described a vector composed of the all the partial derivatives of $u$. We will need to keep this
equation in mind as we concern ourselves with the 3D heat equation.\\

Our end goal will be to come up with a formal solution for the equation
\[
u_{t} = k^{2}(u_{xx} + u_{yy}).
\]
This is, of course, the heat equation expanded to a cylinder. Recall that when we dealt with Laplace's
equation on a disk, we moved into polar coordinates. The 3D heat equation contains the laplacian in it,
represented as
\[
u_{xx} + u_{yy}.
\]
Moving into polar coordinates will allow us to have rectangular bounds on our function. Transforming our
heat equation yields
\[
u_{t} = k^{2}(u_{rr} + {1 \over r}u_{r} + {1\over r^{2}}u_{\theta\theta}).
\]
We now have a rectangular bound, which allows us to use the familiar separation of variables technique to
solve the PDE. In this case, we want to separate the spacial variables ($r,\theta$) from the time variable
($t$). We have a general solution of
\[
u = V(r,\theta)T(t) = 0.
\]
Setting them equal to a common value yields
\[
{T \over k^{2}T} = {V_{rr} + {1\over r}V_{r} + {1\over r^{2}}V_{\theta\theta} \over V} = -\lambda^{2}.
\]
As expected, we are left with 2 ODEs. The first ODE, concerning $T$, will be left alone for now.
The remaining ODE can be solved by using separation of variables again. Doing so yields
\begin{gather*}
V = R(r)\nu(\theta)\\
\begin{cases}
\eta^{''} + \gamma\eta = 0\\
r^{2}R'' + rR' + (k^{2}r^{2} - \gamma^{2})R = 0
\end{cases}.
\end{gather*}
Again, we will leave the $\eta$ function for later, and instead turn our attention to the equation
concerning $R$. This leads us to our discussion of Bessel Functions.\\
