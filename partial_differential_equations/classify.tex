\section{Classification of Partial Differential Equations}
\hrule
\noindent\\\\
\indent We mentioned earlier the topic of classification of PDE's. Since we only
have this and Fourier Series to talk about, we will go ahead and knock out
classification. There's some information that we need to go ahead and review
here. This will be explained in the problem, but it's handy to have it for
reference:
\begin{align*}
b^{2} - ac < 0 &\implies \text{Elliptic with complex roots}\\
b^{2} - ac = 0 &\implies \text{Parabolic with the same root twice}\\
b^{2} - ac > 0 &\implies \text{Hyperbolic with two distinct real roots}
\end{align*}
\noindent Another helpful piece of information is what's known as the
\textit{classical trinity} of PDEs.
\begin{align*}
u_{t} - ku_{xx} &= 0 \qquad \text{The heat equation (parabolic)}\\
u_{tt} - c^{2}u_{xx} &= 0 \qquad \text{The wave equation (hyperbolic)}\\
u_{xx} + u_{yy} &= 0 \qquad \text{The Laplace equation (elliptic)}
\end{align*}
Let's go ahead and look at an example.\\\\
\indent \textbf{Ex. }Classify the following PDE:
\[yu_{xx} + u_{yy} = 0\]
\indent \textbf{\textit{Solution:}} The first thing we need to identify are the
coefficients on the $u_{xx}$, $u_{yy}$, and $u_{xy}$ or $u_{yx}$ terms. Note
that the $u_{xy}$ and $u_{yx}$ are interchangeable. We will assign these
coefficients to $a$, $b$, and $c$ corresponding to $u_{xx}$, $u_{xy}$, and
$u_{yy}$, respectively. Looking at the PDE, we can see that
\[
a = y \qquad b = 0 \qquad c = 1
\]
\noindent Now we look at $b^{2} - ac$. We are really dealing with
$b = \frac{b_{0}}{2}$. In other words, we need to divide our $b$ value by $2$.
In the case of the first problem, we are dealing with $b = 0$. Now we can
classify our PDE as follows:
\[
b^{2} - ac = 0 - y < 0 \implies \text{Elliptic}
\]
Note that we assume $y > 0$. So our PDE is elliptic. Now we need to make it look
like it's corresponding equation from the classical trinity. In this case we
have an Elliptic equation, so we want to make it look like Laplace's equation.
We need to find values for $u_{xx}$ and $u_{yy}$, and then substitute those into
the equation. This is similar to how we solved the first order PDE's; let's look
at our characteristic equation:
\begin{gather*}
am^{2} + bm + c = 0\\
ym^{2} + 1 = 0\\
m = \pm \frac{i}{\sqrt{y}}
\end{gather*}
\noindent So, now we have two solutions, and they are both imaginary as expected.
Next we need to find values for $S$ and $T$ that we can substitute into the
equation. We will do both of them at the same time here. Notice that we will
drop the $i$:
\begin{gather*}
\frac{dy}{dx} = \frac{1}{\sqrt{y}} \qquad\qquad\qquad \frac{dy}{dx} = -\frac{1}{\sqrt{y}}\\
\int\frac{dy}{dx} =\int\frac{1}{\sqrt{y}} \qquad\qquad\qquad \int\frac{dy}{dx} = -\int\frac{1}{\sqrt{y}}\\
S = \frac{2}{3}y^{\left(\frac{3}{2}\right)} - x \qquad\qquad\qquad T = \frac{2}{3}y^{\left(\frac{3}{2}\right)} + x
\end{gather*}
\noindent Now we can substitute into our equation. Unfortunately, we will need
to take the partial derivative twice in order to get the substitutions we need.
Let's start with $u_{xx}$.
\begin{align*}
u_{x} &= u_{S}\frac{\partial S}{\partial x} + u_{T}\frac{\partial T}{\partial x}\\
&= -u_{S} + u_{T}\\
u_{xx} &= \frac{\partial}{\partial x}(-u_{S} + u_{T})\\
&= u_{SS} + u_{TT}
\end{align*}
\noindent We will skip the math for finding $u_{yy}$, which would give us
\[
u_{yy} = yu_{SS} + yu_{TT}
\]
\noindent Now we plug everything in. Remember that we were under the assumption
that $y > 0$:
\begin{gather*}
y(u_{SS} + u_{TT}) + yu_{SS} + yu_{TT} = 0\\
2yu_{SS} + 2yu_{TT} = 0\\
u_{SS} + u_{TT} = 0
\end{gather*}
\noindent And we are finished.
