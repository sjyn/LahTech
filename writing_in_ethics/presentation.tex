\documentclass{beamer}

\usepackage{url}
\usepackage{graphicx}

\makeatletter
\g@addto@macro{\UrlBreaks}{\UrlOrds}
\makeatother

\setbeamertemplate{bibliography item}{}
\setbeamertemplate{bibliography entry title}{}
\setbeamertemplate{bibliography entry location}{}
\setbeamertemplate{bibliography entry note}{}

\title{DLC}
\author{Steven Rosendahl}
\date{}

\usetheme{metropolis}

\begin{document}
\frame{\titlepage}

\begin{frame}
\frametitle{The Issue}
\begin{itemize}
\pause
\item Is the concept of Downloadable Content (DLC) ethical?
\pause
\item Should companies charge money for a product they market as free?
\pause
\item Should companies market an incomplete product and force users to pay more for the rest of the product?
\pause
\item When does DLC change from a fun add on to a game to a required purchase for a whole game?
\end{itemize}
\end{frame}

\begin{frame}
\frametitle{Examples}
% \begin{columns}
% \begin{column}{0.5\textwidth}
\begin{enumerate}
\item The Saint's Row Franchise
\pause
\begin{itemize}
\item Saints Row: The Third released November 15, 2011
\pause
\item Saint Row IV released August 20, 2013
\begin{itemize}
\pause
\item A re-skinned version of Saints Row: The Third
\end{itemize}
\pause
\item Saints Row: Gat Outta Hell released January 20, 2015
\begin{itemize}
\pause
\item Marketed as a separate game, but only about $1/8$ as long as Saints Row IV
\end{itemize}
\end{itemize}
\pause
% \begin{figure}
% \end{figure}
\begin{columns}[T]
\begin{column}{0.5\textwidth}
\item Train Simulator 2016
\begin{itemize}
\item Base game: \$29.99.
\pause
\item All DLC: $> \$5,400.00$.
\end{itemize}
\end{column}

\hspace{-1.8cm}

\begin{column}{0.5\textwidth}
\pause
\includegraphics[scale=0.1]{p3vp4}
\pause

\vspace{0.2cm}

\includegraphics[scale=0.4]{train_sim_dlc}
\end{column}

\hspace{-2cm}

\end{columns}
\end{enumerate}
\end{frame}

\begin{frame}
\frametitle{Kantianism}
\begin{itemize}
\item ``Never use people as a means to an end"
\pause
\begin{itemize}
\item Companies no longer care about making games
\pause
\item Profit can be maximized by continually making DLC at the expense of the gamer
\pause
\item All major game companies follow the DLC pattern, so gamers are forced to buy into it
\pause
\item The intention is to make more money off of gamers, which is using them as a means to an end
\pause
\end{itemize}
\end{itemize}
\begin{center}
\includegraphics[scale=0.3]{unethical}
\end{center}
\end{frame}

\begin{frame}
\frametitle{Utilitarianism}
\begin{itemize}
\item ``The most amount of good for the most amount of people"
\pause
\begin{itemize}
\item Companies turn large profits from DLC
\pause
\item Companies use that profit to create more content for gamers
\pause
\item Gamers enjoy playing the new games
\pause
\item The cycle repeats
\end{itemize}
\end{itemize}
\pause
\begin{center}
\includegraphics[scale=0.3]{ethical}
\end{center}
\end{frame}

\begin{frame}
\frametitle{Social Contract}
\begin{itemize}
\item ``Rules that are agreed upon by all"
\pause
\begin{itemize}
\item Game Companies vs. Gamers
\pause
\item Game companies want to turn a large profit
\pause
\item Gamers want complete games at a reasonable price
\pause
\item There has to be a compromise that both parties can agree upon
\pause
\item Current system favors the game companies, not the gamers
\end{itemize}
\end{itemize}
\begin{center}
\includegraphics[scale=0.3]{unethical}
\end{center}
\end{frame}

\begin{frame}[shrink=20]
\frametitle{References}
\vspace{1cm}
\nocite{*}
\bibliographystyle{alpha}
\bibliography{pres_bib}
\end{frame}

\end{document}
