\documentclass{paper}
\title{Catfish}

\addbibresource{pt_bib.bib}

\begin{document}
\drafth{}
\inlinetitle

\begin{linenumbers}

The television show, Catfish, is a show where the hosts, Nev and Ariel, attempt to help others find out the truth behind online relationships. Often times, the subject of the show is a victim of ``Catfishing," or being deceived by an anonymous person online who claims they are someone who they are not. The ``Catfishers" are people who create fake Facebook or Twitter profiles, use other's property, such as pictures, and hide behind the anonymity of the internet to deceive others. While it is unethical to lie to others under most ethical theories, a Utilitarian perspective provides insight into the ethics of many of the methods that Catfishs use to Catfish others.

The key focus of Act Utilitarianism is that actions are determined ethical or unethical based on the quantitative outcome of the action \cite[74]{Ethics}. In almost every situation presented in the show, the victim only communicates to the Catfish over the internet, usually Facebook or a similar platform \cite{wiki}. Since Facebook requires users to use actual names, many Catfishers steal other's real names. Under Utilitarianism, this may not actually be an inherently unethical action. For example, a Catfish may steal a name without any intention of causing harm to the person from whom they stole the name. In one episode of the show, a woman undergoing Transgender surgery assumed the name of a close friend of the victim because she found the name interesting \cite{ctfs}. The outcome of this action is just that the Catfish goes under another name, rather than his or her own name, which has no negative effect. In other situations, Catfishers have been known to assume the identity of another person to harm that person. The negative effect of this action far outweighs the neutrality of simply picking the name. In that respect, stealing another's name is certainly unethical.

Technology, especially the internet, has made it very easy for people to take other people's property, such as pictures and other artistic works. In many episodes of Catfish, Catfishers will steal the profile pictures of others. As Nev stated in one episode of the show, ``...that's when you know [the catfishing] is bad... when they use someone else's profile picture" \cite{ctfs}. Taking pictures from the internet is not necessarily unethical or even illegal. People's Facebook profile pictures are easy to download, and there is no Facebook policy that prevents users from taking other's profile pictures. Since Utilitarianism does not focus on the intention of the action, it may not even be unethical for a Catfish to take another's profile pictures and use them as their own. In order to determine the ethical implications, one must consider the amount of good and bad produced by the action. Posing as someone else only does harm if the Catfish is attempting to make the victim look bad. If the Catfish does not defame or boost the victim's character, then the Catfish has not done anything unethical. Moreover, if the Catfish makes the original picture's owner look better, then the amount of good done actually outweighs the bad.

Catfish often try to remain anonymous online.

Not every episode of Catfish ends with deceit. In some cases, the Catfish is struggling with other issues that end up misinterpreted by the other party. One episode in particular resulted in the marriage of two lovers after they met following an eight year online relationship.

\end{linenumbers}
\newpage

\printbibliography

\end{document}
