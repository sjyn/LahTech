\documentclass{paper}

\title{Utilitarianism}
\date{}
\author{}

\addbibresource{generic_bib.bib}

\begin{document}
\drafth{}
\inlinetitle

The ethical theory of Utilitarianism can be used to analyze the questions that arise from Alexis'
situation.\\

The first question that needs to be answered is whether or not Alexis has done anything that can be
considered morally wrong. According to the principal of utility, it appears that Alexis has not actually
committed a moral wrong. In Alexis' situation, she has two choices; she either uses the computers, as she
has chosen, or she does not. If Alexis had chosen to not use the other user's login information, she would
not have had an outcome that increased her happiness. Even though Alexis did steal the information from
another student, that student did not experience any decrease in happiness, according to the scenario.\\

Bentham proposes several attributes that should be taken into account when calculating the utility of an
action. The attributes that apply most directly to Alexis' situation are \textit{Intensity},
\textit{Duration}, and \textit{Extent} \cite{ethics}.

\newpage
\printbibliography

\end{document}
