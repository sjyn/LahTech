\documentclass{paper}

\title{Utilitarianism}

\addbibresource{generic_bib.bib}

\linenumbers

\begin{document}
\drafth{}
\inlinetitle

The theory of Utilitarianism can be used to quantitatively analyze various situations that arise in day to day life; it can be applied the scenario presented in \textit{Ethics for the Information Age} that concerns Alexis' education. In order to analyze the situation and determine whether or not her actions were ethically ``good" or ``bad", the total utility change needs to be calculated. Alexis' actions can be considered ethically good if the benefit of her actions exceed the harms. The situation can be analyzed using both Rule and Act Utilitarianism.

Jeremy Bentham, a famous English philosopher, proposed several attributes that should be taken into account when calculating the utility of an action \cite[75]{ethics}. The first attribute, \textit{Intensity}, is concerned with the magnitude of the experience. The magnitude of Alexis' experience can be measured in the amount of happiness she experienced. She was able to achieve her goal and even positively affected others around her; presumably she was able to save her family and herself a large amount of money while also bettering her own life in the process through a college education. Both of these aspects should be taken into account as contributing positively to the magnitude of the situation.

The next attribute, \textit{Duration}, applies greatly to Alexis' scenario. The duration of the experience extends far beyond the time that Alexis spent in the library on the computer; in fact, the education she received will continue to affect her for the rest of her life. However, Alexis stole the information from another student at the university. The duration of that action, coupled with the amount of time she spent using the stolen information, is still greatly overshadowed by the overall happiness that Alexis experienced. It is reasonable to say that the amount of happiness gained under the \textit{Duration} criterion is far greater than the amount of happiness lost.

Bentham mentions the idea of certainty in his criteria for testing actions under Act Utilitarianism. There was always a chance that Alexis' end goal could have never been met, in which case the overall unhappiness produced would have far outweighed the happiness. However, Alexis did ultimately achieve her goal to be accepted into college; on top of just being accepted into college, she was also provided with a full ride, which positively affected her parents as well. As a result, the attribute of \textit{Certainty} greatly adds to the overall happiness gained from Alexis' decision.

The final attribute that Bentham proposes is concerned with the number of people affected by the action. Alexis' actions do in fact affect several other beings, and not all the interactions are not necessarily positive. As mentioned before, Alexis' family benefitted greatly from the situation. They were already poor; this fact may have greatly affected Alexis' ability to attend college. However, due to her actions, Alexis was able to obtain a scholarship that provided her with the funds to attend college for free. In this way, Alexis affected, in a positive way, a large number of people that were important to her. However, she did also affect the student from whom she stole the login information. Her actions had a slight negative affect on that student, but not enough of a negative affect to outweigh the positive affect she had on her family.

Bentham's methods applied to Alexis' actions under the ethical theory of Act Utilitarianism. Under Act Utilitarianism, Alexis has made an ethically good decision; the amount of overall happiness produced by her actions far outweighs the unhappiness produced. However, many of the calculations are not exact since the certainty of the consequences of Alexis' actions were often assumed \cite[77]{ethics}. The theory of Rule Utilitarianism can also provide a viewpoint to determine whether or not Alexis' actions were ethically right or wrong.

Rule Utilitarianism, much like Kantianism, is based in rules that should be morally followed. Unlike Kantianism, however, Rule Utilitarianism adopts rules based on the total overall universal happiness that would result from adopting said rule. For the most part, the outcome of Alexis' situation is the same under both Rule and Act Utilitarianism. The main moral dilemma that Alexis faces is whether or not to steal and use the student's information. Rather than calculating the increase in her own happiness, Alexis must now calculate the impact her decision will have if the rule were to be adopted universally. Simply put, Alexis must decide whether or not it is universally okay to steal private information for one's own personal gain. As a universal rule, it is obvious that theft of private information should not be universally accepted as alright. Therefore, Alexis' decision, under Rule Utilitarianism, was morally wrong.

Both Rule and Act Utilitarianism provide interesting views on the ethical decision that Alexis made. Under Rule Utilitarianism, Alexis' decision was ethically wrong; under Act Utilitarianism, her decision was ethically right. Both of the methods used to analyze the situation provide a method to determine whether or not Alexis made the right choice; nether Rule nor Act Utilitarianism are absolute.

\newpage
\printbibliography

\end{document}
