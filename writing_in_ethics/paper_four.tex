\documentclass{paper}
\title{The Code of Ethics}

\addbibresource{pfour.bib}

\begin{document}
\finalh{}
\inlinetitle

\begin{linenumbers}

The ACM code of ethics provides those in technical industries with guidelines to follow when making tough ethical decisions. One such ethical decision arose in January of 1986 when NASA's Challenger space shuttle exploded above Cape Canaveral due to the neglect of suggestions from various engineers who were part of the team. One engineer, Roger Boisjoly, testified against NASA and the the engineering company Morton Thiokol during the trial, despite the recommendations given Morton Thiokol's lawyers. Although Boisjoly broke some of the guidelines of the ACM code of Ethics, his actions were not ultimately unethical since he took steps to prevent the launch and remained honest throughout the resulting court trials.

The explosion of the Challenger space shuttle could have been prevented. Boisjoly attempted to warn NASA and the other engineers multiple times. The Code of Ethics explicitly states that one should avoid harm to others. In Boisjoly's letter to the vice president of engineering, he only noted that the O-ring failure would result in damages to the shuttle and the launch pad, but not the loss of human life. It is possible to argue that Boisjoly did not stress the seriousness of the situation enough to convince his managers to take action. In this regard, Boisjoly did not completely comply with the Code of Ethics. The Code notes
\begin{quote}
Well-intended actions...may lead to harm unexpectedly. One way to avoid unintentional harm is to carefully consider potential impacts on all those affected by decisions made during design and implementation.
\end{quote}
By only stressing the damage that could be done to the property, and not to human life, Boisjoly violated the code of ethics.

Despite this pseudo-violation, Boisjoly did uphold many of the other guidelines of the Code of Ethics. Arguably one of the most important ideas outlined by the Code is that one should be honest and trustworthy. Despite the pressure from his peers who did not listen or acknowledge his warnings, he continued to warn NASA and Morton Thiokol about the dangers of launching the shuttle. By not accepting the faulty device, Boisjoly was upholding the ACM Code. In addition, Boisjoly could have easily gone to the press and released details about the O-ring and NASA's carelessness, but he did not. Instead, he worked to tell others inside the company about what was happening. The ACM Code states that one should honor confidentiality, which Boisjoly did by not whistle-blowing on the company until the time came for him to testify in court. One can argue that it would have actually been ethical to make the information public, which would have put pressure on NASA and Morton Thiokol to reconsider the launch. Boisjoly's warnings may have been in vain regardless, as the issues were not heavily considered until the night before the launch. In addition, NASA prevented the communication of the issues to anyone despite the situation, so it may not have been possible for Boisjoly to make the issue public even if he had gone to an outside source.

When Boisjoly testified in court, he told the jury about the faulty O-ring inside the rocket. This action can be seen as ethical under the ACM Code. If Boisjoly had omitted the details and obeyed the orders given to him by Morton Thiokol's lawyers, he would have been violating the Code in several ways. One aspect of the Code of Ethics is that one should provide appropriate professional review. Boisjoly had already done this numerous times before the Challenger shuttle was destroyed. During his testimony, he presented the same information that was already known by the managers at NASA and Morton Thiokol. In addition, if Boisjoly had not told the panel what he knew, then he would have been omitting the truth. The Code of Ethics states that a professional must be honest about circumstances that may cause a conflict of interest. Omission of the truth, especially when dealing with loss of human life, would have certainly been a conflict of interest. One can argue that by not following the advice of Morton Thiokol's lawyers was a violation of the Code of Ethics, since he did not keep engineering work confidential. The Code, however, states that the obligation to keep work confidential can be changed when the law or other sections of the Code require that the information be divulged.

Overall, Boisjoly's actions were not completely in violation of the ACM Code of Ethics. While he did not fully consider the implications of his actions, he did everything he could to act ethically given the circumstances. Ultimately, Boisjoly left his job at Morton Thiokol due to pressure related to his actions during the trial, but his actions ensured that a disaster of this magnitude could never take place again. By following the ACM Code of Ethics, Boisjoly was able to act ethically and make decisions that will help others in the future.
\end{linenumbers}
\newpage

\nocite{*}
\printbibliography

\end{document}
