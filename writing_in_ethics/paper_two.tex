\documentclass{paper}

\addbibresource{generic_bib.bib} \addbibresource{paper_two_bib.bib}

\title{Kantian Patriots} \begin{document} \drafth{} \inlinetitle \begin{linenumbers}

Governmental power has drastically increased over the past decade. Much of the growth of power is
due to acts of terrorism that have greatly impacted the nation. One of the most devastating acts of
terrorism that the United States has seen is the attack on the World Trade Centers on September 11,
2001. As a result, the government passed the USA PATRIOT act, also known as the Patriot Act. Many of
the provisions in the Patriot Act grant the government the ability to avoid due process. Under the
Patriot Act, the government is given excessive power to bypass the law. Kantian ethics provide an
insight that can help determine whether or not use of the liberties granted to the government are
ethically sound.

Section 6.6.6 of the text presents a situation in which a British news magazine, known as
\textit{The Guardian}, published an article describing how the FBI indirectly requested that Verizon
turn over all of its phone records. A key tenant of Kantian ethics is that others should never be
used as a means to an end. The FBI's intention was never revealed, so it is hard to say whether or
not they were using the phone records for personal gain. However, this fact alone is a breach of
Kantian ethics; the FBI wanted to have access to private information, but refused to divulge the
reason. Under the first Categorical Imperative, moral actions should be able to be applied
universally. The FBI did not release their private information, so, under the Categorical
Imperative, they should not have taken others private information.

The power over communications granted to the FBI is greatly increased by the Patriot Act as well.
Section 6.6.2 discusses the use of National Security Letters (NSLs), namely in regards to the
Library Connection of Connecticut. In this scenario, the FBI used a provision of the Patriot Act to
demand a user's browser history from a library computer. It follows from the argument made about the
FBI requesting Verizon phone records that NSLs are a breach of Kantian ethics. the FBI prohibits
individuals and companies that receive NSL's to reveal that the FBI had contacted them in the first
place. The Categorical Imperative states that actions should be willed to universal laws, so it is
unethical to demand private information and to demand that their private information not be given
out. This scenario differs from the Verizon scenario in that the FBI wanted the information of only
one individual. The use of the NSL aside, the FBI's actions are still unethical under Kantian
analysis. The agency requested that the Library Connection hand over information about another
individual. In doing so, the FBI was using the Library Connection as a means to an end, which
violates the Categorical Imperative. In this scenario, as well as the last, it is difficult to
completely analyze the situations using Kantian analysis, since the FBI's intention is not actually
know.

The third scenario presented by the text involves Brandon Mayfield, an attorney from Portland,
Oregon. The FBI was presented with a partial fingerprint recovered from a terrorist attack in Spain
that was linked back to Mayfield. The FBI then proceeded to investigate Mayfield, going as far as
entering his home and seizing private information without proper legal authorization. This scenario
differs from others in that the FBI's actions were directly aimed at Mayfield. They did not use
anyone as a means to an end, as they did in other scenarios. However, they still violated Kant's
Categorical Imperative. Kantianism focuses on the idea that moral laws should be able to be willed
into universal laws, and as such, no party involved should be able to opt-out. Following this logic,
Mayfield, as well as anyone else who wanted to, should be able to take information without legal
authorization as the FBI did. This scenario differs from the previous two in that the FBI's
intention is known. The agency sought to bring a suspected terrorist into custody. The methods used
to reach that goal were not ethical however, which makes their actions unethical.

It appears that under Kantian analysis, many of the actions taken by the FBI are unethical. The
Patriot Act allows for the government to interfere with the lives of the American population with
little regard for the law, but the legality of an action does not necessarily determine whether that
action is ethically right or wrong. It is hard to say whether or not the Patriot Act itself is
ethical or not; Kantian ethics do however focus on the morality of a decision. The Patriot Act was
put in place to protect the citizens of the United States, which Kant would argue is a morally right
cause. Many of the actions taken by the FBI in the name of the Patriot Act were presumably done for
the same reason, but there is no way to know for certain. \end{linenumbers} \newpage \nocite{*}
\printbibliography \end{document} 
