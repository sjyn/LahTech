\documentclass{article}
\usepackage{amssymb,amsmath}
\usepackage{mathtools}
\usepackage{array}
\topmargin=-0.3in
\textheight=9.2in
\textwidth=168mm
\oddsidemargin=-0.2in
\evensidemargin=-0.2in

\newcommand{\dy}{\mathop{dy}}
\newcommand{\dx}{\mathop{dx}}
\newcommand{\ddxx}{\mathop{dx^{2}}}
\newcommand{\ddyy}{\mathop{d^{2}y}}
\newcommand{\dydx}{\frac{\dy}{\dx}}
\newcommand{\ptl}{\partial}
\newcommand{\Wr}{\mathcal{W}}
\newcommand{\Interval}{\mathcal{I}}

\title{Methods For Solving ODE's}
\date{}
\begin{document}
\maketitle
\section{First Order Equations}
\hrule
\noindent\\\\
\begin{minipage}[t]{0.5\textwidth}
\subsection{Separable Equations}
\begin{gather*}
\frac{\dy}{\dx} = g(x)h(y)\\
\frac{\dy}{h(y)} = g(x)\dx\\
\int \frac{\dy}{h(y)} = \int g(x)\dx\\
\end{gather*}
\end{minipage}
\begin{minipage}[t]{0.5\textwidth}
\subsection{The Integrating Factor}
\begin{gather*}
\dydx + \rho(x)y(x) = q(x)\\
\dydx e^{-\int \rho(x)\dx} + \rho(x)y(x)e^{-\int \rho(x)\dx} = q(x)e^{-\int \rho(x)\dx}\\
\frac{d}{\dx}\left(y(x)e^{-\int \rho(x)\dx}\right) = q(x)e^{-\int \rho(x)\dx}\\
\int\frac{d}{\dx}\left(y(x)e^{-\int \rho(x)\dx}\right) = \int q(x)e^{-\int \rho(x)\dx}
\end{gather*}
\end{minipage}
\subsection{Exact Differential Equations}
\begin{gather*}
M(x,y)\dx + N(x,y)\dy = 0\\
\frac{\ptl M}{\ptl y} = \frac{\ptl N}{\ptl x} \implies 
\begin{cases}
M = \frac{\ptl F}{\ptl x}\\
N = \frac{\ptl F}{\ptl y}
\end{cases}\\
F(x,y) = \int \frac{\ptl F}{\ptl x}\dx\\
\text{A function $g(y)$ will be left over; use $N$ to solve for it by $N = \frac{\ptl F}{\ptl y}$}
\end{gather*}
\newpage


\section{Second Order Equations}
\hrule
\noindent\\\\
\subsection{The Wronskian and Linear Independence}
\begin{enumerate}
\item If $\Wr(x,y)(x_{0})$ for some $x_{0}$ in an interval $\Interval$ is \textit{\textbf{not}} equal to 0, then the two functions are linearly independent.
\item If $f(x)$ and $g(x)$ are linearly dependent on $\Interval$, the $\Wr(f,g)(x) = 0$ for all $x$ in $\Interval$.
\item If $f$ and $g$ are functions of the same variable, then 
\[
\Wr(f,g)(x) = \left|
\begin{array}{c c}
f(x) & g(x)\\
f'(x) & g'(x)
\end{array}
\right|.
\]
\item Given that $y_{1}$ and $y_{2}$ are two solutions to $y'' + p(x)y' + q(x)y = 0$,
\[
\Wr(y_{1},y_{2})(x) = ce^{-\int p(x)\dx}.
\]
\end{enumerate}
\subsection{Algebraic Methods}
\subsubsection*{Homogeneous}
\[
ay'' + by' + cy = 0\qquad\qquad\text{becomes}\qquad\qquad ar^{2} + br + c = 0
\]
\indent The roots of $r$ determine the formal solution to the ODE\\\\
\begin{minipage}[t]{0.3\textwidth}
\underline{Two distinct roots in $\mathbb{R}$}
\begin{gather*}
y_{1}(x) = e^{r_{1}x}\\
y_{2}(x) = e^{r_{2}x}\\
y(x) = c_{1}e^{r_{1}x} + c_{2}e^{r_{2}x}
\end{gather*}
\end{minipage}
\begin{minipage}[t]{0.3\textwidth}
\underline{Two identical roots in $\mathbb{R}$}
\begin{gather*}
y_{1}(x) = e^{rx}\\
y_{2}(x) = xe^{rx}\\ 
y(x) = c_{1}e^{rx} + c_{2}xe^{rx}
\end{gather*}
\end{minipage}
\begin{minipage}[t]{0.3\textwidth}
\underline{Two conjugate roots in $\mathbb{C}$}
\begin{gather*}
r_{1} = \alpha + i\beta\\
r_{2} = \alpha - i\beta\\
y(x) = e^{\alpha x}(c_{1}\cos{(\beta x)} - c_{2}\sin{(\beta x)})
\end{gather*}
\end{minipage}
\subsection{The Method of Undetermined Coefficients}
Given $y'' + p(x)y' + q(x)y = g(x)$, we know that $y(x)$ is the sum of the homogeneous solution and the non-homogeneous solution of the ODE.
\end{document}