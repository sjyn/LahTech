\documentclass{article}
\usepackage{amsmath, amssymb}
\usepackage[margin=1in]{geometry}
\usepackage{setspace}
\usepackage{graphicx}

\doublespacing

\title{The R\"{o}ssler System}
\author{Steven Rosendahl}
\date{}

\begin{document}
\maketitle

In 1976, Otto R\"{o}ssler proposed a system of nonlinear ordinary differential equations that illustrated the simplest possible strange attractor. An attractor is a set of values towards which a system moves when initial conditions are \textit{near} the attractor; to call an attractor strange means that the attractor exhibits fractal behavior. Often, strange attractors are associated with chaotic systems. R\"{o}ssler's strange attractor is a chaotic attractor that solves his proposed system
\begin{align}
    \dot{x} &= -y-z\\
    \dot{y} &= x+ay\\
    \dot{z} &= b+z(x-c)
\end{align}
where $a,b,$ and $c$ are arbitrary values. R\"{o}ssler studied the effects that small (i.e. less than 1) $a$ and $b$ paired with a relatively large $c$ had on the system's chaotic behavior.


\end{document}
