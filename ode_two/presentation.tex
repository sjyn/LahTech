\documentclass{beamer}
\usepackage{amsmath, amssymb, amsthm}
\usepackage{graphicx}

\title{The R\"{o}ssler System}
\author{Steven Rosendahl}
\date{}

\usetheme{Rochester}
\usecolortheme{whale}

\begin{document}

\begin{frame}
    \titlepage
\end{frame}

\begin{frame}{Introduction}
    \begin{itemize}
        \item Created by Otto R\"{o}ssler in 1976
        \pause
        \item System constructed to display simplest possible \textit{Strange Attractor}
        \pause
    \end{itemize}
    \begin{definition}
        An \alert{attractor} is a set of values towards which a system moves when initial conditions are \textit{near} the attractor.
        \pause
        An attractor is called \alert{strange} if it exhibits fractal behavior.
    \end{definition}
    \pause
    \begin{itemize}
        \item Strange attractors are often associated with chaotic systems.
    \end{itemize}
\end{frame}

\begin{frame}{System}
    \begin{align*}
        \dot{x}&= -y-z\\
        \dot{y}&= x+ay\\
        \dot{z}&= b+z(x-c)
    \end{align*}
    \pause
    \begin{itemize}
        \item $a,b,c$ are responsible for attractive behavior
        \pause
        \item Want to analyze the effect small $a$ and $b$ have on system
    \end{itemize}
\end{frame}

\end{document}
