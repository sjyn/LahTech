\documentclass[8pt]{article}
\usepackage{extsizes}
\usepackage{enumitem}
\usepackage{amsmath}
\setlist{nosep}
\usepackage[margin=0.2in]{geometry}

\begin{document}
\section*{Test 2 Review}
\begin{enumerate}
    \item Who is Ahura Mazda? Who is Angra Mainyu?
    \begin{enumerate}
        \item Ahura Mazda
        \begin{itemize}
            \item Creator God in Zoroastrianism
            \item Sole God of the religion (Zoroaster believed in one God)
            \item Followers worship his attributes (\textit{Amesha Spentas})
            \item \textit{Amesha Spentas} are good spirits
            \item Had evil twin spirit named Angra Mainyu
            \item Truth teller; truth was very important to Zoroastrianism
        \end{itemize}
        \item Angra Mainyu
        \begin{itemize}
            \item Evil counterpart to Ahura Mazda
            \item created Devas - evil spirits that paralleled \textit{Amesha Spentas}
            \item Attempted to lead Zoroaster astray
            \item Can also create things (evil is a creative force)
            \item Lier and deceiver (similar to Satan)
        \end{itemize}
    \end{enumerate}

    \item List some similarities between Zoroastrianism, Christianity, and post-exile Judaism.
    \begin{itemize}
        \item Central God; Zoroastrianism: Ahura Mazda, Judaism/Christianity: Yahweh
        \item Morality comes from God
        \item Dualism - contest between good an evil (only some sects of Judaism)
        \item Some sects of Judaism look for Messiah; Christianity has Jesus; Zoroastrianism looked for Saoshyant
        \item Coming of God - 1000 years
        \item Similar eschatology (final judgment, raising of the dead)
        \item Devil figure (Angra Mainyu/Satan)
    \end{itemize}

    \item Discuss what the church said about Jesus Christ at Nicea and Chalcedon.
    \begin{enumerate}
        \item Nicea
        \begin{itemize}
            \item First ecumenical council of the Church
            \item Needed to establish doctrine to unite Christians
            \item Hosted by Constantine
            \item Athanasius proposed homoousios
            \begin{itemize}
                \item Nature of Father and Son was very important topic
                \item Father and son have same nature but are not the same
                \item Father and son share the same essence.
            \end{itemize}
            \item Arius: Jesus was separate from God
            \begin{itemize}
                \item Jesus was first creation of God
                \item Jesus was not divine
                \item Rejected by council
            \end{itemize}
            \item Difference between \textit{made} (Arius) and \textit{begotten} (Athanasius)
            \item Trinity idea proposed (later rejected by enlightenment Christians)
        \end{itemize}
        \item Chalcedon
        \begin{itemize}
            \item Council met to settle disputes over Jesus' nature (divine vs. human)
            \item Greek ideas (God is separate) clashing with Hebrew ideas (God is with humanity)
            \item Leo's Tome: Jesus has two full natures in one person (Orthodox position)
            \item Solution proposed by Leo (bishop of Rome)
            \begin{itemize}
                \item Decided Jesus was fully God an fully man
                \item United in one person
                \item Hypostatic union
                \item Gave Leo position of bishop of bishops
            \end{itemize}
            \item Established Mary's position in the church (Humans < Mary < God)
            \item Gave Mary the status of mother of God
        \end{itemize}
    \end{enumerate}

    \item Describe the evolution of the Papacy.
    \begin{itemize}
        \item Peter considered to be head of Church after Jesus' death
        \item Bishops are established throughout the known world
        \item Chalcedon: Bishop of Rome (Leo) gains authority in establishing Doctrine
        \item Gregory the Great is the first recognized Pope
        % \item Innocent III and Gregory VII
        % Innocent III and Gregory VII, nation states, loss of papal power, Vatican I and II
        \item Pope reigns above kings in the middle ages
        \item Holy Roman empire formed by Constantine
        \item Church of England splits from the Catholic church over disputes (power of Pope weakened)
        \item Other theologians begin to reject Church's teaching (Martin Luther, John Calvin, etc.) further weakening the Pope's authority
    \end{itemize}

    \item What do Pelagius, Augustine, Erasmus, and Calvin believe about God's sovereignty and human freedom?
    \begin{enumerate}
        \item Pelagius
        \begin{itemize}
            \item Humans have free will
            \item Good deeds will lead you to heaven
            \item Possibility to serve God
            \item Did not believe in God's saving grace
            \item Reacted against Augustine's teachings
        \end{itemize}
        \item Augustine
        \begin{itemize}
            \item Humans were predestined
            \item Give what you command and command what you will
            \item God selects certain people to receive his grace
            \begin{itemize}
                \item Not all who receive grace are selected (can lose grace)
            \end{itemize}
            \item Human beings are totally depraved (only God's grace can save)
            \item Only those who have been given the grace of God can follow
            \item Works on Earth do not matter
        \end{itemize}
        \item Erasmus
        \begin{itemize}
            \item Grace of God and works lead to salvation
            \item God's grace is far above the good works (do good through God's grace)
            \item Humans have free will to follow or reject God
            \item Strong focus on working with God
            \item Followed by Catholic church
            \item Salvation can be lost/reacquired
        \end{itemize}
        \item Calvin
        \begin{itemize}
            \item Humans have no free will (impunes sovereignty of God)
            \begin{itemize}
                \item Double predestination
                \item Some are chosen for Heaven, others for Hell
            \end{itemize}
            \item Salvation comes \textbf{only} from the grace of God
            \item Humans are inherently tainted with sin (total depravity)
            \begin{itemize}
                \item Every aspect is tainted with iniquity
            \end{itemize}
            \item Strong emphasis on God's role in human life
        \end{itemize}
    \end{enumerate}

    \item Explain the following theories of atonement: Penal Substitution, Classical Redemption, Acceptation Theory, and Moral Influence
    \begin{enumerate}
        \item Penal Substitution
        \begin{itemize}
            \item Jesus was punished for sins of humanity
            \item God forgives all sins through this sacrifice
        \end{itemize}
        \item Classical Redemption
        \begin{itemize}
            \item
        \end{itemize}
        \item Acceptation Theory
        \begin{itemize}
            \item
        \end{itemize}
        \item Moral Influence
    \end{enumerate}

    \item Differentiate the Catholic, Lutheran, Reformed, and Anabaptist theories of the sacraments (baptism and eucharist)
    \begin{enumerate}
        \item Catholic
        \begin{itemize}
            \item Baptism of infants
            \item Original Sin is forgiven through baptism
            \item Eucharist is \textit{Transubstantiation}: transforming into literal body and blood
            \item Eucharist is central focus of the mass (sacrifice)
        \end{itemize}
        \item Lutheran
        \begin{itemize}
            \item Baptism of infants
            \item Original Sin is forgiven through baptism
            \item Eucharist is \textit{Consubstantiation}: elements are the same but Christ is present in them
        \end{itemize}
        \item Reformed
        \begin{itemize}
            \item Baptism of infants
            \item Baptism is an oath to bring up child in name of Christ
            \item Baptism is Symbol of the Covenant
            \item Eucharist is symbolic of Christ; used to remember his sacrifice
        \end{itemize}
        \item Anabaptist
        \begin{itemize}
            \item Baptism of adults
            \item Baptism is symbolic of conversion and acceptance to Christ
            \item Eucharist is symbolic of Christ; used to remember his sacrifice
        \end{itemize}
    \end{enumerate}

    \item Discuss the religious and secular heritage of this country
    \begin{itemize}
        \item Puritans had democratic structure in churches

    \end{itemize}

    \item Explain the following: Mass, The Council of Trent, Franciscans, Jesuits, Immaculate Conception, Papal Infallibility, and Vatican II
    \begin{enumerate}
        \item Mass
        \begin{itemize}
            \item Religious service for Catholics
            \item Focused mainly on Eucharist
            \item Was traditionally performed in Latin or Greek; later moved to vernacular
        \end{itemize}
        \item Council of Trent
        \begin{itemize}
            \item Council to make certain practices of the church into official dogma
            \item Cleared up disputes between different sects of Catholicism
        \end{itemize}
        \item Franciscans
        \begin{itemize}
            \item Order of monks founded by Francis of Assisi
            \item Live very simple lives
        \end{itemize}
        \item Jesuits
        \begin{itemize}
            \item
        \end{itemize}
        \item Immaculate Conception
        \begin{itemize}
            \item Involves Mary's conception of Jesus
            \item Catholics believe this leaves Mary sinless
        \end{itemize}
        \item Papal Infallibility
        \begin{itemize}
            \item Pope is direct line to God
            \item Pope's decrees are final
        \end{itemize}
        \item Vatican II
        \begin{itemize}
            \item Sought to bring the Church in to the modern age
            \item Discussed relationships with other religions
        \end{itemize}
    \end{enumerate}

    \item Who are the Methodists, Mormons, Quakers, and Fundamentalists?
    \begin{enumerate}
        \item Methodists
        \begin{itemize}
            \item Founded by John Wesley
            \item Theology focuses on the relationship between faith and character
            \item Focus on works that lead to salvation
        \end{itemize}
        \item Mormons
        \begin{itemize}
            \item Founded by Joseph Smith
            \item Believe their leader is a prophet of God
            \item Believe that Jesus came to America
        \end{itemize}
        \item Quakers
        \begin{itemize}
            \item Focus on finding God in each person
            \item Live removed lives
            \item Very simple dress/lifestyle
        \end{itemize}
        \item Fundamentalists
        \begin{itemize}
            \item Reaction to modernization of the church
            \item Believe that many theologians had strayed from the true teaching
            \item Sought to reform teachings by referring to literal text in the Bible
        \end{itemize}
    \end{enumerate}

    \item Explain Hinduism and Taoism
    \begin{enumerate}
        \item Hinduism
        \begin{itemize}
            \item
        \end{itemize}
    \end{enumerate}
\end{enumerate}
\end{document}
