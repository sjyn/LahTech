\documentclass[12pt]{article}

\usepackage{setspace}
\usepackage[margin=1in]{geometry}

\title{Taoism}
\date{}

\newcommand{\inlinetitle}{\vspace{-1.5cm}{\let\newpage\relax\maketitle}\vspace{-1.5cm}}
\doublespacing

\begin{document}
\noindent Steven Rosendahl\\
\noindent Religions of the West\\
\inlinetitle

Taoism has been an incredibly influential religion in Chinese culture over the last several centuries. It's reach has been broad; several revolutions have been sparked by Taoist ideals, and many Chinese dynasties have claimed to share the lineage of Laozi, the Chinese philosopher responsible for writing down the Tao Te Ching. Taoism is unique in that it combines religion, philosophy, and myth. the Tao is considered to be the ultimate goal of Taoism, and texts such as the Tao Te Ching and the Chuang-tzu were written to be moral guides that teach the way to reach the Tao. Much of the history of Taoism involves tales of magic and immortality that are widely considered to be myth, and the religion does not focus on God in the way western religions do. The history of Taoism has been shaped by China just as much as Taoism has shaped the history of China.

Laozi is considered to be a sort of prophet, in the loosest sense of the word. Born at the age of 61, Laozi lived in a time when the Chou dynasty was crumbling, and much of China was divided into small territories all vying for power. During this period of turmoil, Laozi decided to leave the civilized area of China and live as a hermit in the wild. It was during this time that he wrote the Tao Teh Ching. According to legend, Laozi was leaving China through the western gate when a soldier stopped him. The soldier felt that Laozi had much to teach, and refused to let him exit until he had written his teachings down. Laozi saw no way out of the situation, and agreed to produce the Tao Teh Ching.

The Tao Teh Ching, a set of eighty-one cryptic verses, is considered to be the most popular text in Taoism. Many of the verses focus on how to better one's self, so that they may follow the Tao, which literally translates into \textit{Path}. Several very famous saying have originated from the Tao Teh Ching, such as:
\begin{quote}
We make a vessel from a lump of clay;\\
It is the empty space within the vessel that makes it useful.\\
We make doors and windows for a room;\\
But it is these empty spaces that make the room livable.\\
\textbf{$-$ Laozi, Tao Teh Ching, Ch. 11}\\
He who knows does not speak.\\
He who speaks does not know.\\
\textbf{$-$ Laozi, Tao Teh Ching, Ch. 56}
\end{quote}
The ideas in the Tao Teh Ching reflect Laozi's feelings towards the state of China at the time. The above quotation emphasizes the usefulness of emptiness during a time in which many of the feudal lords sought to have as much power and land as possible. The beginning lines of chapter 29 emphasize that mankind is not the ruler of the nature.
\begin{quote}
Does anyone want to take the world and do what he wants with it?\\
I do not see how he can succeed.\\
\textbf{$-$ Laozi, Tao Teh Ching, Ch. 29}
\end{quote}
Laozi's writings primarily focus on the importance of nature and the temporariness of life; he believed that power and possessions were not what the people of China should be focused on. The Tao Teh Ching is not the only sacred writing in the Taoist canon, however.

Another very famous set of writings is the Chuang-tzu, named after it's author. Chuang-tzu lived in the time of the Chou dynasty as well, although he was born after Laozi. During his younger years, Chuang-tzu experienced the horrors of the wars that were tearing apart China. Many of his writings were in response to the death and pain that he saw around him. One legend surrounding Chuang-tzu describes how, upon his wife's passing, he was found playing music and happily singing. When questioned, Chuang-tzu replied that he was happy for his wife's passing, as she had returned to nature and her death was only a loss from the human perspective. Chuang-tzu believed that the wants of humans were petty and ultimately ignored by the universe. The teachings of Laozi and Chuang-tzu are considered to be two sides to the same coin; both philosophers had a very different view on the Tao, yet both taught the same fundamental ideas.

Laozi and Chuang-tzu, while being incredibly influential in Taoist teaching, never considered themselves to be religious teachers; In Taoist mythology, Laozi was revered as a deity and Chuang-tzu was considered to be immortal. The first true religious teacher of Taoism is considered to be Chang Tao-Ling. According to legend, Tao-Ling had mastered the Tao Teh Ching as well as the Five Classics of Confucianism by the time he was seven. Taoist lore states that Tao-Ling was a magician, capable of turning invisible, teleporting, and cloning himself. When he came of age, Tao-Ling decided that he would search for the Tao; he abandoned the teaching of Confucius and set out to find the Tao. During his time in the mountains of China, he encountered the deified form of Laozi, who gave him the secret to immortality. Tao-Ling, however, saw this as a chance to help all of the people of China. He is credited with founding the first religious organization for Taoism, as well as being the first true teacher of the religion. Many of his teaching are still widely followed today.

One of the key ideas of Taoism is the relationship that humans have with nature. According the the teachings of Taoism, all humans came from nature, and, as such, should be one with nature during one's life. The Tao is meant to guide humans in being one with nature; being in tune with the Tao is being in tune with nature. Some very well known symbols, such as Yin-Yang and parts of the Chinese zodiac originate from Taoism. One must be well-balanced to stay with the Tao, which is reflected in the Yin-Yang symbol. The Tao itself is the central focus of Taoism. There is no one deity to which Taoists must bow; instead, they simply must seek out the Tao. In the Tao Teh Ching, Laozi describes the Tao by saying
``I know not its name, so I style it the way. I give it the makeshift name of the great."
Chuang-tzu described the Tao as an force that
``gave spirits and rulers their spiritual powers."
Interestingly, the Tao, while being the creator of all, is not immediately present in the world. While many religions present gods who involve themselves in mankind, the Tao is considered to just be; it does not impose itself on humanity, since, according to Chuang-tzu, humanity is too petty for it. Laozi explains this behavior by saying that ``The Tao nurtures her children by allowing them to be themselves, so that when they are grown, they will be able to stand on their own."
In other words, the Tao creates everything to be independent, without forcing its creations to follow it.

The concept of balance is very important to Taoism. Taoists believe that there is always another side to every event that transpires in the universe. For example, Taoists believe that the human world and the non-human world are connected so that events in one world impact the other. Many political leaders of China followed this principal and would attempt to please the non-human world in order to bring good fortune to their lives. The Tao is also considered to be \textit{self-balancing}, meaning that if the Tao was brought off balance, it would bring balance back by whatever means necessary. To avoid drastic changes in balance, followers of Taoism would seek out balance in life, and avoid doing anything considered to be extreme.
%The ideas of nature and balance are reflected in the teaching of the Five Elements.

%The Five Elements are a combination of balance, nature, and humankind. They are Wood, Fire, Earth, Metal, and Water, and each one is related to every other element. For example, Fire can destroy Metal, help the Earth, gain strength from Wood, and be weakened by Water. While the Yin-Yang symbol shows the duality of actions, Taoist belief states that it is not always evenly divided into one side or the other.
Taoism is a uniquely different religion from many of the religions of the west and the east. The Tao is not a God in the traditional sense; it is surrounded in myth, and does not interfere with the lives of humans. Humans can become closer to the Tao by searching for peace and balance in their lives. Taoism is not a religion of rules and guidelines, but rather a religion that focuses on bettering one's self, which will in turn lead to a better society. In this way, Taoism can almost be considered a philosophy by which to conduct one's self.

\end{document}
