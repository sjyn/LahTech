\documentclass{hw}

%page 7: 7,9,10,14,16,17,21,23

\begin{document}
\makeheader{1}
\begin{enumerate}
\item
Let $A$ be the point with coordinates $(1, 0, 2)$, let $B$ be the point with coordinates $(−3, 3, 1)$, and
let $C$ be the point with coordinates $(2, 1, 5)$.
\begin{enumerate}
\item Describe the vectors $\vv{AB}$ and $\vv{BA}$.
\begin{quote}
$\vv{AB}$ will be the vector from $(1,0,2)$ to $(−3, 3, 1)$, which is the same as the vector $(-4,3,-1)$.
\end{quote}
\item Describe the vectors $\vv{AC},\vv{BC}$, and $\vv{AC} + \vv{CB}$.
\begin{quote}
$\vv{AC}$ will be the vector from $(1,0,2)$ to $(2,1,5)$, which is the same as the vector $(1,1,3)$.\\
$\vv{BC}$ will be the vector from $(-3,3,1)$ to $(2,1,5)$, which is the same as the vector $(5,-2,4)$.\\
$\vv{CB} = (1,2,-4)$, so $\vv{AC} + \vv{CB} = (2,3,-1)$.
\end{quote}
\item Explain with pictures why $\vv{AC} + \vv{CB} = \vv{AB}$.
\vspace{3cm}
\end{enumerate}
\item If $(-12,9,z)+(x,7,-3) = (2,y,5)$, what are $x,y$, and $z$?
\begin{alignat*}{4}
& -12 + x = 2\qquad\qquad && 9 + 7 = y\qquad\qquad &&& z-3= 5\\
& x = 14 && y = 16 &&& z = 8
\end{alignat*}
\item What is the length of the vector $<3,1>$?
\begin{quote}
The length of the vector is $\sqrt{3^{2} + 1^{2}}$, or $\sqrt{10}$.
\end{quote}
\item Find the displacement vectors from $P_{1}$ to $P_{2}$, and sketch $P_{1}, P_{2}$, and
$\vv{P_{1}P_{2}}$.\\\\
\begin{minipage}{0.25\textwidth}
(a) $P_{1}(1,0,2)$, $P_{2}(2,1,7)$
$$\vv{P_{1}P_{2}} = (1,1,5)$$
\end{minipage}
\begin{minipage}{0.25\textwidth}
(b) $P_{1}(1,6,−1),P_{2}(0,4,2)$
$$\vv{P_{1}P_{2}} = (-1,-2,3)$$
\end{minipage}
\begin{minipage}{0.25\textwidth}
(c) $P_{1} (0,4,2), P_{2} (1,6,−1)$
$$\vv{P_{1}P_{2}} = (1,2,-3)$$
\end{minipage}
\begin{minipage}{0.25\textwidth}
(d) $P_{1}(3,1), P_{2}(2,−1)$
$$\vv{P_{1}P_{2}} = (-1,-2)$$
\end{minipage}

\vspace{3cm}
\item If $A$ is the point in $\RR^{3}$ with coordinates $(2,5,-6)$ and the displacement
vector from $A$ to $B$ is $(12,-3,7)$, what are the coordinates of $B$?
\begin{quote}
The vector $\vv{AB} = (12,-3,7) = (x_{B} - 2, y_{B} - 5, z_{B} + 6)$. Setting up a system of
equations yields
\begin{alignat*}{4}
& 12 = x_{B} - 2\qquad\qquad && -3 = y_{B} - 5 \qquad\qquad &&& 7 = z_{B} + 6\\
& x_{B} = 14 && y_{B} = 2 &&& z_{B} = 1
\end{alignat*}
Therefore $B = (14,2,1)$.
\end{quote}
\newpage
\item Suppose that you and your friend are in New York talking on cellular phones. You inform each other
of your own displacement vectors from the Empire State Building to your current position. Explain how
you can use this information to determine the displacement vector from you to your friend.
\begin{quote}
The Empire State Building acts as the origin. If you know your point ($A$) and your friend's point, ($B$),
you can calculate the displacement vector $\vv{AB}$, which will provide you with the displacement vector
from you to your friend by $\vv{AB} = (x_{b} - x_{a}, y_{b} - y_{a})$.
\end{quote}
\item
\begin{enumerate}
\item Let $\vv{a} = (2,0)$ and $\vv{b} = (1,1)$. For $0\leq s\leq 1$ and $0\leq t\leq 1$, consider the
vector $\vv{x} = s\vv{a} + t\vv{b}$. Explain why the vector $\vv{x}$ lies in the parallelogram
determined by $\vv{a}$ and $\vv{b}$.
\begin{quote}
For $\vv{x}$ to be in the parallelogram $\vv{a}\vv{b}$, $\vv{x}$ cannot be greater than $(3,1)$ and
cannot be less than $(0,0)$. $\vv{x}$ is defined as
\begin{align*}
\vv{x} &= s\vv{a} + t\vv{b}\\
&= (2s,0) + (t,t)\\
&= (2s + t, t),
\end{align*}
which at $s = 0$ and $t = 0$ still lies within the bound, and at $s = 1$ and $t=1$ still lies within
the bound.
\end{quote}
\item Now suppose that $\vv{a} = (2,2,1)$ and $\vv{b} = (0,3,2)$. Describe the set of vectors
$\{\vv{x} = s\vv{a} + t\vv{b} | 0\leq s\leq 1, 0\leq t\leq 1\}$.
\begin{quote}
The set of vectors will be in the form of
\[
(2s, 2s+3t, s+2t)
\]
\end{quote}
\end{enumerate}
\item A flea falls onto marked graph paper at the point $(3,2)$. She begins moving from that point with
velocity vector $\vv{v} = (−1, −2)$ (i.e., she moves 1 graph paper unit per minute in the negative
x-direction and 2 graph paper units per minute in the negative y-direction).
\begin{enumerate}
\item What is the speed of the flea?
\begin{align*}
\text{speed} &= \sqrt{1 + 4}\\
&= \sqrt{5}
\end{align*}
\item Where is the flea after 3 minutes?
\begin{quote}
The flea's position can be expressed as $f(t) = (3-t,2-2t)$. Plugging in 3 for $t$ yields $(0,4)$.
\end{quote}
\item How long does it take the flea to get to the point $(-4,-12)$?
\begin{quote}
We have the equation $f(t) = (3-t,2-2t)$. We setting it equal to $(-4,-12)$ and solving for $t$ yields
$t = 7$.
\end{quote}
\item Does the flea reach the point $(−13, −27)$? Why or why not?
\begin{quote}
Using the equation above, we can solve again for $t$. Doing so yields $t=16$ and $t=14.5$, which is
impossible. Therefore the flea does not reach that point.
\end{quote}
\end{enumerate}
\end{enumerate}
\end{document}
