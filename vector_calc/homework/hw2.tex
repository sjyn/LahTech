\documentclass{hw}

% 11,13,17,19,23,25,27,31,35,37,42

\begin{document}
\makeheader{2}

\begin{enumerate}
\item Let $\vv{a_{1}} = (1,1)$ and $\vv{a_{2}} = (1,-1)$.
\begin{enumerate}
\item Write the vector $\vv{b_{1}} = (3, 1)$ as $c_{1}a_{1} + c_{2}a_{2}$, where
$c_{1}$ and $c_{2}$ are appropriate scalars.
\begin{align*}
(3,1) = (c_{1} + c_{2}, c_{1} - c_{2})\\
\left[
\begin{array}{c c | c}
1 & 1 & 3\\
1 & -1 & 1
\end{array}
\right] \to
\left[
\begin{array}{c c | c}
1 & 0 & 2\\
0 & 1 & 1
\end{array}
\right]\\
\vv{b} = 2(1,1) + 1(1,-1)
\end{align*}

\item Do the same for $(3,-5)$.
\begin{align*}
(3,1) = (c_{1} + c_{2}, c_{1} - c_{2})\\
\left[
\begin{array}{c c | c}
1 & 1 & 3\\
1 & -1 & -5
\end{array}
\right] \to
\left[
\begin{array}{c c | c}
1 & 0 & -1\\
0 & 1 & 4
\end{array}
\right]\\
\vv{b} = -1(1,1) + 4(1,-1)
\end{align*}

\item Show that any vector $\vv{b} = (b1, b2)$ in $\RR^{2}$ may be written in the form $c_{1} a_{1} +
c_{2} a_{2}$ for appropriate choices of the scalars $c_{1}, c_{2}$.
\begin{quote}
Assume that we can express any vector $(x_{1},x_{2})$ as $(c_{1} + c_{2}) + (c_{1} - c_{2})$, where
$x_{1},x_{2}\in\RR$. Then the solution for the corresponding matrix is
\[
\left[
\begin{array}{c c | c}
1 & 0 & \frac{x_{1} + x_{2}}{2}\\
0 & 1 & \frac{x_{1} - x_{2}}{2}
\end{array}
\right].
\]
Since $\frac{x_{1} + x_{2}}{2}$ and $\frac{x_{1} - x_{2}}{2}$ are both in $\RR$, any vector of the form
$(x_{1},x_{2})$ can be written as a linear combination of $\vv{a_{1}}$ and $\vv{a_{2}}$.
\end{quote}
\end{enumerate}

\item Write the following as a set of parametric equations: The line in $\RR^{3}$ through the point $(2,
−1, 5)$ that is parallel to the vector $i + 3j - 6k$.
\[
\begin{cases}
x = t+2\\
y = 3t - 1\\
z = 5 - 6t
\end{cases}
\]
\item Write the following as a set of parametric equations: The line in $\RR^{3}$ through the points
$(1,4,5)$ and $(2,4,1)$.
\[
\begin{cases}
x = t + 1\\
y = 4\\
z = 5 - 6t
\end{cases}
\]
\item Write the following as a set of parametric equations: Write a set of parametric equations for the
line in $\RR^{4}$ through the point $(1, 2, 0, 4)$ and parallel to the vector $(−2, 5, 3, 7)$.
\[
\begin{cases}
x_{1} =1-2t\\
x_{2} =5t+2\\
x_{3} =3t\\
x_{4} =7t+4
\end{cases}
\]

\item Give a symmetric form for the line having parametric equations $x = t + 7$, $y = 3t - 9$, $z = 6 -
8t$.
\[
\begin{cases}
x = t + 7\\
y = 3t - 9\\
z = 6-8t
\end{cases}\to
\begin{cases}
t = x - 7\\
t = \frac{y}{3} + 3\\
t = \frac{6 - z}{8}
\end{cases}
\]

\item Give a set of parametric equations for the line with symmetric form
\[
\frac{x + 5}{3} = \frac{y - 1}{7} = \frac{z + 10}{-2}.
\]
\[
\begin{cases}
t = \frac{x + 5}{3}\\
t = \frac{y - 1}{7}\\
t = \frac{z + 10}{-2}
\end{cases}\to
\begin{cases}
x = 3t - 5\\
y = 7t + 1\\
z = -2t -10
\end{cases}
\]

\item Show that the two sets of equations
\[
\frac{x-2}{3} = \frac{y-1}{7} = \frac{z}{5} \text{ and } \frac{x+1}{-6} = \frac{y+6}{-14} =
\frac{z+5}{10}
\]
represent the same line in $\RR^{3}$.
\begin{quote}
If we move to parametric coordinates, we have
\[
\begin{cases}
x = 3t + 2\\
y = 7t + 1\\
z = 5t
\end{cases}
\text{ and }
\begin{cases}
x = -6t - 1\\
y = -14t - 6\\
z = -10t - 5
\end{cases}.
\]
If we take the second vector and scale it by $-\frac{1}{2}$, we get
\[
\begin{cases}
x = 3t + \frac{1}{2}\\
y = 7t + 3\\
z = 5t = \frac{5}{2}
\end{cases}.
\]
We can now replace $t$ with an arbitrary linear combination of $t$ that we will call $s$. We can now
rewrite the vector in terms of $s$ as such:
\[
\begin{cases}
x = 3s + 2\\
y = 7s + 1\\
z = 5s
\end{cases}.
\]
This vector lies along the same line described by the original equation.
\end{quote}

\item Do the parametric equations $x = 5t^{2} - 1, y = 2t^{2} + 3, z = 1 − t^{2}$ determine a line?
Explain?
\begin{quote}
No, the given parametric equations do not form a line. If we represent the equations in symmetric
form, they will have nonlinear terms.
\end{quote}

\item Find the points of intersection of the line $x = 2t − 3, y = 3t +2, z = 5-t$ with each of the
coordinate planes $x = 0, y = 0$, and $z = 0$.
\begin{quote}
The line intersects the plane at $(\frac{3}{2},-\frac{2}{3},5)$.
\end{quote}

\item Does the line $x = 5-t, y=2t-3, z=7t+1$ intersect the plane $x - 3y+z=1$? Why?
\begin{quote}
Substituting in our values for $x,y,z$ into the equation of the plane yields
\[
(5-t) - 3(2t-3) + (7t+1) = 2.
\]
Solving for $t$ gives us
\[
-3 = 2,
\]
which is impossible. Therefore, the line does not intersect the plane.
\end{quote}

\item Find the point of intersection of the two lines $l_{1}: x = 2t + 3,y = 3t + 3,z = 2t + 1$ and
$l_{2}: x = 15 - 7t, y = t - 2, z = 3t - 7$.
\begin{quote}
To find the point of intersection, we can set the values in each equation equal to each other.
\[
\begin{cases}
2s + 3 = 15 - 7t\\
3s + 3 = t - 2\\
2s + 1 = 3t - 7
\end{cases}.
\]
We only have two unknowns, so we only need two equations:
\begin{align*}
\begin{cases}
2s + 7t &= 12\\
2s - 3t &= 8
\end{cases}.
\end{align*}
Solving the system yields $t = 2$. Therefore the two lines intersect at $t = 2$.
\end{quote}
\end{enumerate}
\end{document}
