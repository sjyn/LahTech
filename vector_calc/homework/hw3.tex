\documentclass{hw}
\begin{document}
\makeheader{3}

\begin{enumerate}
\item Compute $a\cdot b$, $||a||$ and $||b||$ where $\vv{a} = 4i -3j + k$ and $\vv{b}=i+j+k$.
\begin{quote}
\begin{align*}
a\cdot b &= 4 - 3 + 1\\
&= 2\\
||a|| &= \sqrt{16+9+1}\\
&= \sqrt{26}\\
||b|| &= \sqrt{1+1+1}\\
&= \sqrt{3}
\end{align*}
\end{quote}

\item Find the angle between $\vv{a} = \sqrt{3}\ i + j$ and $\vv{a} = -\sqrt{3}\ i + j$.
\begin{align*}
\theta &= \arccos{\left({a\cdot b \over ||a||\ ||b||}\right)}\\
&= \arccos{\left({-2\over 4}\right)}\\
&= \arccos{\left(-{1\over 2}\right)}\\
&= {2\pi \over 3}
\end{align*}

\item Find the angle between $\vv{a} = i + j$ and $\vv{b} = i + j + k$.
\begin{align*}
\theta &= \arccos{\left({a\cdot b \over ||a||\ ||b||}\right)}\\
&= \arccos{\left({2 \over \sqrt{6}}\right)}
\end{align*}

\item Find the angle between $\vv{a} = (1,-2,3)$ and $\vv{b} = (3,-6,-5)$.
\begin{align*}
\theta &= \arccos{\left({a\cdot b \over ||a||\ ||b||}\right)}\\
&= \arccos{(0)}\\
&= {\pi \over 2}
\end{align*}

\item Calculate $proj_{a}b$ where $\vv{a} = i + j$ and $\vv{b} = 2i + 3j - k$.
\begin{align*}
proj_{a}b &= {a\cdot b \over ||a||^{2}}\vv{a}\\
&= {4\over 2}(i + j)\\
&= 2i + 2j
\end{align*}

\item Calculate $proj_{a}b$ where $\vv{a} = i + j + 2k$ and $\vv{b} = 2i - 4j + k$.
\begin{align*}
proj_{a}b &= {a\cdot b \over ||a||^{2}}\vv{a}\\
&= {0\over 4}(i+j+2k)\\
&= 0
\end{align*}

\item Give a vector of length 3 that points in the same direction as the vector $i + j - k$.
\begin{quote}
The length is just a scalar value multiplied by the vector. Therefore we have the unit vector $i + j - k$,
with length 3 will be
\[3i + 3j - 3k.\]
\end{quote}

\item Is there ever a case where $proj_{a}b = proj_{b}a$? If so, under what conditions?
\begin{quote}
If we use the explicit formulas equal to each other, we have
\begin{align*}
\left({a\cdot b \over a\cdot a}\right)\vv{a} &= \left({a\cdot b \over b\cdot b}\right)\vv{b}\\
{||a|| \over ||b||}\vv{a} &= \vv{b}
\end{align*}
Therefore, the projections are equal only when $\vv{a} = \vv{b}$.
\end{quote}

\item Prove part 1 of Proposition 3.4.
\begin{quote}
We need to show that $||k\vv{a}|| = |k|\ ||\vv{a}||$. Then
\begin{align*}
||k\vv{a}|| &= ||(ka_{1},ka_{2},\dots,ka_{n})||\\
&= \sqrt{k^{2}a_{1}^{2} + k^{2}a_{2}^{2} + \cdots + k^{2}a_{n}^{2}}\\
&= \sqrt{k^{2}(a_{1}^{2} + a_{2}^{2} + \cdots + a_{n}^{2})}\\
&= \sqrt{k^{2}}\ \sqrt{a_{1}^{2} + a_{2}^{2} + \cdots + a_{n}^{2}}\\
&= |k|\ ||\vv{a}||.
\end{align*}
$\triangle$
\end{quote}

\item In physics, when a constant force acts on an object as the object is displaced, the work done by
the force is the product of the length of the displacement and the component of the force in the
direction of the displacement. Figure 1.48 depicts an object acted upon by a constant force $F$, which
displaces it from the point $P$ to the point $Q$. Let $\theta$ denote the angle between $F$ and the
direction of
displacement.
\begin{enumerate}
\item Show that the work done by $F$ is determined by the formula $F\cdot\vv{PQ}$.
\begin{quote}
We can determine the $x$ component of $F$ as $||F||\cos{\theta}$. Then the work can be computed as
$||F||\ ||\vv{PQ}||\cos{\theta}$, which is $F\cdot\vv{PQ}$ by definition.
\end{quote}
\item Find the work done by the (constant) force $F = i + 5j + 2k$ in moving a particle from the point
$(1, -1, 1)$ to the point $(2, 0, -1)$.
\begin{quote}
We need to find the dot product $(1,5,2)\cdot(1,1,-2)$, which is equal to $2$. Therefore, the work done
is 2 units.
\end{quote}
\end{enumerate}

\item Using vectors, prove that the diagonals of a parallelogram are perpendicular if and only if the
parallelogram is a rhombus. (Note: A rhombus is a parallelogram whose four sides all have the same
length).
\begin{quote}
We need to show that the vectors $\vv{ab}$ and $\vv{ba}$ have a dot product of 0. We know that the
lengths of $\vv{a}$ and $\vv{b}$ are equal, and that $\vv{ab} = (a_{1} - b_{1}, a_{2} - b_{2})$ and
$\vv{ba} = (b_{1} + a_{1}, b_{2} + a_{2})$. Therefore
\begin{align*}
\vv{ab}\cdot\vv{ba} &= (a_{1} - b_{1})(b_{1} + a_{1}) + (a_{2} - b_{2})(b_{2} + a_{2})\\
&= a_{1}^{2}+a_{2}^{2}-b_{1}^{2}-b_{2}^{2}\\
&= b_{1}^{2}+b_{2}^{2}-b_{1}^{2}-b_{2}^{2},\text{ Since the lengths are equal,}\\
&= 0
\end{align*}
\end{quote}
\end{enumerate}
\end{document}
