\documentclass{article}
\usepackage{amssymb,amsmath}
\usepackage{mathtools}
\usepackage{array}
\topmargin=-0.3in
\textheight=9.2in
\textwidth=168mm
\oddsidemargin=-0.2in
\evensidemargin=-0.2in
\title{Calculus Two Review}
\date{}

\newcommand{\beq}{\begin{quote}}
\newcommand{\eeq}{\end{quote}}
\newcommand{\dx}{\mathop{dx}}

\DeclareMathOperator{\arcsec}{arcsec}

\begin{document}
\section{Logarithmic Equations}
\begin{minipage}[t]{0.3\textwidth}
It can often be helpful to express a logarithm in terms of base $e$.
The following relationship can help us get to the natural log, $\ln$.
\[\log_{n}x = \frac{\ln{x}}{\ln{n}}\]
\end{minipage}
\hspace{0.5cm}
\begin{minipage}[t]{0.3\textwidth}
We can also use the properties of logarithmic functions to simplify exponential functions.
\[n^{x} = e^{\ln{x}}\]
\end{minipage}
\hspace{0.5cm}
\begin{minipage}[t]{0.3\textwidth}
Applying calculus to logarithmic and exponential functions can yield interesting results.
\[\frac{d}{dx}(\log_{n}{(x)}) = \frac{1}{x\ln{(n)}}\]
\[\int n^{x}dx = \frac{n^{x}}{\ln{(n)}}\]
\end{minipage}


\section{Inverse Trigonometry Functions}
The derivatives of inverse trig functions are useful for integration.
We have the following derivatives:
\[
\frac{d}{dx}(\arcsin{(x)}) = \frac{1}{\sqrt{1-x^{2}}}\qquad\qquad
\frac{d}{dx}(\arctan{(x)}) = \frac{1}{1+x^{2}}\qquad\qquad
\frac{d}{dx}(\arcsec{(x)}) = \frac{1}{|x|\sqrt{1+x^{2}}}
\]


\section{Integration Techniques}
The main focus of Calculus II is integration.
In Calculus I, we learned how to integrate by using a $u$ substitution, but that technique is often not powerful enough to handle more complex integrals.
Luckily, we have several different methods that allow us to integrate.
\subsection{Integration By Parts}
\[\int u dv = uv = \int v du\]
This technique is useful for integrating two functions, $f(x)$ and $g(x)$.
Suppose we want to find the integral $\int xe^{x^{2}}dx$, or $\int x\sin{\pi x}dx$, or even $\int (3t + 5)\ln{(t/5)}dt$.
All of these examples are set up to be solved by integration by parts.
We want to split the integrand into two separate functions.
We then differentiate one ($u$), and integrate the other ($dv$).
Applying our formula will often simplify the original integral into something that we know how to easily integrate.
Typically, when we choose our $u$ and $dv$, we want to choose the function whose integral we know for $dv$, and the other one for $u$ (remember, it is often easier to take a derivative than integrate).

\subsection{Trig Combination}
Ultimately, this technique will end up using integration by parts in most cases.
There are a few trig identities that we need to know first:
\[
\sin^{2}{(x)} + \cos^{2}{(x)} = 1\qquad\qquad
\sin^{2}{(x)} = \frac{1}{2}(1-\cos{(2x)})\qquad\qquad
\cos^{2}{(x)} = \frac{1}{2}(1+\cos{(2x)})
\]
There are some common patterns that often appear in integrals that we can solve using theis method.
\begin{enumerate}
\item If we see a $\cos{x}$ to an odd power, then we should let $u = \sin{x}$.
\item If we see a $\sin{x}$ to an odd power, then we should let $u = \cos{x}$.
\item If we see a $\sec{x}$ to an odd power, then we should let $u = \tan{x}$.
\item If we have a $\tan{x}$ to an odd power, and we have $\sec{x}$ present, then we should let $u=\sec{x}$.
\item If we have $\sec{x}$ to an odd power and $\tan{x}$ to an even power, then convert the $\tan{x}$ to $\sec{x}$.
\item If we \textbf{only} have $\sec{x}$ to an odd power, then we need to use integration by parts.
\item If we have a $\tan{x}$ and do \textbf{not} have a $\sec{x}$ present, then we need to convert the $\tan{x}$ into $\sec{x}$.
\item If we see a trig function to an even power, then we need to use the half angle formulas first.
\end{enumerate}

\subsection{Trig Substitution}
This technique for integration is applicable mostly to integrals that contain functions that look like the derivatives of inverse trig functions.
We have a few special cases that tell us what substitution to make.
\begin{alignat*}{3}
\sqrt{a^{2} - u^{2}} &\to 1 - \sin^{2}{\theta} = \cos^{2}{\theta} &&\to u = a\sin{\theta}\\
\sqrt{a^{2} + u^{2}} &\to 1 + \tan^{2}{\theta} = \sec^{2}{\theta} &&\to u = a\tan{\theta}\\
\sqrt{u^{2} - a^{2}} &\to \sec^{2}{\theta} - 1 = \tan^{2}{\theta} &&\to u = a\sec{\theta}\\
\end{alignat*}

\subsection{Partial Fractions}
We want to use this technique to solve fraction of functions:
\[
\int \frac{\mathcal{P}(x)}{\mathcal{Q}(x)}dx\ \text{Where the degree of }\mathcal{P}\ \text{is less than the degree of }\mathcal{Q}
\]
To use this method, we need to
\begin{enumerate}
\item Factor the denomiator
\item Determine the decomposition
\item Determine the unknown values of the numerator
\end{enumerate}


\section{Series and Sequences}
\subsection{Telescoping Series}
A telescoping series takes the form of
\[
\sum_{n=1}^{\infty} \frac{1}{n} - \frac{1}{n+1}
\]
The series will expand in such a way that all the terms will cancel except for the first and last terms.
This allows us to say that the series will converge at
\[
\lim_{n\to\infty} 1 - \frac{1}{n+1}
\]

\subsection{Geometric Series}
A geometric series has the form
\[
\sum_{n=1}^{\infty}ar^{n-1}
\]
where $a\neq 0$ and $r$ is constant.
We know that
\[
|r| \geq 1 \implies\ \text{Divergence}\qquad\qquad |r| < 1\implies\ \text{Convergence}
\]
If the sequence converges, it converges to
\[
\frac{s_{0}}{1 - r}
\]
where $s_{0}$ is the first term in the series.

\subsection{P Series}
\[
\sum_{n=1}^{\infty}\frac{1}{n^{p}}
\]
If $p \leq 1$, then the series diverges. If $p > 1$, then the series converges.

\subsection{Tests of Convergence}
\subsubsection{The Divergence Test}
If
\[
\lim_{n\to\infty}a_{n}\neq 0
\]
then the series $\sum a_{n}$ diverges.
Note that this test \textbf{only} determines when the series diverges.
It says nothing about the series converging.

\subsubsection{Comparison Test}
Let $\sum a_{n}$ and $\sum b_{n}$ be positive termed series.
\begin{itemize}
\item If $b_{n}$ converges, $a_{n} < b_{n} \implies a_{n}$ converges.
\item If $b_{n}$ diverges, $a_{n} > b_{n} \implies a_{n}$ diverges.
\end{itemize}

\subsubsection{Limit Comparison Test}
Let $\sum a_{n}$ and $\sum b_{n}$ be positive termed series.
If
\[
\lim_{n\to\infty}\frac{a_{n}}{b_{n}} = \mathcal{L}
\]
then
\begin{enumerate}
\item If $\mathcal{L} = 0$ and $\sum b_{n}$ converges, then $\sum a_{n}$ converges.
\item If $\mathcal{L} \to \infty$, and $\sum b_{n}$ diverges, then $\sum a_{n}$ diverges.
\item If $\mathcal{L} > 0$, then both series either both converge or both diverge.
\end{enumerate}

\subsubsection{Ratio and Root Test}
\[
\lim_{n\to\infty}\frac{a_{n+1}}{a_{n}} = \mathcal{L}\qquad\qquad\qquad
\lim_{n\to\infty}\sqrt[\leftroot{-2}\uproot{2}n]{a_{n}} = \mathcal{L}
\]
\begin{enumerate}
\item $\mathcal{L} < 1 \implies a_{n}$ converges
\item $\mathcal{L} > 1 \implies a_{n}$ diverges
\item $\mathcal{L} = 1$ is inconclusive
\end{enumerate}

\subsubsection{Alternating Series Test}
The alternating series
\[
\sum (-1)^{n+1}a_{n}
\]
converges if
\begin{enumerate}
\item $a_{n+1} \leq a_{n}$
\item $\lim_{n\to\infty}a_{n} = 0$
\end{enumerate}

\subsubsection{Integral Test}
Given the series
\[\sum_{n=j}^{\infty} a_{n}\]
let $f(n) = a_{n}$.
The series will converge if and only if
\[
\int_{j}^{\infty}f(n)dn
\]
exists.


\subsection{Types of Convergence}
\subsubsection{Absolute Convergence}
A series converges absolutely if both $\sum a_{n}$ and $\sum |a_{n}|$ converge.

\subsubsection{Conditional Convergence}
A series converges conditionally if $\sum a_{n}$ converges and $\sum |a_{n}|$ diverges.

\subsection{Special Series}
\subsubsection{Power Series}
A power series is a special type of p-series that takes the form
\[
\sum_{n=0}^{\infty} a_{n}(x-x_{0})^{n}
\]
All power series converge for $x = x_{0}$.
With power series, we often want to know the radius of convergence of the series.
The radius of convergence, $\mathcal{R}$, can be determined by finding the distance that $x$ can move from $x_{0}$ while the series still converges.
The interval $\mathcal{I} = [x_{0} - \mathcal{R}, x_{0} + \mathcal{R}]$ is called the interval of convergence.

\subsubsection{Taylor Series}
The Taylor series representation of a function can be given by
\[
\sum_{n=0}^{\infty}\frac{f^{(n)}(x_{0})}{n!}(x-x_{0})^{n}
\]
where $f^{(n)}$ is the $n^{\text{th}}$ derivative of $f$.
Taylor series provide an appromation of a function.

\subsubsection{Binomial Series}
A binomial series is given by
\[
\sum_{k=0}^{n}\frac{n!}{k!(n-k)!}x^{k}
\]
\end{document}
