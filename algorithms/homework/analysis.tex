\documentclass{hw}

\usepackage{minted}
\usepackage{mathtools}
% \DeclarePairedDelimiter\floor{\lfloor}{\rfloor}
% \DeclarePairedDelimiter\ceil{\lceil}{\rceil}

\newcommand{\ceil}[1]{\left\lceil #1\right\rceil}

\begin{document}
\makeheader{1}
\noindent\\

Derive the order of magnitude worst-case running time of the following algorithm, and give a clear
explanation of how you arrived at this bound, in the same manner as the CLRS insertion-sort example
from chapter 2 which we did this in class. As I did in class, you can leave out the constants
$c_{i}$ for each line of code; but, where appropriate, you must analyze loops by setting up a sum
over the loop iterations.
Here swap($A, i, j$) means interchange array elements $A[i]$, $A[j]$. Note that swap is not one of
our basic constant-time operations; how will you deal with this?\\

\begin{minted}[linenos]{ruby}
i = 1
j = N
while i < j
    for k = i to j - 1
        if A[k] > A[k + 1]
            swap(A, k, k + 1)
    for k = j - 1 to i + 1 by -1
        if A[k - 1]>A[k]
            swap(A, k - 1, k)
    i = i + 1
    j = j - 1
\end{minted}

We know that the while loop will execute until $i=j$. This means that the sum will execute at most
$\ceil{N \over 2}$. We can analyze both the inner for loops individually in order to formulate a
sum that is representative of the number of instructions for that loop. The first loop executes
$N-1$ times on the first iteration, then $N-3$ times on the second iteration, and so on. This
yields the sum
\[
\sum_{i=0}^{N} N-(2i+1).
\]
Similarly, the second for loop will execute almost the exact same number of times; however, only
the even $i$ terms will be taken into account. As such, we have that the second loop will execute
\[
\sum_{i=0}^{N} N-2i
\]
times. Assuming the worst case scenario, we can say that the comparison in the if statement of both
loops will return true every iteration. As a result, the swap function will be executed on every
iteration. We don't actually know the execution time of the swap function, so we can assign it the
value $\mathcal{S}_{t}$. Combining both of the sums we already have and multiplying them by the
swap time yields
\[
\mathcal{S}_{t}\sum_{i=0}^{N}2N-4i-1.
\]
The only other operations in the while loop are constant time, so we will ignore those.\\

We now need to express the number of times the while loop executes as a sum. We know that every
time the loop executes, we increment $i$ and decrement $j$. As a result, we know that the loop will
only execute $\ceil{N \over 2}$ times. We also know that our $i$ starts at 1. Combining the sums
together gives us
\[
\mathcal{S}_{t}\sum_{i=1}^{\ceil{N \over 2}}\sum_{i=0}^{N}2N-4i-1.
\]
Of course, we don't want to have a sum of sums. Through algebraic manipulation, we have that
\[
\sum_{i=0}^{N}2N-4i-1 = -(1-n),
\]
so we are now dealing with
\[
\mathcal{S}_{t}\sum_{i=1}^{\ceil{N \over 2}}-(1-N).
\]
We have two cases to deal with now: when $N$ is even and when $N$ is odd. If $N$ is even, then we
are dealing with
\[
\mathcal{S}_{t}\sum_{i=1}^{N \over 2}-(1-N).
\]
This can be simplified to
\[
{1\over2}(N-1)N = {N^{2} - N \over 2}
\]
which is $O(N^{2})$. If $N$ is odd, then we have
\[
\mathcal{S}_{t}\sum_{i=1}^{N + 1 \over 2}-(1-N),
\]
which is equivalent to
\[
{N^{2} - 1 \over 2},
\]
and that is still $O(N^{2})$.
\end{document}
