\documentclass{hw}

\usepackage{minted}

\begin{document}
\section*{String Splitting Algorithm}
\noindent Talia Hicks and Steven Rosendahl

\vspace{1cm}

We chose to implement the string breaking algorithm. To do this, we built a recursive algorithm
that looks down the left and right side of the string, using the $i^{\text{th}}$ breakpoint as a
starting index. The program loops through all possible starting indexes, and takes the least of them.

\begin{minted}{java}
for(int i = 0; i < cuts.length; i++){
    cost = Math.min(nval = splitStringRec(cuts, n, i), cost);
}
\end{minted}

This loop runs $n$ times, where $n$ is the size of the list of breakpoints. The function
\textit{splitStringRec} is a recursive function that runs down the left and right sides of the
string, trying out different breaks.

\begin{center}
\begin{minted}{java}
//Looping down the left
for(int j = 0; j < cpy.length; j++){
    cutlen = Math.min(splitStringRec(cpy, first, j), cutlen);
}

//Looping down the right
for(int i = 0; i < cpy.length; i++){
    cutr = Math.min(splitStringRec(cpy, newcost, i), cutr);
}
\end{minted}
\end{center}

The loop that runs through the left breakpoints runs once the first time through, then twice on the
second time through, and so on until it has run $n-1$ times. Similarly, the loop that runs through the
right breakpoints runs $n-1$ times, then $n-2$, $n-3$, and so on until it only runs once. Since each
loop takes the minimum of the cost, we have that only the minimum cuts will be preserved, while the
rest are thrown away. The runtime of such an algorithm is close to
\begin{align*}
\sum_{n=0}^{i}\left(
\sum_{k=0}^{n}(n-k) +
\sum_{k=0}^{n}(k+n)
\right)
&=\sum_{n=0}^{i}\left(O(n^{2}) + O(n^{2})\right)\\
&=\sum_{n=0}^{i}O(n^{2})\\
&=O(n^{3})
\end{align*}

To state the recurrence relation, let $n$ be the length of the string, $C$ be the array of cuts,
and $i=|C|$. We want to build a definition such that we loop down the left half of the string, making
cuts, which results in a new string of size $n-c$. Similarly, we want to move down the right side of
the string, making cuts which result in strings of size $c-n$. We also want to ``remove" the cuts that
we have already made, so we create a new array that contains all the remaining cuts without the cut we
just made (i.e. $C\backslash c_{0}$ where $c_{0}$ is the cut we just made).
\[
\text{min\_cost}(n, C)=
\begin{cases}
n,\hspace{6.2cm} i = 1\\
n + \text{min}
\begin{cases}
\text{min\_cost}(n - c_{0},\ C\backslash c_{0})\\
\text{min\_cost}(c_{0} - n,\ C\backslash c_{0})
\end{cases},
\hspace{0.8cm} i > 1
\end{cases}
\]




\end{document}
