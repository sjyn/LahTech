\documentclass{hw}

\begin{document}

    \noindent We can define a function $f:\RR\to\RR$ by
    \[
        f(x)=
        \begin{cases}
            0, \qquad x\in\QQ\\
            1, \qquad x\in\RR\setminus\QQ
        \end{cases}.
    \]
    We know this function to be dicontinuous under the usual absolute value. In $\RR$ under the usual absolute value,
    we define a function $h$ to be continuous at a point $a$ in the domain of $h$ by
    \[
        \forall\epsilon > 0,\ \exists\delta > 0 \text{ such that } 0<|x-a|<\delta\implies|h(x)-h(a)|<\epsilon.
    \]
    We can reformulate this definition in the p-adic sense by
    \[
        \forall\epsilon > 0,\ \exists\delta > 0 \text{ such that } 0<|x-a|_{p}<\delta\implies|h(x)-h(a)|_{p}<\epsilon.
    \]
    If we refomulate our original function, then we can show that under the p-adic absolute value, the function is
    continuous at every point. We will define $g:\QQ_{p}\to\QQ_{p}$ as
    \[
        g(x)=
        \begin{cases}
            0, \qquad x\in\ZZ\\
            1, \qquad x\in\QQ_{p}\setminus\ZZ
        \end{cases}
    \]
    as an analog to the funtion $f$ we defined earlier. We will show that this function is indeed continuous.

    \begin{quote}
        \textit{\textbf{Proof:}} Let $\epsilon>0$ be given. If $g$ is continuous at a point $c$, then $\exists\delta>0$
        such that $\forall\epsilon >0$, $|x-c|_{p}<\delta\implies|g(x)-g(c)|_{p}<\epsilon$. If we let $\delta=1$, then we have 

    \end{quote}
\end{document}
