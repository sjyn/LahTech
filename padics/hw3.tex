\documentclass{hw}

\begin{document}
    \makeheader{3}
    \begin{enumerate}
        \item Suppose $F$ is a field with a non-Archemedian absolute value $|\cdot|$ and $\IRing{F}$ is the ring of
        integers of $F$. Prove that a point $x\in\IRing{F}$ is a unit if and only if $|x|=1.$
        \begin{quote}
            Suppose $x\in\IRing{F}$ is a unit and consider $x^{-1}\in\IRing{F}$ as the inverse of $x$. We know
            $x^{-1}\in\IRing{F}$ since $x$ is a unit in $\IRing{F}$. Since $x\in\IRing{F}$, we know
            \begin{align*}
                |x| &\leq 1\\
                |x^{-1}||x| &\leq |x^{-1}|\leq1 \qquad\text{(since $x^{-1}\in\IRing{F}$)}\\
                |x^{-1}x| &\leq |x^{-1}|\leq1\\
                |1| &\leq |x^{-1}| \leq1\\
                1 &\leq |x^{-1}| \leq 1.
            \end{align*}
            The only choice we have for $x^{-1}$ is 1, so
            \begin{align*}
                x^{-1} &= 1\\
                xx^{-1}&= x\\
                x&=1\\
                |x|&=1.
            \end{align*}

            Suppose $|x| = 1$. We want to show that $x^{-1}\in\IRing{F}$. Consider $|x^{-1}|$ where $x^{-1}\in F$.
            Then
            \begin{align*}
                |x|&=1\\
                |x||x^{-1}| &= |x^{-1}|\\
                |xx^{-1}| &= |x^{-1}|\\
                |1| &= |x^{-1}|\\
                1 &= |x^{-1}|.
            \end{align*}
            Therefore $|x^{-1}|=1$, so $x^{-1}\in\IRing{F}$ and therefore $x$ is a unit. 
        \end{quote}

        \item If $n$ is a non-negative integer a $p$ is prime, define
        $\mathcal{I}_{n}=\{x\in\IRing{p}:|x|_{p}\leq p^{-n}\}$ where\\ $\IRing{p}=\{x\in\QQ : |x|_{p}\leq1\}$.
        \begin{enumerate}
            \item If $x,y\in\mathcal{I}_{n}$ prove that $x+y\in\mathcal{I}_{n}$.
            \begin{quote}
                Suppose $x,y\in\mathcal{I}_{n}$, and consider $|x+y|_{p}$. Then
                \begin{align*}
                    |x+y|_{p} &\leq \mmax{|x|_{p}}{|y|_{p}}\\
                    &\leq \mmax{p^{-n}}{p^{-n}}\\
                    &= p^{-n}
                \end{align*}
                Therefore $|x+y|_{p}\leq p^{-n}$, so $x+y\in\mathcal{I}_{n}$.
            \end{quote}

            \item If $x\in\mathcal{I}_{n}$ and $r\in\IRing{p}$ prove that $rx\in\mathcal{I}_{n}$.
            \begin{quote}
                Let $|\cdot|$ be the p-adic absolute value and consider $|rx| = |r|\cdot|x|$. Since $|x|\leq p^{-n}$ and $|r|\leq1$, we have $|r|\cdot|x|\leq 1\cdot p^{-n} = p^{-n}$, so $|rx|\leq p^{-n}$ which implies that $rx\in\mathcal{I}_{n}$.
            \end{quote}

            \item What is another way to describe the set $\ZZ\cap\mathcal{I}_{n}$?
            \begin{quote}
                We know by definition of the p-adic absolute value that for all $x\in\ZZ$, $|x|_{p}\leq1$. This implies
                $\ZZ\subset\IRing{p}$. We want to find all $x\in\ZZ$ where $|x|_{p}\leq p^{-n}$ for some $n\in\NN$.
                We cannot consider any $x$ where $x$ is a multiple of $p$, since those values would be in $\QQ$. This
                leaves only 1 (when $x$ does not share any factors of $p$) or 0 (when $x=0$). For any value $p$,
                $0 < p^{-n} < 1$, so the only value we have is $x=0$. Therefore, $\ZZ\cap\mathcal{I}_{n}=\{0\}$.
            \end{quote}
        \end{enumerate}
    \end{enumerate}
\end{document}
