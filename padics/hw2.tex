\documentclass{hw}

\geometry{margin=1in}

\begin{document}
\makeheader{2}

\begin{enumerate}
	\item If $F$ is a finite field, prove that the only absolute value on $F$ is the trivial absolute
	      value.
	      \begin{quote}
	      	\textit{\textbf{Proof:}} By definition, if we take any $x\in F$ such that $x\neq0$, then there is some
	      	integer $m\in\ZZ$ such that $x^{m}=1$, meaning that $|x|=1$. If $x=0$, then $|x|=0$, so the absolute
	      	value is trivial.
	      \end{quote}

	\item Suppose $p$ is prime. Prove that $|p^{n}|_{p}$ tends to 0 as $n\to\infty$.
	      \begin{quote}
	      	\textit{\textbf{Proof:}} We know that $v_{p}(x)>0$ for all $x\in\ZZ$, and that $v_{p}(x)$ counts the
	      	number of factors of $p$ in $x$. If we consider $p^n$, there are $n$ factors of $p$, meaning that
	      	$v_{p}(p^{n})=n$. Then $|p^{n}|_{p}=p^{-v_{p}(p^{n})}=p^{-n}$, so
	      	\[
	      		\lim_{n\to\infty}p^{-n}=0.
	      	\]
	      \end{quote}

	\item Suppose that $F$ is a field and $|\ |$ is a non-Archemedian absolute value on $F$. If $c\in F$
	      and $r$ is a positive real number, we define the \textit{open ball (or disk) centered at c of radius r}
	      by
	      \[
	      	B(c,r)=\{x\in F:|x-c|<r\}.
	      \]
	      If $d\in B(c,r)$, prove that $B(c,r)=B(d,r)$.
	      \begin{quote}
	      	\textit{\textbf{Proof:}} Suppose $x\in B(d,r)$. Then
			\begin{align*}
				|x-d| &= |x-d-c+c|\\
				&= |x-c + c-d|\\
				&\leq \text{max}\{|x-c|,|c-d|\}\\
				&= \text{max}\{|x-c|,|d-c|\}\\
				&\leq\text{max}\{|x-c|,r\}.
			\end{align*}
			Suppose $|x-d|\leq |x-c|$. Then the value $|x-d|$ is always within the disk $B(c,r)$.\\\\
			Suppose $x\in B(c,r)$. Then
			\begin{align*}
				|x-c| &= |x-c-d+d|\\
				&= |x-d + d-c|\\
				&\leq \text{max}\{|x-d|,|d-c|\}\\
				&<\text{max}\{|x-d|,r\}.
			\end{align*}
			Suppose max$\{|x-d|,r\} = |x-d|$. Then $|x-c|$ is always within the disk $B(d,r)$.
	      \end{quote}
\end{enumerate}
\end{document}
