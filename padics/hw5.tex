\documentclass{hw}

\begin{document}
\makeheader{5}
\begin{enumerate}
	\item Let $f(x,y)=x^2 + y^2$. Observe that $(x,y)=(0,0)$ defines a solution to the equation $f(x,y)=0$. Do the remainder of this problem, we shall call $(0,0)$ the trivial solution to $f(x,y)=0$.
	      \begin{enumerate}
	      	\item Prove that $f(x,y)=0$ has no non-trivial solutions with $x,y\in\QQ$.
	      	\item Prove that there exists a prime $p$ such that $f(x,y)=0$ has a nontrivial solution with $x,y\in\QQ_{p}$.
	      	\item Verify that $f$ satisfies the Hasse Principle without referring to the Hasse-Minkowski Theorem.
	      \end{enumerate}

	\item Suppose that $\{ a_{n} \}_{n=0}^{\infty}$ is a sequence of points in $\ZZ_{p}$. Prove that the series
	      \[
	      	\sum_{n=0}^{\infty}a_{n}p^{n}
	      \]
	      converges in $\QQ_{p}$.
	      \begin{quote}
	      		\textit{\textbf{Proof:}} We know that a series will converge if and only if its sequence of partial sums converges. We can test for convergence by calculating
				\[
					\lim_{n\to\infty}\left|a_{n+1}p^{n+1}-a_{n}p^{n}\right|_{p}.
				\]
				We can say
				\begin{align*}
					\left|a_{n+1}p^{n+1}-a_{n}p^{n}\right|_{p} &= \max\left\{%
						\left|a_{n+1}p^{n+1}\right|_{p},\left|a_{n}p^{n}\right|_{p}%
					\right\}\\
					&= \max\left\{%
						\frac{1}{p^{v_{p}(a_{n+1}) + n + 1}}, \frac{1}{p^{n+v_{p}(a_{n})}}%
					\right\}.\\
					\lim_{n\to\infty}\max\left\{%
						\frac{1}{p^{v_{p}(a_{n+1}) + n + 1}}, \frac{1}{p^{n+v_{p}(a_{n})}}%
					\right\} &= \max\left\{0,0\right\}\\
					&= 0.
				\end{align*}
				Therefore, the sequence of partial sums is Cauchy, and thus convergent, so the series converges.
	      \end{quote}

	\item Suppose that $k\in\NN$.
	      \begin{enumerate}
	      	\item Prove that the series
	      	      \[
	      	      	\sum_{n=1}^{\infty}\frac{1}{n^{k}}
	      	      \]
	      	      does not converge in $\QQ_{p}$ for any prime $p\neq\infty$.
				  \begin{quote}
					  \textit{\textbf{Proof:}} Suppose otherwise. Then
					  \[
					  	\lim_{n\to\infty}\left|%
							\frac{1}{(n+1)^{k}}-\frac{1}{n^{k}}%
						\right|_{p} = 0.
					  \]
					  We can analyze the $p$-adic absolute value inside the limit:
					  \begin{align*}
						  \left|\frac{1}{(n+1)^{k}}-\frac{1}{n^{k}}\right|_{p} &\leq
						  \max\left\{%
						  	p^{-v_{p}\left(1/(n+1)^{k}\right)}, p^{-v_{p}\left(1/n^{k}\right)}%
						  \right\}
					  \end{align*}
					  We can solve for the valuation in the first term:
					  \[
					  	v_{p}\left(
							\frac{1}{(n+1)^{k}}
						\right) =
						-\frac{
							k\log{(n+1)}+\log{m}
						}{
							\log{p}
						},
					  \]
					  where $m\nmid p$. If we take the limit as $n\to\infty$ of this expression, we see that it tends towards negative infinity, so the $p$-adic absolute value tends towards infinity. Similarly, we can evaluate the limit of $|1/n^{k}|_{p}$, and we will see that it too tends towards infinity. Therefore,
					  \[
					  	\lim_{n\to\infty}\left|%
							\frac{1}{(n+1)^{k}}-\frac{1}{n^{k}}%
						\right|_{p} \neq 0,
					  \]
					  so the series does not converge since its sequence of partial sums does not converge.
				  \end{quote}

	      	\item Prove that the series
	      	      \[
	      	      	\sum_{n=1}^{\infty}n^{k}
	      	      \]
	      	      does not converge in $\QQ_{p}$ for any prime $p\neq\infty$.
				  \begin{quote}
					  \textit{\textbf{Proof:}} We want to show that
					  \[
					  	\lim_{n\to\infty}\left|(n+1)^{k} - n^{k}\right|_{p}\neq 0.
					  \]
					  We can use a similar trick as above:
					  \begin{align*}
						  \left|(n+1)^{k} - n^{k}\right|_{p} &\leq \max\left\{
						  	
						  \right\}
					  \end{align*}
				  \end{quote}
	      \end{enumerate}
\end{enumerate}
\end{document}
