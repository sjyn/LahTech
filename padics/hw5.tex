\documentclass{hw}

\begin{document}
\makeheader{5}
\begin{enumerate}
	\item Let $f(x,y)=x^2 + y^2$. Observe that $(x,y)=(0,0)$ defines a solution to the equation $f(x,y)=0$. For the remainder of this problem, we shall call $(0,0)$ the trivial solution to $f(x,y)=0$.
	      \begin{enumerate}
	      	\item Prove that $f(x,y)=0$ has no non-trivial solutions with $x,y\in\QQ$.
	      	      \begin{quote}
	      	      	Suppose that $f(x,y)$ has a non-trivial zero in $\QQ$. Then $x^{2}+y^{2} = 0$ for some pair $(x,y)\neq(0,0)$. For any choice of $x\neq0$ or $y\neq0$, we know that $x^{2}$ and $y^{2}$ are both positive. The sum of two positive numbers is still positive, so $x^{2}+y^{2} > 0$ for $x\neq0$ and $y\neq0$. Therefore there cannot be any other zeros besides $(0,0)$.
	      	      \end{quote}
	      	\item Prove that there exists a prime $p$ such that $f(x,y)=0$ has a nontrivial solution with $x,y\in\QQ_{p}$.
	      	      \begin{quote}
	      	      	Let $p=2$ and $y=1$. We want to show by Hensel's Lemma that $f(x,1)=g(x)=x^{2}+1$ has a solution near $x=1$. We can see that the first property of Hensel's Lemma is satisfied since
	      	      	\[
	      	      		g(1)=1^{2}+1=2\equiv0\mod2.
	      	      	\]
	      	      	Similarly, the second property is satisfied, since
	      	      	\[
	      	      		g'(1)=1 \not\equiv 0\mod2.
	      	      	\]
	      	      	Hence, there is a non-trivial zero of $f$ at $(x,y)=(1,1)$.
	      	      \end{quote}
	      	\item Verify that $f$ satisfies the Hasse Principle without referring to the Hasse-Minkowski Theorem.
	      	      \begin{quote}
	      	      	We have shown that $f$ has has no non-trivial solutions in $\QQ$ and that $f$ has a non-trivial solution in $\QQ_{2}$, so it satisfies the negation of the Hasse Principle.
	      	      \end{quote}
	      \end{enumerate}

	\item Suppose that $\{ a_{n} \}_{n=0}^{\infty}$ is a sequence of points in $\ZZ_{p}$. Prove that the series
	      \[
	      	\sum_{n=0}^{\infty}a_{n}p^{n}
	      \]
	      converges in $\QQ_{p}$.
	      \begin{quote}
	      	\textit{\textbf{Proof:}} We know that a series will converge if and only if its sequence of partial sums converges. We can test for convergence by calculating
	      	\[
	      		\lim_{n\to\infty}\left|a_{n+1}p^{n+1}-a_{n}p^{n}\right|_{p}.
	      	\]
	      	We can say
	      	\begin{align*}
	      		\left|a_{n+1}p^{n+1}-a_{n}p^{n}\right|_{p} & \leq \max\left\{%
	      		\left|a_{n+1}p^{n+1}\right|_{p},\left|a_{n}p^{n}\right|_{p}%
	      		\right\}\\
	      		                                           & = \max\left\{%
	      		\frac{1}{p^{v_{p}(a_{n+1}) + n + 1}}, \frac{1}{p^{n+v_{p}(a_{n})}}%
	      		\right\}.\\
	      		\lim_{n\to\infty}\max\left\{%
	      		\frac{1}{p^{v_{p}(a_{n+1}) + n + 1}}, \frac{1}{p^{n+v_{p}(a_{n})}}%
	      		\right\}                                   & = \max\left\{0,0\right\} \\
	      		                                           & = 0.
	      	\end{align*}
	      	Since the $p$-adic absolute value is greater than or equal to 0, we know
	      	\[
	      		\lim_{n\to\infty}\left|a_{n+1}p^{n+1}-a_{n}p^{n}\right|_{p} = 0.
	      	\]
	      	Therefore, the sequence of partial sums is Cauchy, and thus convergent, so the series converges.
	      \end{quote}

	\item Suppose that $k\in\NN$.
	      \begin{enumerate}
	      	\item Prove that the series
	      	      \[
	      	      	\sum_{n=1}^{\infty}\frac{1}{n^{k}}
	      	      \]
	      	      does not converge in $\QQ_{p}$ for any prime $p\neq\infty$.
	      	      \begin{quote}
	      	      	\textit{\textbf{Proof:}} If we can show that the sequence of partial sums does not converge, then the series will not converge. We want to evaluate (and eventually take the limit of)
	      	      	\[
	      	      		\left|
	      	      		\frac{1}{(n+1)^{k}}-\frac{1}{n^{k}}
	      	      		\right|_{p}.
	      	      	\]
	      	      	We know by definition that
	      	      	\begin{align*}
	      	      		\left|
	      	      		\frac{1}{(n+1)^{k}}-\frac{1}{n^{k}}
	      	      		\right|_{p} & =
	      	      		\left|
	      	      		\frac{n^{k}-(n+1)^{k}}{n^{k}(n+1)^{k}}
	      	      		\right|_{p}\\
	      	      		            & = p^{-
	      	      		\left[
	      	      		v_{p}(n^{k}-(n+1)^{k}) - v_{p}(n^{k})-v_{p}((n+1)^{k})
	      	      		\right]
	      	      		}.
	      	      	\end{align*}
	      	      	Through some tedious calculation, we can solve for the valuation of each term in the exponent.
	      	      	\begin{gather*}
	      	      		v_{p}\left(n^k-(n+1)^k\right) =
	      	      		\frac{
	      	      			\log{(n^k-(n+1)^k) - \log(m_{0})}
	      	      			}{
	      	      			\log{(p)}
	      	      			}\\
	      	      		v_{p}(n^{k}) =
	      	      		\frac{
	      	      			k\log{(n)}-\log{(m_{1})}
	      	      			}{\log{(p)}}\\
	      	      		v_{p}((n+1)^k) =
	      	      		\frac{
	      	      			k\log{(n+1)}-\log{(m_{2})}
	      	      			}{
	      	      			\log{(p)}
	      	      		},
	      	      	\end{gather*}
	      	      	where $m_{0},m_{1}$, and $m_{2}$ do not divide $p$. Combining these valuations as required yields
	      	      	\[
	      	      		-\left[v_{p}(n^{k}-(n+1)^{k}) - v_{p}(n^{k})-v_{p}((n+1)^{k})\right]=
	      	      		\frac{
	      	      			\log{\left(
	      	      				\frac{n^{k}(n+1)^{k}}{n^k-(n+1)^k}
	      	      				\right)}
	      	      			}{
	      	      			\log{p}
	      	      		} + c
	      	      	\]
	      	      	where $c$ is some constant number independant of $n$. We can now evaluate the real valued limit as $n\to\infty$.
	      	      	\[
	      	      		\lim_{n\to\infty}\frac{
	      	      			\log{\left(
	      	      				\frac{n^{k}(n+1)^{k}}{n^k-(n+1)^k}
	      	      				\right)}}{\log{p}} + c = c+\frac{1}{\log{p}}\lim_{n\to\infty}\log{\left(
	      	      		\frac{n^{k}(n+1)^{k}}{n^k-(n+1)^k}
	      	      		\right)}.
	      	      	\]
	      	      	The function on the inside of the limit is continuous for all $k\in\NN$, so we can determine the limit as follows:
	      	      	\begin{align*}
	      	      		\lim_{n\to\infty}\log{\left(
	      	      		\frac{n^{k}(n+1)^{k}}{n^k-(n+1)^k}
	      	      		\right)} & = L      \\
	      	      		e^{
	      	      		\lim_{n\to\infty}\log{\left(
	      	      		\frac{n^{k}(n+1)^{k}}{n^k-(n+1)^k}
	      	      		\right)}
	      	      		}        & = e^{L}  \\
	      	      		\lim_{n\to\infty}e^{\log{\left(
	      	      		\frac{n^{k}(n+1)^{k}}{n^k-(n+1)^k}
	      	      		\right)}
	      	      		}        & = e^{L}  \\
	      	      		\lim_{n\to\infty}\left(
	      	      		\frac{n^{k}(n+1)^{k}}{n^k-(n+1)^k}
	      	      		\right)  & = e^{L}.
	      	      	\end{align*}
	      	      	For the fraction inside the limit, notice that after multiplication, the highest order term in the numerator will be $n^{2k}$, while the highest order term in the denominator will be $n^{k}$, so the limit behaves like
	      	      	\[
	      	      		\lim_{n\to\infty}\frac{n^{2k}}{n^k}=\lim_{n\to\infty}n^{k}=\infty = e^{L}.
	      	      	\]
	      	      	Taking the logarithm of both sides changes nothing, nor does dividing by $\log{p}$ where $p\neq\infty$, or adding $c$. Hence, we have that
	      	      	\[
	      	      		\lim_{n\to\infty}
	      	      		\left|
	      	      		\frac{1}{(n+1)^{k}}-\frac{1}{n^{k}}
	      	      		\right|_{p} = \infty,
	      	      	\]
	      	      	so the sequence of partial sums does \textbf{not} converge, and therefore the series does not converge.
	      	      \end{quote}

	      	\item Prove that the series
	      	      \[
	      	      	\sum_{n=1}^{\infty}n^{k}
	      	      \]
	      	      does not converge in $\QQ_{p}$ for any prime $p\neq\infty$.
	      	      \begin{quote}
	      	      	\textit{\textbf{Proof:}} We want to show that the individual terms in the series do not tend towards 0 $p$-adically. Consider all the elements of $\ZZ/p\ZZ$. We know that all multiples of $p$ will live in $\overline{0}$ and as such will have a $p$-adic absolute value less than 1. All other elements will have a $p$-adic absolute value of 1, since they contain a factor of $p^{0}$. Hence, the sum tends towards infinity, so the series does not converge.
	      	      \end{quote}
	      \end{enumerate}
\end{enumerate}
\end{document}
