\documentclass{hw}

% \usepackage{geometry}
\usepackage{amssymb,amsmath}
% \geometry{margin=1in}

% \newcommand{\ZZ}{\mathbb{Z}}
% \newcommand{\NN}{\mathbb{N}}

\begin{document}
\begin{enumerate}
\item If $F$ is a field in which 0 has a mutiplicative inverse, show $|F|=1$.
\begin{quote}
\textbf{\textit{Proof:}} Suppose $|F|>1$, and let $x\in F$ such that $x=0^{-1}$. Then $0x=1=0$. We know that under multiplication, any element in $F$ can be expressed as $1\times y\ \forall y\in F$, and since $1=0$, $1\times y = 0 \times y = 0$. Therefore the only element in $F$ is 0, and $|F|=1$
\end{quote}

\item Suppose that $n>1$ is an integer and let $\ZZ_{n}$ be equipped with addition and multiplication modulo $n$. Prove that $\ZZ_{n}$ is a field if and only if $n$ is prime.
\begin{quote}
In order to be a field, $\ZZ_{n}$ must fulfill the following axioms:
\begin{enumerate}
\item[\textbf{(A1)}] Addition is commutative on $F$.
\item[\textbf{(A2)}] Addition is associative on $F$.
\item[\textbf{(A3)}] There is a uniqe additive identity, called 0.
\item[\textbf{(A4)}] There is an additive inverse $-a$ for all $a\in F$.
\item[\textbf{(M1)}] Multiplication is commutative on $F$.
\item[\textbf{(M2)}] Multiplication is associative on $F$.
\item[\textbf{(M3)}] There is a unique multiplicative identity called 1.
\item[\textbf{(M4)}] There is an multiplicative inverse element $a^{-1}$ for every $a\in F$.
\item[\textbf{(D)}] For all $x,y,z\in F$, $x\cdot(y+z)=(x\cdot y)+(x\cdot z)$.
\item[\textbf{(ZO)}] The additive and multiplicative identity are distinct.
\end{enumerate}

\textbf{\textit{Proof:}} Suppose $n$ is prime. Then every element in $\ZZ_{n}$ is a unit, so $\ZZ_{n}$ is equipped with a multiplicative inverse. Since $\ZZ_{n}$ has addition and multiplication, we know it is commutative and associateve for both those operations. We know that $\ZZ_{n}$ contains 0, so it has the additive inverse. Addition and multiplication form the distributive law, so $\ZZ_{n}$ is equipped with the distributive law. Since $n>1$, $\ZZ_{n}$ $0\neq1$. Therefore, by defintion, $\ZZ_{n}$ is a field since it fulfills all the field axioms.\\\\
Suppose $\ZZ_{n}$ is a field with $n$ not prime. Since $n$ is not prime, then we can find an element $x\in\ZZ_{n}$ such that $gcd(n,x)\neq1$. Therefore there is an element in $\ZZ_{n}$ that is not a unit, so $\ZZ_{n}$ is not a field, since there is an element without an inverse.
\end{quote}

\item Suppose that $F$ is a finite field and $x\in F\setminus\{0\}$. Prove that there exists $n\in\NN$ such that $x^{n}=1$.
\begin{quote}
\textit{\textbf{Proof:}} Consider $x^{r}=x^{s}$ for $s\neq r\in\NN$ where $s>r$ without loss of generality. We know that for every element in $F$, there is a multiplicative inverse of that element, so we define $(x^{r})^{-1}=x^{-r}$. Then $x^{-r}x^{r}=x^{-r}x^{s}$ and $1=x^{s-r}$. Since $s>r$, $s-r\in\NN$, so we have found the $n$ we were looking for.
\end{quote}
\end{enumerate}
\end{document}
