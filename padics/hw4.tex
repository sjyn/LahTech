\documentclass{hw}
\begin{document}
\makeheader{4}

\begin{enumerate}
    \item Recall that $f(x)=x^2 + 1$ has two zeros $\alpha,\beta\in\ZZ_{5}$ satisfying $|\alpha-2|_{5}\leq 1/5$ and
    $|\beta-3|_{5}\leq 1/5$. Find the first three terms $b_{0}, b_{1}$, and $b_{2}$ of the \textit{p-adic} expansion
    of $\beta$.
    \begin{quote}
        Since $b=3$ satisfies $f(x)=0$, we know $b_{0}=3$. We can consider $\beta$ of the form
        \[
            \beta = b_{0} + 5b_{1} + 5^{2}b_{2} + \dots,
        \]
        where $0\leq b_{n} \leq 5$. To find $b_{1}$, we look at $f(\beta)\equiv 0\mod 5^{2}$. Then
        \begin{gather*}
            (3+5b_{1})^{2} + 1 \equiv 0\mod 25\\
            9 + 5b_{1} + 25b_{2}^{2} + 1 \equiv 0\mod 25\\
            10 + 5b_{1} \equiv 0\mod 25\\
            5b_{1}\equiv 15\mod 25\\
            \implies b_{1}=3.
        \end{gather*}
        Similarly, for $b_{2}$, we have
        \begin{gather*}
            (18+25b_{2})^{2} + 1 \equiv 0\mod 125\\
            74 + 25b_{2} + 1 \equiv 0 \mod 125\\
            25b_{2} \equiv 50\mod 125\\
            5b_{2}\equiv 10\mod 25\\
            \implies b_{2} = 2.
        \end{gather*}
        Therefore $\beta \approx 3 + 5\cdot3 + 5^{2}\cdot2$.
    \end{quote}

    \item User Hensel's Lemma to verify that $f(x)=x^3+1$ has a zero $\alpha\in\ZZ_{7}$. Find an integer $n$ such that
    $|\alpha-n|_{7} \leq 1/7$.
    \begin{quote}
        If we let $\alpha=3$, then $f(\alpha)=27+1\equiv 0\mod 7$. We also have that
        $f'(\alpha)=3(9)=27\not\equiv0\mod7$, so Hensel's Lemma tells us that there is a zero of the function. Let $n  10$. Then $|3-10|_{7}=|-7|_{7}=|7|_{7}=1/7 \leq 1/7$.
    \end{quote}

    \item Show that Hensel's Lemma fails to apply to the polynomial $f(x)=x^3+1$ in $\ZZ_{3}$. Is this sufficient to
    conclude that $f$ has no zeros in $\ZZ_{3}$?
    \begin{quote}
        Hensel's Lemma requires that $f'(x)\not\equiv 0 \mod p$. In this case, $f'(x)=3x^{2}$, and
        $3x^{2}\equiv0\mod 3$, so we cannot use it. This is not enough to say that $f$ has no zeros. Hensel's Lemma only tells us about the properties of a zero, but it does not guarantee the existence of a zero.
    \end{quote}
\end{enumerate}
\end{document}
