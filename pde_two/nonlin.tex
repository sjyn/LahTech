\section{Non Linear Equations}
\subsection{Brook's Equation}

Brooks Equation is the simplest form of an equation describing the movement of shockwaves. The equation
is given as follows:
\[
u_{t} - uu_{x} = 0.
\]
We will attempt to solve Brook's equation in a similar fashion to how we solved first order PDE's.
\begin{quote}
\textit{\textbf{Example:}} Find a solution to the equation
\[
\begin{cases}
u_{t} - uu_{x} = 0\\
u(x,0) = x
\end{cases}.
\]
\textit{\textbf{Solution:}} We will begin by noticing that
\[
{dx\over dt} = u(x,t).
\]
This implies that the solution depends on the function itself, and, as a result, we cannot solve this in
the typical way. We can, however, analyze the characteristic equation. Note that $u(x,t)$ is constant. We
know this, since
\[
{d\over dt}u(x(t),t)=
{\partial u\over\partial x}{\partial x \over \partial t} + {\partial u\over \partial t}=
0.
\]
This is not a trivial fact; it actually leads us to several conclusions that we can use:
\begin{enumerate}
\item Any associated characteristic curve is a straight line. We know this since...
\item ...the solution is constant on each characteristic curve. This leads us to...
\item ...the slope of each line is the same as the value of $u(x,t)$ on that line.
\end{enumerate}
We can use (3) to help us here. We will now take our initial condition, $\phi(x)=u(x,0)=x$, into
consideration. We know by (1) that there are infinitely many solutions that are straight lines, so
we can consider the case where we have a solution that passes through $(x,t)$ and $(x_{0},t_{0})$. Note
that $t_{0} = 0$. We can setup a relationship based on the slope $m$ as follows:
\begin{alignat*}{3}
m &= {x-x_{0}\over t-0}\qquad\qquad &&\text{by the definition of a slope}\\
&= {dx\over dt} &&\text{since the slope is constant}\\
&= u(x,t) &&\text{by the characteristic equation}\\
&= u(x_{0},0) &&\text{by (3)}\\
&= \phi(x_{0}) &&\text{by the definition of the initial value}.
\end{alignat*}
Now we have a very nice group of equalities, which we can manipulate as we please. We will take
\[
{x-x_{0}\over t-0} = \phi(x_{0}),
\]
which can be rearranged as
\[
x-x_{0} = t\phi(x_{0}).
\]
We also know that
\[
u(x,t) = \phi(x_{0}),
\]
so we would like to find a value for $\phi(x_{0})$. In our case, $\phi(x) = x$, so
\begin{align*}
&\qquad\ \ x-x_{0} = t\phi(x_{0})\\
&\implies x-x_{0} = tx_{0}\\
&\implies x = 2tx_{0}\\
&\implies x_{0} = {x\over 2t},\quad t\neq0.
\end{align*}
We had that $u(x,t) = \phi(x_{0})$, so
\[
u(x,t) = {x\over2t}.
\]
We are not done yet. Our solution for $u$ is only applicable for $t\neq0$, but we know that the initial
condition applies to $t=0$.
\end{quote}
