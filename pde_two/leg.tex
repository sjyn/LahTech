\documentclass{hw}

\title{Problems Involving Legendre Polynomials}
\date{}

\begin{document}
\maketitle
\begin{enumerate}
\item Show $\int_{-1}^{1}x^2p_{l}(x)\dx = 0$ for $l\geq 3$.
\begin{quote}
We know that the general form for a Legendre polynomial $p_{l}(x)$ is given by Rodrigues' Formula to be
\[
p_{l}(x)={1\over2^ll!}{d^l\over \dx^l}\left[(x^2-1)^l\right].
\]
We can use integration by parts to perform the integration:\\
\begin{center}
\begin{tabular}{c c c}
$(+)$ & $x^2$ & ${d^{l-1}\over\dx^{l-1}}\left[(x^2-1)^l\right]$\\
& & \\
$(-)$ & $2x$ & ${d^{l-2}\over\dx^{l-2}}\left[(x^2-1)^l\right]$\\
& & \\
$(+)$ & $2$ & ${d^{l-3}\over\dx^{l-3}}\left[(x^2-1)^l\right]$.\\
\end{tabular}
\end{center}
We now have that
\[
\int_{-1}^{1}x^2p_{l}(x)\dx =
x^2{d^{l-1}\over\dx^{l-1}}\left[(x^2-1)^l\right] -
2x{d^{l-2}\over\dx^{l-2}}\left[(x^2-1)^l\right] +
2{d^{l-3}\over\dx^{l-3}}\left[(x^2-1)^l\right].
\]
It is clear that the $l\geq3$ requirement is necessary, as any $l<3$ would lead to a negative
order derivative. We now need to show that this value is equal to 0; to do so, we will start by
substituting $Q(x) = (x^2-1)^l$. We now have
\begin{align*}
&x^2{d^{l-1}\over\dx^{l-1}}\left[(x^2-1)^l\right] -
2x{d^{l-2}\over\dx^{l-2}}\left[(x^2-1)^l\right] +
2{d^{l-3}\over\dx^{l-3}}\left[(x^2-1)^l\right]\\
&= x^2{d^{l-1}\over\dx^{l-1}}Q(x) - 2x{d^{l-2}\over\dx^{l-2}}Q(x) + 2{d^{l-3}\over\dx^{l-3}}Q(x)\\
&= x^2Q^{(l-1)}(x) - 2xQ^{(l-2)}(x) + 2Q^{(l-3)}(x).
\end{align*}
Let's consider the case where $l=3$ to show that this integral is indeed 0. Substituting 3 for $l$ yields
\[
x^2Q^{''}(x) - 2xQ^{'}(x) + 2Q(x).
\]
Through some careful integration, we find that
\begin{align*}
Q &= (x^2-1)\\
Q^{'} &= 6x(x^2-1)^2\\
Q^{''} &= 24x^2(x^2-1)+6(x^2-1)^2.
\end{align*}
Notice that in every term of each polynomial there is an $(x^2-1)^n$ term, where $n\in\NN$. If we factor
this term, we find that we have $(x+1)^n(x-1)^n$. The original integral was over the interval $(-1,1)$,
which yields a 0 in each term when we evaluate the anti-derivative at those bounds. Hence,
\begin{align*}
x^2Q^{''}(x) - 2xQ^{'}(x) + 2Q(x)\Big|_{-1}^{1} &= x^2(0)-2x(0)+2(0)\\
&= 0.
\end{align*}
We have shown that for $l=3$, the value of the integral is 0. However, we need to show that
$\forall l\geq3$, the value of the integral is 0. We can use the Leibniz Rule to express the
$l^{\text{th}}$ derivative in terms of a summation.
\[
(f(x)g(x))^{(n)} = \sum_{k=0}^{n}{n\choose k}f^{(n-k)}(x)g^{(k)}(x).
\]
We are trying to prove that $x^2Q^{(l-1)}(x) - 2xQ^{(l-2)}(x) + 2Q^{(l-3)}(x) = 0$ for all $l\geq3$.
We know $Q=(x^2-1)^l$, so we can take one derivative of $Q$, which gives us $2xl(x^2-1)^{l-1}$. Let
$f(x)=2xl$ and $g(x)=(x^2-1)^{l-1}$. Our goal is to find the $l^{\text{th}}$ derivative of $Q$.
By the Leibniz Rule we have
\[
(fg)^{(l)} = \sum_{k=0}^{l}{l\choose k}f^{(l-k)}g^{(k)}.
\]
Consider the case where we take another derivative. We know that we will have a $(x^2-1)^{(l-2)}$ term.
Again, we can take another derivative, which would provide us with $(x^2-1)^{(l-3)}$. We know that this
term will continue to appear until $k>l$. However, the Leibniz Rule will stop when $k=l$. Therefore,
every product $f^{(l-k)}g^{(k)}$ will have a $(x^2-1)^{n}$ term where $n$ is the $l-k$ power.
We can also rewrite $(x^2-1)^{n}$ as $(x-1)^n(x+1)^n$, so we now know that every term in the summation
will have an $(x-1)^n(x+1)^n$ term. Recall that we are evaluating this term from $-1$ to $1$, so
when we ultimately substitute into the equation, we will either have $(x-1)^n = 0$ or $(x+1)^n = 0$.
Therefore, we can conclude that every term in the summation will be 0. This implies that the
$l^{\text{th}}$ derivative of $Q$ is 0 for any $l \geq 3$. Therefore
$x^2Q^{(l-1)}(x) - 2xQ^{(l-2)}(x) + 2Q^{(l-3)}(x) = 0$.
\end{quote}



\newpage
\item We want to find $a_{l}$ in the series
\[
f(x) = \sum_{l=0}^{\infty}a_{l}p_{l}(x)
\]
where
\[
f(x)=
\begin{cases}
x,\ 0\leq x < 1\\
0,\ -1 < x \leq 1
\end{cases}.
\]
We know that we can express $a_{l}$ as
\[
a_{l} = {%
\int_{-1}^{1}f(x)p_{l}(x)\dx%
\over%
\int_{-1}^{1}p_{l}^{2}(x)\dx%
}.
\]
Since $f$ is piecewise defined to be $0$ on $-1 < x \leq 1$, we are only dealing with
\[
a_{l} = {%
\int_{0}^{1}xp_{l}(x)\dx%
\over%
\int_{-1}^{1}p_{l}^{2}(x)\dx%
}.
\]
We will begin by analyzing the numerator. We are solving
\[
\int_{0}^{1}xp_{l}(x)\dx.
\]
Using integration by parts gives us
\begin{align*}
&x{1\over2^ll!}{d^{l-1}\over dx^{l-1}}\left[(x^2-1)^l\right]-
{1\over2^ll!}{d^{l-2}\over dx^{l-2}}\left[(x^2-1)^l\right]\Big|_{x=0}^{x=1}\\
&= xp_{l-1}(x)-p_{l-2}(x)\Big|_{x=0}^{x=1}\\
&= \left[1\cdot p_{l-1}(1) - p_{l-2}(1)\right] -\left[0-p_{l-2}(0)\right]\\
&= p_{l-2}(0).
\end{align*}
We can express $p_{l}(0)$ as
\[
p_{l}(0) =
\begin{cases}
{(-1)^{l/2}\over2^l}{l\choose l/2}\quad x\in\mathbb{E}\\
0\qquad\qquad\quad\ x\in\mathbb{O}
\end{cases}.
\]
We will consider even $l$ here, since all odd $l$ will produce a 0. We can now express $a_{l}$ as
\begin{align*}
a_{l} &= {(2l+1)p_{l-2}(0)\over2}\\
&= {4(2l+1)\over l!}.
\end{align*}
The Fourier series is now
\[
\sum{4(2l+1)\over l!}p_{l}(x)
\]
for even $l$.
\end{enumerate}

\end{document}
