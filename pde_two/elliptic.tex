\documentclass[notitlepage]{hw}
\title{Elliptic Functions}
\date{}
\author{}

\begin{document}
\maketitle

\section{Introduction}

The Jacobi Elliptic functions come from the result of integrating certain algebraic functions. The
functions arose due to the lack on any elementary antiderivative to
\[
u=\int_{0}^{\phi}{d\theta\over\sqrt{1-k^{2}\sin^{2}{\theta}}}.
\]
The Jacobi Elliptic functions provide a way to express the solution to this integral via the properties
\begin{align*}
sn(u)&=sn\left(\int_{0}^{\phi}{d\theta\over\sqrt{1-k^{2}\sin^{2}{\theta}}}\right)=\sin{\phi}\\
cn(u)&=cn\left(\int_{0}^{\phi}{d\theta\over\sqrt{1-k^{2}\sin^{2}{\theta}}}\right)=\cos{\phi}\\
dn(u)&=dn\left(\int_{0}^{\phi}{d\theta\over\sqrt{1-k^{2}\sin^{2}{\theta}}}\right)=
\sqrt{1-k^{2}\sin^{2}{\phi}}.
\end{align*}

The elliptic functions end up being applicable to more than just finding the antiderivative of $u$.
We can express the solutions to several differential equations in terms of these functions, much like
we can with $\sin{}$ and $\cos{}$. For example, the system
\[
\begin{cases}
\dot{x} = yz\\
\dot{y} = -zx\\
\dot{z} = -k^{2}xy
\end{cases}
\qquad
\text{where}
\qquad
\begin{cases}
x(0)=0\\
y(0)=1\\
z(0)=1
\end{cases}
\]
can be solved by these elliptic functions, when subject to the initial conditions. More interestingly,
we can find the solutions for several ODEs via the elliptic functions.

\subsection{Applications to ODEs}
Consider the second order ODE
\[
\begin{cases}
\ddot{x} = (1-x^{2})(1-k^{2}x^{2})\\
\dot{x}(0)=1\\
x(0)=0
\end{cases}.
\]
Recall that we defined the Jacobi Elliptic functions as solutions to a prior system of ODEs. We had
the relationship that $\dot{x}=yz$, where $y$ and $z$ were functions of $t$. Taking another derivative
yields $\ddot{x}=\dot{y}z+y\dot{z}$. We additionally have values for both $\dot{y}$ and $\dot{z}$.
We can further substitute, which leads us to
\[
\ddot{x}=\dot{y}z+y\dot{z}= -xz^{2}-k^{2}xy^{2}.
\]

\end{document}
