\documentclass[notitlepage]{hw}
\title{Elliptic Functions}
\date{}
\author{}

\begin{document}
\maketitle

We will begin the discussion of Elliptic functions by analyzing the familiar trig functions, $\sin$ and
$\cos$. Consider the following system of ODEs:
\[
\begin{cases}
\frac{\dx}{\dt} = y\\
\frac{\dy}{\dt} = -x
\end{cases}
\]
This system has the solution $x(t)=\cos{t}$ and $y(t)=\sin{t}$. We will use this definition to formulate
the Elliptic functions.\\

We define a system of ODE's by letting $k\in(0,1)$ and $t\in\RR$ representative of time. We can define
three functions, $sn(t,k),cn(t,k),dn(t,k)$, as solutions to the system
\[
\begin{cases}
\frac{\dx}{\dt} = yz\\
\frac{\dy}{\dt} = -zx\\
{\dz\over\dt}=-k^{2}xy
\end{cases}
\implies
\begin{cases}
{d\over\dt}sn(t,k)=cn(t,k)dn(t,k)\\
{d\over\dt}cn(t,k)=-dn(t,k)sn(t,k)\\
{d\over\dt}dn(t,k)=-k^{2}sn(t,k)cn(t,k)
\end{cases}
\]
The functions are not arbitrary; they have special properties. The first one we notice is that
\[
\lim_{k\to0^{+}}sn(t,k)=\sin{t}
\qquad\lim_{k\to0^{+}}cn(t,k)=\cos{t}
\qquad\lim_{k\to0^{+}}dn(t,k)=1.
\]
In addition, we have that
\[
\lim_{k\to1^{-}}sn(t,k)=\tanh{t}
\qquad\lim_{k\to1^{-}}cn(t,k)=\sech{t}
\qquad\lim_{k\to1^{-}}dn(t,k)=\sech{t}.
\]

\end{document}
