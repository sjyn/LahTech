\section{High Order PDEs}

We can begin by talking about the heat equation in 2D space. We have the following
\[
u_{t}=ku_{xx},
\]
which, under certain conditions, yields the general solution
\[
u(x,t) = \sum_{n=0}^{\infty}b_{n}e^{-k\left({n\pi\over l}\right)^{2}t}\sin{\left({n\pi x\over l}\right)}.
\]
If we consider the laplacian, $\nabla^{2}u$, we can express the heat equation as
\begin{align*}
u_{t}&=k\nabla^{2}u\\
&=k(u_{xx} + u_{yy})\text{ in }\RR^{3}.
\end{align*}
We can now begin to solve a three-dimensional heat equation.
\begin{quote}
\textbf{\textit{Example:}} Solve the following heat equation for $u(x,t)$:
\[
\begin{cases}
u_{t} = a^{2}(u_{xx} + u_{yy})\\
u(x,t) = f(x,y)\\
u_{2D} = 0\qquad\text{(there is no external source of energy in the system)}
\end{cases}.
\]
\textit{\textbf{Solution:}} We begin by noticing that this equation is taking place on a cylinder.
much like the process when solving Laplace's equation on a disk, we can transform our PDE from
cartesian coordinates to polar coordinates. The time variable, $t$ is not affected by this
transformation, and we know
\[
\nabla^{2} u=u_{xx} + u_{yy} = u_{rr} + {1\over r}u_{r} + {1\over r^{2}}u_{\theta\theta}.
\]
Applying our transformation yields the new PDE
\[
\begin{cases}
u_{t}=a^{2}\left(u_{rr} + {1\over r}u_{r} + {1\over r^{2}}u_{\theta\theta}\right)\\
u_{t=0}=f(r,\theta)\\
u_{r=r}=0
\end{cases}.
\]
We will now have a general solution of $u(r,\theta,t)$, but we cannot jump immediately to a Fourier
series solution. Instead, we will use separation of variables to provide us with a solution. We have
\begin{gather*}
u = V(r,\theta)T(t)\\
{T^{'}\over a^{2}T}= {V_{rr} + {1\over r}V_{r}+{1\over r^{2}}V_{\theta\theta}\over V} = -k^{2}.
\end{gather*}
We now have two ODEs to deal with:
\[
T'+a^{2}k^{2}T=0
\qquad\qquad\text{and}\qquad\qquad
V_{rr} + {1\over r}V_{r}+{1\over r^{2}}V_{\theta\theta}+k^{2}V=0
\]
We can again apply separation of variables to $V(r,\theta)$, and we find that we have two more resulting
ODEs.
\[
\begin{cases}
\theta''+\mu\theta=0\\
r^{2}R''+rR'+(k^{2}r^{2}-\mu)R=0
\end{cases}.
\]
The equation involving $\theta$ is a simple ODE; the equation involving $R$ will take a little more
work to solve. We will discover a method to solve this ODE in the next section.
\end{quote}

\subsection{Power Series Solutions to ODEs}

From the previous problem, we saw that we need to develop a method to solve
\begin{gather*}
x^{2}y''+xy'+(x^{2}-n^{2})y = 0\\
p(x)y''+q(x)y'+r(x)y = 0.
\end{gather*}
We will attempt to solve this using power series solutions to ODEs. There is, however, an issue that
we will face. Since $p(x)=x^{2}$, we cannot recover $y''$ near zero ($p(0)=0$). The regular power
series expansion we would use would take the form
\[
y(x) = \sum_{n=0}^{\infty}a_{n}x^{n},
\]
but we cannot use this method since $y''$ is lost near zero. We can, however modify our sequence
in such a way as to preserve $y''$. Doing so gives us the sequence
\[
y(x)=x^{\alpha}\sum_{n=0}^{\infty}a_{n}x^{n},
\]
where $\alpha$ is called the characteristic value. The strategy here will be to determine the values
of $\alpha$, and then determine the value of $a_{n}$.
\begin{quote}
\textbf{\textit{Example:}} Determine the solution to the following ODE
\[
x^{2}y''+x^{2}y'+ry=0.
\]
\textit{\textbf{Solution:}} We start by saying
\[
y(x) = \sum_{n=0}^{\infty}a_{n}x^{n+\alpha},
\]
so
\begin{align*}
y'(x) &= \sum_{n=0}^{\infty}(n+\alpha)a_{n}x^{n+\alpha-1}\\
y''(x) &= \sum_{n=0}^{\infty}(n+\alpha)(n+\alpha-1)a_{n}x^{n+\alpha-2}.
\end{align*}
We now have enough to substitute back into our original ODE.
\begin{gather*}
x^{2}\sum_{n=0}^{\infty}(n+\alpha)(n+\alpha-1)a_{n}x^{n+\alpha-2} +
x^{2}\sum_{n=0}^{\infty}(n+\alpha)a_{n}x^{n+\alpha-1}+
r\sum_{n=0}^{\infty}a_{n}x^{n+\alpha} = 0\\
\end{gather*}
\end{quote}
