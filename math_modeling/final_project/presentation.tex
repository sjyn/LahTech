\documentclass{beamer}

\usepackage{multirow}

\AtBeginSection[]
{
  \begin{frame}
    \frametitle{Table of Contents}
    \tableofcontents[currentsection]
  \end{frame}
}

\title{Modeling Zombies and Infection}
\author{Ricky Marske and Steven Rosendahl}
\date{}

\usetheme{metropolis}

\begin{document}

\frame{\titlepage}

\section{A Simple Model}

\begin{frame}{Population Decay}
\begin{itemize}
\item Population decay can be modeled by
\[y=a(1-r)^{x}\]
\pause
\item $a:$ Initial amount of population
\pause
\item $r:$ Decay rate
\pause
\item $x:$ The amount of time that has passed
\end{itemize}
\end{frame}

\begin{frame}{A Zombie Model}
\begin{itemize}
\item Population decay model is a good starting point, but is not strong enough to model zombie
outbreak
\pause
\item Final size population model
\pause
\begin{itemize}
\item Size of the population as $t\to\infty$.
\pause
\item Let $S_{\infty}$ be final size of susceptible population
\pause
\item Let $I_{\infty}$ be final size of infected population
\pause
\item Let $R_{\infty}$ be final size of removed population
\pause
\item Left with the system
\begin{gather*}
\begin{cases}
S'=-\beta(t)SI\\
I'=\beta(t)SI-vI\\
R'=vI
\end{cases}
\end{gather*}
\end{itemize}
\end{itemize}
\end{frame}

\section{Adding Complexity}

\begin{frame}{Probability}
\begin{itemize}
\item In the initial model, we assumed
\pause
\begin{enumerate}
\item Non-Infected would have no response to infected
\pause
\item No one was immune
\pause
\item No one could survive the virus or be cured
\end{enumerate}
\pause We will look at each one of these additional features individually
\end{itemize}
\end{frame}

\begin{frame}{Reaction to Zombies}
\begin{itemize}
\item The heat equation can be used to model the spread of zombies
\pause
\item Happens when humans randomly flee from zombies
\end{itemize}
\[
\frac{\partial Z}{\partial t} (x,t) =
D \frac{\partial^2 Z}{\partial x^2} (x,t) \;\;\;\;\;\; \text{(the partial differential equation)}
\]
\pause
\[
Z(x,0) = \begin{cases}
Z_0 & \text{for $0 \leq x \leq 1$}\\
0 & \text{for $x > 1$}
\end{cases} \;\;\;\;\;\;\ \text{(the initial condition)}
\]
\pause
\[
\frac{\partial Z}{\partial x} (0,t) = 0 =
\frac{\partial Z}{\partial x} (L,t) \;\;\;\;\; \text{(zero-flux boundary condition)}
\]
\noindent\\
\noindent\\
\noindent\\
\pause
\[
Z(x,t) = \frac{Z_0}{L} + \sum_{n=1}^{\infty}
\frac{2Z_0}{n \pi}\sin\left(\frac{n \pi}{L}\right)
\cos\left(\frac{n \pi}{L} x\right) e^{\left(-\left(\frac{n \pi}{L}\right)^2 Dt\right)}
\]
\end{frame}

\begin{frame}{Non-Infected Responses}
\begin{itemize}
\item Humans will have a natural response to virus outbreak
\pause
\begin{itemize}
\item One reaction may be to form groups away from the infected
\pause
\item Prevalent in a zombie scenario
\pause
\item Twitch.tv provides a real life example of this
\pause
\item Popular users are like the groups of non-infected
\pause
\item We can analyze what happens when these large groups of non-infected are suddenly hit with the
zombie virus
\end{itemize}
\pause
\item This behavior can be modeled using NetLogo
\end{itemize}
\end{frame}

\begin{frame}{NetLogo Zombie Model}
\begin{itemize}
\item Four different scenarios
\pause
\begin{enumerate}
\item Zombies age and die
\pause
\item Virus dying out in carriers
\pause
\item Vaccination combating virus
\pause
\item Humans dying from old age
\end{enumerate}
\end{itemize}
\begin{center}
\pause
\includegraphics[scale=0.5]{classes}
\end{center}
\end{frame}

\begin{frame}{Societal Groups}
\begin{center}
\includegraphics[scale=0.25]{groups}
\includegraphics[scale=0.6]{grouping}
\end{center}
\end{frame}

\begin{frame}{Twitch Zombie Outbreak}
\begin{itemize}
\item We created a program to ``infect" twitch users
\pause
\begin{enumerate}
\item Start with initial user/users
\pause
\item Mark all users in that chat as infected
\pause
\item Look through all the infected users
\pause
\begin{itemize}
\item If the user is streaming, then infect all their viewers in their chat
\end{itemize}
\pause
\item Repeat indefinitely
\end{enumerate}
\pause
\item We expect to see bursts of infected after lulls of no infected
\pause
\item This can be modeled by INSERT EQUATION HERE
\[
\]
\end{itemize}
\end{frame}

\begin{frame}{Immunities}
\begin{itemize}
% \item Not every person is necessarily susceptible to a disease
% \pause
% \item There is a chance that someone in contact with an infected will not become infected
% \pause
\item We can model this with a modified version of the Twitch program
\pause
\begin{enumerate}
\item Follow the same procedure again, but mark a user as either infected or immune
\pause
\item If the user is immune then they can not infect their users
\pause
\item Does not account for carriers of the disease
\end{enumerate}
\end{itemize}
\end{frame}

% \begin{frame}{Immunities}
% \begin{itemize}
% \item This behavior can be modeled in NetLogo
% \end{itemize}
% \end{frame}

\begin{frame}{Cures and Survival}
\begin{itemize}
\item The time at which a cure is introduced affects the model
\pause
\item Can be represented by a wave equation:
\[
u_{tt}-k^{2}u_{xx}=0
\]
\pause
\item To show the offset, we have
\[
u_{tt}-k^{2}u_{xx}=\zeta
\]
\pause
\item $\zeta$ represents the time offset
\pause
\item This yields the solution
\[
u(x,t)=\zeta+\sum_{n=1}^{\infty}(k_{1}\sin{t}+k_{2}\cos{t})\sin{n\pi x}
\]
\end{itemize}
\end{frame}

\begin{frame}{Cures and Survival}
\begin{center}

\begin{minipage}{0.4\textwidth}
\includegraphics[scale=0.3]{cure_01}\\
$n=1$
\end{minipage}
\begin{minipage}{0.4\textwidth}
\pause\includegraphics[scale=0.3]{cure_02}\\
$n=100$
\end{minipage}

\begin{minipage}{0.4\textwidth}
\pause\includegraphics[scale=0.3]{cure_0201}\\
$n=500$
\end{minipage}
\begin{minipage}{0.4\textwidth}
\pause\includegraphics[scale=0.3]{cure_03}\\
$n=1000$
\end{minipage}

\end{center}
\end{frame}

\begin{frame}{Cures and Survival}
\[
u(x,t)=\zeta+\sum_{n=1}^{\infty}(k_{1}\sin{t}+k_{2}\cos{t})\sin{n\pi x}
\]
\begin{itemize}
\item There is a saddle regardless of $\zeta$
\pause
\item The system will move towards the saddle no matter what
\end{itemize}
\end{frame}

\begin{frame}{References}

\end{frame}

\end{document}
