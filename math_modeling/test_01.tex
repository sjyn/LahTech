% pg 21, #33,34,35
% pg 61, #29
% pg 100, #8,11,1,2
% pg 138, #6,14,29
% pg 176, #4,5,6,13
% pg 224, #3,4
\documentclass{hw}

\usepackage{cancel}

\begin{document}

\section{Formulas/Methods}
\begin{enumerate}
\item Population Growth: $P = P_{0}e^{at}$
\item Logistic Population Growth: $P^{'}=bP({a\over b}-P)$
\item Account Balance: $B_{n}=(1+r)^nB_{0}$
\end{enumerate}

\newpage

\section{Mathematical Models}
\begin{enumerate}
\item A bank charges 18\% annual interest on credit card balances. What is the value of an original
balance of \$1,000 after 1 year if the bank computes new balances
\begin{enumerate}
\item once a year?
\begin{quote}
We can use the equation $B_{n} = (1+r)B_{n-1}$ to calculate the balance. The original balance was \$1000,
so after 1 year, we have $B_{1} = (1+0.18)(\$1000)$, so $B_{1}=\$1180$.
\end{quote}
\item twice a year?
\begin{quote}
Similarly, we can solve the problem using the same equation as above. We know that after the first 6
months, the balance is \$1180. If we calculate again, we find that the new balance is $B_{2}=\$1392.40$.
\end{quote}
\item every three months?
\begin{quote}
After the first 8 months, the balance is \$1392.40. Calculating again yields $B_{3}=\$1643.03$.
\end{quote}
\item every week?
\begin{quote}
In this case, it helps to come up with a more generalized model, since there are 52 weeks in a year.
We can abstract the model into $B_{n}=(1+r)^nB_{0}$, so we have that $B_{52}=(1+0.18)^{52}(\$1000)$,
so $B_{52} = \$5468451.69$.
\end{quote}
\item every day?
\begin{quote}
We can again use our abstracted model, so we have $B_{365} = (1+0.18)^{365}(\$1000)$, so
$B_{365} = \$1.73\times10^{29}$.
\end{quote}
\end{enumerate}
\end{enumerate}

\newpage
\section{Stable and Unstable Arms Races}
\begin{enumerate}
\item Determine the outcome of an arms race governed by the Richardson model:
\[
\begin{cases}
{dx\over dt}=10y-14x-12\\
{dy\over dt}=8x-4y-24
\end{cases}
\]
when the initial level is
\begin{enumerate}
\item (4,4)
\item (13,6)
\end{enumerate}
\begin{quote}
We will begin by finding a general solution to the provided system of ODE's. As it stands now, we
need to find a substitution that will allow us to solve the system. We can do so by finding the stable
point of the system.
\[
\left[
\begin{array}{c c | c}
-14 & 10 & 12\\
8 & -4 & 24
\end{array}
\right]
\to
\left[
\begin{array}{c c | c}
1 & 0 & 12\\
0 & 1 & 18
\end{array}
\right].
\]
We can make the substitution of $\alpha = x - 12$ and $\gamma = y - 18$. If we substitute in terms
of $\alpha$ and $\gamma$ we get the system
\[
\begin{cases}
\alpha' = 10\gamma - 14\alpha\\
\gamma' = 8\alpha - 12\gamma
\end{cases}.
\]
We can solve this. Our $A$ matrix is
\[
\left[
\begin{array}{c c}
-14 & 10\\
8 & -12
\end{array}
\right]
\implies
\left[
\begin{array}{c c}
-14-\lambda & 10\\
8 & -12-\lambda
\end{array}
\right].
\]
Our eigenvalues can be found by
\begin{gather*}
\lambda^2 -(a+d)\lambda+(ad-bc) = 0\\
\lambda^2 + 26\lambda+88 = 0\\
\lambda = -22\qquad \lambda = -4.
\end{gather*}
Our first eigenvector is given by
\[
\left[
\begin{array}{c c}
-14+22 & 10\\
8 & -12+22
\end{array}
\right]
\to
\left[
\begin{array}{c c}
8 & 10\\
8 & 10
\end{array}
\right]
\to
\left[
\begin{array}{c c}
1 & {5\over 4}\\
0 & 0
\end{array}
\right]
\implies
\vv{\lambda}_{1} =
\left[
\begin{array}{c}
1\\
-{4\over5}
\end{array}
\right].
\]
We can find the second eigenvector in a similar fashion:
\[
\vv{\lambda}_{2} =
\left[
\begin{array}{c}
1\\
1
\end{array}
\right].
\]
We now have solutions in terms of $\alpha$ and $\gamma$.
\[
\begin{cases}
\alpha = c_{1}e^{-22t}+c_{2}e^{-4t}\\
\gamma = -{4\over5}c_{1}e^{-22t} + c_{2}e^{-4t}
\end{cases}.
\]
We can back substitute, and we find that $x = \alpha + 12$ and $y = \gamma + 18$, so
\[
\begin{cases}
x = c_{1}e^{-22t}+c_{2}e^{-4t} + 12\\
y = -{4\over5}c_{1}e^{-22t} + c_{2}e^{-4t}+18
\end{cases}.
\]
Finally, we can apply our initial conditions, as specified by the problem. With the initial vector
$(4,4)$, we have that $c_{1} = 10/3$ and $c_{2} = -34/3$.
\end{quote}
\end{enumerate}

\newpage
\section{Single Species Ecological Models}
\begin{enumerate}
\item The rate of growth of a certain population of bacteria in a culture is directly proportional
to the size of the population. If an experiment begins with 1,000 bacteria and one hour later
the count is 1,500 bacteria, then how many bacteria are present at the end of 24 hours?
\begin{quote}
We can model population growth with the equation $P = P_{0}e^{rt}$. To find the rate, we can solve
for $r$, which gives us
\[
r = {\log{P\over P_{0}}\over t}.
\]
Using the information provided, we have that
\[
r = {\log{1500\over 1000}\over 1} = 0.4054.
\]
After 24 hours, we will have
\[
P = 1000e^{(0.4054)(24)} = 16834112.19.
\]
\end{quote}

\item Suppose that 20 years ago the population of a town was 2,000, and that the population increased
continuously at a rate proportional to the existing population. If the population of the town is now
6,000, what has been the rate of growth?
\begin{quote}
If we take 20 years ago to be $t=0$, then we have $P_{0} = 2000$. We can similarly find the rate
as above. Doing so yields $r = 0.0549$.
\end{quote}

\item If the population of a country is undergoing exponential growth at a rate of r percent per
year, show that the population doubles every $(\log{2})/r$ years. This number is called the
``doubling time." Compute the doubling time if r = 2.
\begin{quote}
We know that population growth can be modeled by the equation $P = P_{0}e^{rt}$. We can divide and
take the natural log of both sides to get $\log{(P/P_{0})} = rt$. $P$ is the current population,
so if it double the original, $P = 2P_{0}$. This yields $\log{(2)} = rt$, or $(\log{(2)})/r = t$.
When $r=2$, we have $t \approx 0.34$.

\item Assume that the U.S. population has grown exponentially. Estimate the growth rate using
each of the following years in place of the year 1830 as done in the text.
\begin{enumerate}
\item 1800
\begin{quote}
From the text, we have that the population of the United States can be modeled by the equation
\[
P(t) = 3.929e^{0.029643(t - 1790)}.
\]
For the year 1800, we have that $P(t) = 5.28$ million.
\end{quote}
\item 1850
\[
P(t) = 23.25\text{ million}.
\]
\item 1900
\[
P(t) = 102.32\text{ million}.
\]
\item 1970
\[
P(t) = 814.44\text{ million}.
\]
\end{enumerate}
\end{quote}
\end{enumerate}

\newpage
\section{Interacting Species Ecological Models}
\begin{enumerate}
\item Show that $y=0$, $x =e^{at}$ is a solution of both predator-prey and competitive hunters models. What
do the orbits look like? In which direction are they traced out? Answer the same questions if it is
specified that $x = 0$ is one of the functions in a solution.
\begin{quote}
The predator-prey model can be written as
\[
\begin{cases}
{dx\over dt} = ax-bxy\\
{dy\over dt} = mxy - ny
\end{cases}.
\]
We can simply plug into the solutions into the equations to check if they are valid. Since $y=0$,
$y'=0$, and therefore $0=mx(0)-n(y) = 0$. We also have that $x'=ae^{at}$, so we have
\begin{gather*}
ae^{at} = ae^{at} - b(e^{at})(0)\\
ae^{at} = ae^{at}.
\end{gather*}
We can do the same procedure for $x=0$, and we will find that the answer is still the same.
\end{quote}
\end{enumerate}

% \newpage
% \section{Tumor Growth Models}
% \begin{enumerate}
% \item
% \end{enumerate}

\newpage
\section{Voting Methods}
\begin{enumerate}
\item Consider the following distribution of votes among 55 voters.\\
\begin{center}
\begin{tabular}{c | c}
Number of Voters & Ranking\\
\hline
18 & $t>m>h>k>c$\\
\hline
12 & $c>h>m>k>t$\\
\hline
10 & $k>c>h>m>t$\\
\hline
9 & $m>k>h>c>t$\\
\hline
4 & $h>c>m>k>t$\\
\hline
2 & $h>k>m>c>t$
\end{tabular}
\end{center}
Determine the outcome of the election using the Plurality, Pairwise Comparison, Borda, and Hare methods.
\begin{quote}
\begin{enumerate}
\item \textit{Plurality}: In this case, $t$ has the most votes, so $t$ wins the election.\\
\item \textit{Pairwise}: We have to compare every possible combination of matches and determine the
outcome from there. We have
\begin{align*}
&t\ vs.\ m:\textbf{ m} \qquad &&t\ vs.\ h:\textbf{ h} \qquad &&&t\ vs.\ k:\textbf{ k} \qquad &&&&t\ vs.\ c: \textbf{ c}\\
&m\ vs.\ h:\textbf{ h}  &&m\ vs.\ k:\textbf{ m}  &&&m\ vs.\ c: \textbf{ m} &&&&\\
&h\ vs.\ k:\textbf{ h}  &&h\ vs. c:\textbf{ h} &&& &&&&\\
&k\ vs.\ c:\textbf{ k} && &&& &&&&
\end{align*}
In this case, $h$ has 4 votes, $m$ has 3 votes, $k$ has 2 votes, $c$ has 1 vote, and $t$ has 0 votes,
so $h$ is the winner.\\
\item \textit{Borda}: Under the Borda method, we consider how many points each candidate receives based
on their ranking in each row. The first place candidate in each row receives 4 times the number of votes
points, the next receives 3 times the number of votes points, and so on to the last candidate, who receives
0 points.\\
\begin{center}
\begin{tabular}{c | c | c | c | c | c}
& $t$ & $m$ & $h$ & $k$ & $c$\\
\hline
18 $\times$ & 4 & 3 & 2 & 1 & 0\\
\hline
12 $\times$ & 0 & 2 & 3 & 1 & 4\\
\hline
10 $\times$ & 0 & 1 & 2 & 4 & 3\\
\hline
9 $\times$ & 0 & 4 & 2 & 3 & 1\\
\hline
4 $\times$ & 0 & 2 & 4 & 1 & 3\\
\hline
2 $\times$ & 0 & 2 & 4 & 3 & 1\\
\hline
\textbf{Total: }& 72 & 136 & 134 & 107 & 101
\end{tabular}
\end{center}
\noindent\\
\noindent In this case, $m$ had the most votes at 136, so $m$ wins the election.\\
\item \textit{Hare}: The Hare method focuses on finding the majority vote needed to win the election.
There were 55 voters in this example, so the majority would be 28 votes. In the Hare method, the candidates
have runoffs until someone reaches a majority vote.\\
\begin{center}
\begin{tabular}{c | c | c | c | c}
& 1 & 2 & 3 & 4\\
\hline
$t$ & 18 & 18 & 18 & 18\\
\hline
$c$ & 12 & 16 & $\cancel{16}$ & \\
\hline
$m$ & 9 & $\cancel{9}$ & & \\
\hline
$h$ & $\cancel{6}$ & & & \\
\hline
$k$ & 10 & 12 & 21 & 22\\
\hline
\end{tabular}
\end{center}
\noindent\\
\noindent Here, neither party reached 28, but $k$ has more votes than $t$, so $k$ wins.
\end{enumerate}
\end{quote}

\end{enumerate}

\end{document}
