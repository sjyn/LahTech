\documentclass{hw}

\begin{document}
\section*{Stochastic Modeling}

A stochastic matrix is a square matrix whose elements are probabilities and where each row sums to
one. One special example of a stochastic matrix is a matrix of transition probabilities.

\begin{quote}
\textit{\textbf{Example}}\\
Consider the case where a student enters a college as a one major, but changes to another.

\begin{center}
\begin{tabular}{c | c c c}
& STEM & Humanities & Business\\
\hline
STEM & 0.6 & 0.3 & 0.1\\
Hum. & 0.05 & 0.7 & 0.25\\
Bus. & 0.2 & 0.3 & 0.5
\end{tabular}
\end{center}

In this matrix, each $a_{ij}$ entry represents the probability that a student will change to that
major (or keep that major if it is the same). Suppose we have 1000 freshmen that enter the college
\[
x_{0}=[300,500,200]
\]
in the order STEM, Humanities, and Business. If we want to know the number of STEM majors in the
class's sophomore year, we can multiply our initial vector by the STEM column of our matrix:
\begin{align*}
\text{STEM Majors}&=(0.6)(300)+(0.05)(500)+(0.2)(200)\\
&=245\text{ students}\\
\text{Hum. Majors}&=(0.3)(300) + (0.7)(500) + (0.3)(200)\\
&= 500\text{ students}\\
\text{Bus. Majors}&=(0.1)(300) + (0.25)(500) + (0.5)(200)\\
&=255\text{ students}
\end{align*}
We can continue this process for each year that the students attend the college.
\end{quote}

This process for creating a sequence of vectors is called a Markov Chain. In general, we can
start with a vector $\bf{x}_{0}$ and a matrix $P$, and we get that
\begin{alignat*}{3}
\bf{x}_{1}&=\bf{x}_{0}\cdot P\\
\bf{x}_{2}&=\bf{x}_{1}\cdot P &&= \bf{x}_{0}\cdot P\cdot P\\
\bf{x}_{3}&=\bf{x}_{2}\cdot P &&= \bf{x}_{0}\cdot P\cdot P\cdot P,\\
\end{alignat*}
which yields the recurrence relation
\[
\bf{x}_{n}=\bf{x}_{0}P^{n}.
\]
$P^{n}$ is called the $n$-step transition matrix. In this matrix, the $a_{ij}$ element gives the
probability of going from state $i$ to state $j$.

\begin{quote}
\textit{\textbf{Example}}\\
Consider the movement of people from the city to the suburb.
\begin{center}
\begin{tabular}{c | c c}
& Cities & Suburbs\\
\hline
Cities & 0.96 & 0.04\\
Suburbs & 0.1 & 0.99
\end{tabular}
\end{center}
with the initial vector
\[
\bf{x}_{0}=[60,125].
\]
We can introduce some ``noise" into the system by discussing the eigenvalues of this system. In the
usual case, $\lambda=1$. We can show this by solving the system.
\begin{align*}
(\bf{P}-\lambda)\bf{x}&\to
\left[
\begin{array}{c c c}
-0.4 & 0.4 & 0\\
0.1 & 0.1 & 0
\end{array}
\right]\\
&\to
\left[
\begin{array}{c c c}
1 & -1 & 0\\
0 & 0  & 0
\end{array}
\right],
\end{align*}
so $x=y=1$.
\end{quote}

Often in Markov processes, we come across matrices raised to a power. To help us calculate these
products, we can come up with a new matrix, $\bf{S}$. We can then calculate
$\bf{S}^{-1}\bf{D}\bf{S}$, where $D$ is diagonalized matrix of the eigenvalues of $\bf{P}$. We know
that we can diagonalize $\bf{P}$ by the following theorem:
\begin{quote}
An $n\times n$ matrix $\bf{P}$ is diagonalizable if and only if it has $n$ linearly independent
eigenvectors.\\\\

\textit{\textbf{Example}}\\
Consider three types of detergent that yield the Markov matrix
\begin{center}
\begin{tabular}{c | c c c}
& Tide & Bounty & Bounce\\
\hline
Tide & .8 & .1 & .1\\
Bounty & .2 & .6 & .2\\
Bounce & .3 & .3 & .4
\end{tabular}
\end{center}
where the entries indicate the probability that a user will stay with that product. Rather than
finding the next state, we will consider a vector $\pmb{\pi}$, called the limiting state probability
vector. This vector will be representative of the probabilities of users staying with a brand.
Similar to the previous examples, we have
\[\pmb{\pi}\bf{P}=\pmb{\pi}.\]
From this equation, we get
\[
\begin{cases}
.8\pi_{1}+.2\pi_{2}+.3\pi_{3}=\pi_{1}\\
.1\pi_{1}+.6\pi_{2}+.3\pi_{3}=\pi_{2}\\
.1\pi_{1}+.2\pi_{2}+.4\pi_{3}=\pi_{3}
\end{cases}
\]
The third equation in this case is a linear combination of the first two, so we cannot user it here.
However, we can replace it with a linearly independent row:
\[
\begin{cases}
.8\pi_{1}+.2\pi_{2}+.3\pi_{3}=\pi_{1}\\
.1\pi_{1}+.6\pi_{2}+.3\pi_{3}=\pi_{2}\\
\pi_{1}+\pi_{2}+\pi_{3}=1
\end{cases}
\to
\left[
\begin{array}{c c c | c}
-.2 & .2 & .3 & 0\\
.1 & -.4 & .3 & 0\\
1 & 1 & 1 & 0
\end{array}
\right]
\to
\pmb{\pi}=
\left[
\begin{array}{c}
{6/11}\\
{3/11}\\
{2/11}
\end{array}
\right].
\]
\end{quote}


\end{document}
