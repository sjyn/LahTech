\section{Liquids}
Fluids can be modeled using several different techniques. The classical way involves using
partial differential equations. However, they can also be modeled with an agent based model,
known as DSMC (Direct Simulation Monte-Carlo).
\subsection{Probability}
\textit{\textbf} Find the probability of getting a 5 in 20 rolls of a die.
\begin{quote}
Our probability of not getting a 5 is $5/6$, and we only want one 5, which mean that the chance of not
getting a 5 is $(5/6)^{19}$. The chance of getting a 5 is $1/6$, and we make 20 rolls, so we have
\[
P = \left({5\over 6}\right)^{19}\cdot20\cdot\left({1\over6}\right)
\]
\end{quote}
This problem is easy to simulate on a TI-84 calculator:\\
\begin{quote}
\begin{verbatim}
randInt(1,6,60)->L1
\end{verbatim}
\end{quote}
\textit{\textbf} Given an answer key of A B B C D D A C B A, model how a student who has not studied
at all will do.
\begin{quote}
    
\end{quote}
