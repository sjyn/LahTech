\documentclass{hw}

\usepackage{minted}

\begin{document}
\section{Modeling Free Fall}
\noindent\\

\begin{enumerate}
\item Measure the height of the object above ground
\item Record time in seconds that it takes for object to hit the ground
\item Determine the quantitative relationship between the height and the time (a mathematical
model)
\end{enumerate}
Galileo makes some assumptions before experimenting:
\begin{enumerate}
\item $F = ma$: we already know the mathematical expression for force
\item Gravity is the only force acting on the object
\item Acceleration due to gravity is -32${m\over s^{2}}$.
\begin{gather*}
mg = my''\\
g = y''\\
\iint g = \iint y''\\
-16t^{2} = y + Ct + D\\
y = -16t^{2} + v_{0}t + y_{0}
\end{gather*}
\end{enumerate}
Galileo assumed that gravity was the only force, but what if air resistance is added? We know that
air resistance acts opposite to the velocity. We now have
\begin{gather*}
\kappa v - 32m = mv'\\
v' = 32 - \kappa_{0} v
\end{gather*}

\section{Discreet Modeling}
\noindent\\

We can now consider a discreet case for modeling.
\subsection{Credit Card Balancing}
We can work with the equation
\[
B_{1} = B_{0} - P
\]
as a simple model for a credit card balance. We will eventually find that this model is very
simplistic, and we will need to expand on it later. We have a recurrence relation where
\begin{gather*}
B_{2} = B_{1} - P\\
B_{3} = B_{2} - P\\
\cdots
\end{gather*}
Substituting into the recurrence relation tells us that
\[
B_{n} = B_{0} - nP.
\]
We can reform the model to help use come up with a better representation of the situation.
In the real world, credit cards often come with interest rates. Taking this into account
gives us
\[
B_{\text{new}} = (1+r)B_{\text{old}} - P.
\]
We again have a recurrence relation. We will let $s = (1+r)$:
\begin{align*}
B_{1} &= sB_{0} - P\\
B_{2} &= sB_{1} - P = s^{2}B_{0}-p(1+s)\\
B_{3} &= sB_{2} - P = s^{3}B_{0}-p(1+s+s^{2})\\
B_{4} &= sB_{3} - P = s^{4}B_{0}-p(1+s+s^{2}+s^{3})\\
\cdots
\end{align*}
We can derive a general solution of the form
\begin{align*}
B_{n} &= s^{n}B_{0}-p(1 + s + s^{2} + s^{3} + \cdots + s^{n-1})\\
&= (1+r)^{n}B_{0}-p\left({(1+r)^{n} - 1\over r}\right)
\end{align*}
\subsubsection{Solving With Excel}
We can use Excel to represent the model. We will consider the case where $r=1.5\%$ with an initial
balance of $\$1000$, and $\$10$ per month on payments. In the A1 cell, we want to have
\begin{minted}{text}
=(1.015)^$A1 * 1000 - 10*((1.015)^$A1-1)/0.015
\end{minted}


\section{Modeling Ordinary Differential Equation}
Consider the following system of ODE's
\[
\left[
\begin{array}{c}
Q_{1}^\prime\\
Q_{2}^\prime\\
Q_{3}^\prime
\end{array}
\right]
\]
We will use arbitrary values for the coefficient matrix. We want to solve the system using
eigenvalues:
\[
\left[
\begin{array}{c c c}
5-\lambda & 4 & 2\\
4 & 5-\lambda & 2\\
2 & 2 & 2-\lambda
\end{array}
\right]
= 0
\]
We want to take the determinant of the matrix, but first we can do some row reductions:
\[
\left[
\begin{array}{c c c}
1-\lambda & 0 & 0\\
4 & 9-\lambda & 2\\
2 & 4 & 2-\lambda
\end{array}
\right]
\]
Taking that determinant gives us
\[
\lambda = 1 \qquad\qquad\text{and}\qquad\qquad \lambda = 10.
\]

\subsection{Modeling War}
We will model the Nazi party and the Soviet Union military forces. We want to model the rate
at which Soviet Tanks reduce as Nazi anti-tank guns decrease.
\begin{gather*}
\frac{dx}{dt} = -ay\\
\frac{dy}{dt} = -bx
\end{gather*}
In this system, $x'$ represents the rate at which the Soviet tanks decreased, and $y'$ represents
the rate at which German anti-tank guns decrease. We will consider the battle of Kursk, during
which the number of Soviet tanks decreased by $50\%$ in the first hour. In this model, $a$ is
referred to as the anti-tank kill rate, where $b$ is the tank kill rate. We can use separation
of variables yields
\begin{gather*}
{ay^2\over2}+{ay_{0}^{2}\over2} = {bx^2\over2}+{bx_{0}^{2}\over2}\\
ay^{2}-bx^{2}=C
\end{gather*}
\end{document}
