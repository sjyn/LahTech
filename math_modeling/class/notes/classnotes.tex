\documentclass{hw}

\usepackage{minted}

\begin{document}
\section{Modeling Free Fall}
\noindent\\

\begin{enumerate}
\item Measure the height of the object above ground
\item Record time in seconds that it takes for object to hit the ground
\item Determine the quantitative relationship between the height and the time (a mathematical
model)
\end{enumerate}
Galileo makes some assumptions before experimenting:
\begin{enumerate}
\item $F = ma$: we already know the mathematical expression for force
\item Gravity is the only force acting on the object
\item Acceleration due to gravity is -32${m\over s^{2}}$.
\begin{gather*}
mg = my''\\
g = y''\\
\iint g = \iint y''\\
-16t^{2} = y + Ct + D\\
y = -16t^{2} + v_{0}t + y_{0}
\end{gather*}
\end{enumerate}
Galileo assumed that gravity was the only force, but what if air resistance is added? We know that
air resistance acts opposite to the velocity. We now have
\begin{gather*}
\kappa v - 32m = mv'\\
v' = 32 - \kappa_{0} v
\end{gather*}

\section{Discreet Modeling}
\noindent\\

We can now consider a discreet case for modeling.
\subsection{Credit Card Balancing}
We can work with the equation
\[
B_{1} = B_{0} - P
\]
as a simple model for a credit card balance. We will eventually find that this model is very
simplistic, and we will need to expand on it later. We have a recurrence relation where
\begin{gather*}
B_{2} = B_{1} - P\\
B_{3} = B_{2} - P\\
\cdots
\end{gather*}
Substituting into the recurrence relation tells us that
\[
B_{n} = B_{0} - nP.
\]
We can reform the model to help use come up with a better representation of the situation.
In the real world, credit cards often come with interest rates. Taking this into account
gives us
\[
B_{\text{new}} = (1+r)B_{\text{old}} - P.
\]
We again have a recurrence relation. We will let $s = (1+r)$:
\begin{align*}
B_{1} &= sB_{0} - P\\
B_{2} &= sB_{1} - P = s^{2}B_{0}-p(1+s)\\
B_{3} &= sB_{2} - P = s^{3}B_{0}-p(1+s+s^{2})\\
B_{4} &= sB_{3} - P = s^{4}B_{0}-p(1+s+s^{2}+s^{3})\\
\cdots
\end{align*}
We can derive a general solution of the form
\begin{align*}
B_{n} &= s^{n}B_{0}-p(1 + s + s^{2} + s^{3} + \cdots + s^{n-1})\\
&= (1+r)^{n}B_{0}-p\left({(1+r)^{n} - 1\over r}\right)
\end{align*}
\subsubsection{Solving With Excel}
We can use Excel to represent the model. We will consider the case where $r=1.5\%$ with an initial
balance of $\$1000$, and $\$10$ per month on payments. In the A1 cell, we want to have
\begin{minted}{text}
=(1.015)^$A1 * 1000 - 10*((1.015)^$A1-1)/0.015
\end{minted}


\section{Modeling Ordinary Differential Equation}
Consider the following system of ODE's
\[
\left[
\begin{array}{c}
Q_{1}^\prime\\
Q_{2}^\prime\\
Q_{3}^\prime
\end{array}
\right]
\]
We will use arbitrary values for the coefficient matrix. We want to solve the system using
eigenvalues:
\[
\left[
\begin{array}{c c c}
5-\lambda & 4 & 2\\
4 & 5-\lambda & 2\\
2 & 2 & 2-\lambda
\end{array}
\right]
= 0
\]
We want to take the determinant of the matrix, but first we can do some row reductions:
\[
\left[
\begin{array}{c c c}
1-\lambda & 0 & 0\\
4 & 9-\lambda & 2\\
2 & 4 & 2-\lambda
\end{array}
\right]
\]
Taking that determinant gives us
\[
\lambda = 1 \qquad\qquad\text{and}\qquad\qquad \lambda = 10.
\]

\subsection{Modeling War}
We will model the Nazi party and the Soviet Union military forces. We want to model the rate
at which Soviet Tanks reduce as Nazi anti-tank guns decrease.
\begin{gather*}
\frac{dx}{dt} = -ay\\
\frac{dy}{dt} = -bx
\end{gather*}
In this system, $x'$ represents the rate at which the Soviet tanks decreased, and $y'$ represents
the rate at which German anti-tank guns decrease. We will consider the battle of Kursk, during
which the number of Soviet tanks decreased by $50\%$ in the first hour. In this model, $a$ is
referred to as the anti-tank kill rate, where $b$ is the tank kill rate. We can use separation
of variables yields
\begin{gather*}
{ay^2\over2}+{ay_{0}^{2}\over2} = {bx^2\over2}+{bx_{0}^{2}\over2}\\
ay^{2}-bx^{2}=C
\end{gather*}

\section{Agents}
A Schelling Model is used to model things like segregation. Consider the following example:
\begin{center}
\begin{tabular}{c c c c c c}
x1 & x2 &    &    &    &    \\
x3 & o1 &    & o2 &    &    \\
x4 & x5 & o3 & o4 & o5 &    \\
x6 & o6 &    &    & x7 & x8 \\
   & o7 & o8 & x9 & x10 & x11 \\
   &    & o9 & o10 & o11 & \\
\end{tabular}
\end{center}
We can consider the case where, in the first iteration, all the members of the grid who do not have
three directly adjacent neighbors leave the area. In addition, x's want to be with other x's, and
o's want to be with other o's.
\begin{center}
\begin{tabular}{c c c c c c}
   &    &    &    &    &    \\
x3 & x6 &    & o1 & o2 &    \\
x4 & x5 & o3 & o4 &    &    \\
   & o6 & x2 & x1 & x7 & x8 \\
o11& o7 & o8 & x9 & x10 & x11 \\
   &    & o5 & o9 & o10 & \\
\end{tabular}
\end{center}
We can see that as the system changes, there will be a mass exodus of both types. In the first table,
there was only 1 agent with no neighbors of the opposite type. By the second iteration, there are now 6
agents with no neighbors of the opposite type.

\section{The Game of Life}
The game of life is a mathematical model created by John Conway. The rules are as follows:
\begin{enumerate}
\item \underline{Birth Rule:} If at time $t$ a cell is dead (empty) and the cell has 3 live neighbors
in any direction, then at time $t+1$ the cell becomes alive.
\item \underline{Death Rule:} If at time $t$ a live cell has 0 and 1 neighbors it dies of isolation,
and if a live cell has 4 or more neighbors, it dies of overcrowding.
\item \underline{Survival Rule:} If at time $t$ a live cell has 2 or 3 live neighbors, then at time
$t+1$, the cell is still alive.
\end{enumerate}

\section{Tumor Growth}
Tumor growth can be modeled by the differential equation
\[
{dV\over dt} = aV^{2/3}-bV'.
\]
This equation arises from the logistic equation
\[
{dP\over dt} = P(a-bP),
\]
which has a solution of
\[
P={M\over 1+e^{-\log{be^c}-at}}.
\]
As $t\to\infty$, $P$ heads towards $M$, since the exponential decreases. The tumor equation can be solved
by separation of variables:
\begin{align*}
{dV\over aV^{2/3}-bV}&=dt\\
\int{dV\over aV^{2/3}-bV}&=\int dt\\
3\int{du\over a-bu}&=t+c.
\end{align*}
The previous result arises from the $u$-substitution where $u=V^{1/3}$. We can make another substitution,
where $w=a-bu$. Then
\begin{align*}
3\int{du\over a-bu}&=t+c\\
{-3\over b}\ln{|a-bu|}&=t+c\\
{-3\over b}\ln{|a-bV^{1/3}|}&=t+c.
\end{align*}
We now will specify the integration constant, $c$, in terms of the the volume of the tumor, $V$.
\begin{align*}
c &= t - {3\over b}\ln{|a-bV^{1/3}|}.
\end{align*}
The plot of this function can be expressed as
\[
t = {3\over b}\ln{|a-bV_{0}^{1/3}\over|a-bV^{1/3}},
\]
which simplifies to
\[
V = {e^{-bt}\left(-a+bV_{0}^{1/3}+ae^{bt/3}\right)^{3}\over b^{3}}
\]
This function's plot acts the same as the logistic equation.
\subsection{Gompertz Curve}
Gompertz Curve is an equation that arises from
\[
\begin{cases}
{dV\over dt}=r(t)V\\
{dr\over dt}=-Kr
\end{cases}.
\]
The model has the same curve that the logistic growth model has, but the equation differs.
\[
V=V_{0}^{r_{0}/k}e^{-(r_{0}/k)e^{-kt}}.
\]
The model can also be expressed as a single equation:
\[
\begin{cases}
{dV\over dt}=r(t)V\\
{dr\over dt}=-Kr
\end{cases}
\to
{dV\over dt} = aV-bV\ln{V},
\]
which has the solution
\[
V(t)=V_{0}e^{r_{0}/t}e^{-r_{0}/ke^{-kt}}
\]

\section{Voting}
{\centering
\begin{tabular}{c c c c c c}
22\% & 23\% & 15\% & 29\% & 7\% & 4\%\\
T&T&B&B&I&I\\
B&I&T&I&B&T\\
I&B&I&T&T&B
\end{tabular}
}
\noindent\\
Because of the plurality, T wins the election. If the I candidate is removed, then B wins the election.
Often in this model, we deal with non-transitive systems. Consider the following voter profile:\\
{\centering
\begin{tabular}{c c}
14 & $A>B>C$\\
12 & $C>A>B$\\
10 & $B>C>A$
\end{tabular}
}
\noindent\\
This profile has the following outcome:\\
{\centering
\begin{tabular}{l c c c}
First & A&C&B\\
Second & B&A&C\\
Third & C&B&A\\
& 14&12&10
\end{tabular}
}
\noindent\\
In this case, $A>B$, $B>C$, but $C>A$.

\subsection{Pairwise Comparison}
This is based on combinatorics; if we have 3 candidates, we have
\[
{3 \choose 2}
\]
possibilities.

\end{document}
