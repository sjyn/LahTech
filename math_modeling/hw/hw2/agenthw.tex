% 1, 2a, 2c, 2f, 2h, 2i and 2k
\documentclass{hw}

\begin{document}
\makeheader{}

\begin{enumerate}
\item One famous example of how different models must be used to solve different problems in the same
system is grocery store checkout queues. If you are a customer deciding which queue to enter, how would
you model the problem? What exact question would your model address? What entities and processes would
be in the model? Now, if instead you are a store manager deciding on how to operate the queues for the
next hour or so, what questions would you model address and what would it look like? Finally, if you
are a store designer and the question is how to design the checkout area so that 100 customers can
check out per hour with the fewest employees, what things would you model?
\begin{quote}
\underline{Customer:} The customer needs to take in to consideration how many queues are open,
how many people are currently in the queue, and how fast the queues are moving. They should also consider
how many items they have, since some lanes may be restricted to a certain amount of items. The question
that needs to be answered in this case is ``Which queue should I enter?"\\\\
\underline{Manager:} The manager also needs to consider how many people are in each queue, and how
quickly the queues are moving. They may also consider the number of people that are not in the queue
yet, since those people will eventually join the queue. Ultimately, the manager is trying to answer
``Are the queues moving slowly, and if so how can I fix them?".\\\\
\underline{Designer:} The designer needs to take in to account the number of expected patrons of the store
on a day to day, and maybe even hour to hour basis. They also need to consider the number of employees
that the store can afford, since that will affect the number of lanes that can be open at a given time.
\end{quote}

\item For the following questions, what should be in a model? what kinds of things should be represented,
what variables should those things have to represent their essential characteristics, and what processes
that change things should be in the model? Should the model be agent-based? If the question is not clear
enough to decide, then reformulate the question to produce one that is sufficiently clear.
\begin{enumerate}
\item How closely together should a farmer plant the trees in a fruit orchard?
\begin{quote}
This model should be agent based where the trees are the agent. If the farmer plants the trees too closely
together, then they will not be able to evenly absorb the nutrients. If they are too far apart, then the
farmer loses land that he could use to plant more trees.
\end{quote}
\item Should a new road have one, two, or three lanes in each direction?
\begin{quote}
This problem is not an agent-model. In this case, the amount of regular traffic in the area would need
to be taken in to account. There may also be the issue of cost for using more asphalt, as well as
the amount of space for the road.
\end{quote}
\item How many trees per year should a timber company harvest?
\begin{quote}
This is an agent-model where the trees are the agents. The timber company should consider the impact that
removing too many trees would have on the area; if they remove too many trees, then they will destroy the
environment around the area.
\end{quote}
\item To maximize profit, how many flights per day should Saxon Airlines schedule between Frankfurt and
Leipzig?
\begin{quote}
This is an agent-model where the flights are the agents. The airline company needs to consider how many
flights are taken between the two cities, and how much each flight costs them versus how much profit they
turn from each flight.
\end{quote}
\item To minimize system wide delays and risk of accidents, how many flights per day should the EAA allow
between Frankfurt and Leipzig?
\begin{quote}
This is not an agent-model. The EAA needs to consider the amount of risk associated with high numbers
of flights.
\end{quote}
\item Any other problems or questions from your studies, research, or experience in general, that might
be the basis of a model or agent-based model.
\begin{quote}
A common model that is not agent-based is modeling heat flow through a surface. In 3D space, this can be
modeled with a PDE that looks like
\[
u_{t} = c^{2}\nabla^{2}u.
\]
This formula assumes an infinite length pipe, but in the real world, this does not exist. If we cut off
the ends of the pipe, we are now dealing with a much more complex system. The heat no longer flows
only through the cylinder, but also through the top and bottoms of the pipe.
\end{quote}
\end{enumerate}
\end{enumerate}
\end{document}
