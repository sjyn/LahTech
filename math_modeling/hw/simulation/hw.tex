\documentclass{hw}

% \renewcommand{\arraystretch}{1.5}

\begin{document}
\begin{enumerate}
\item Simulate 36 rolls of a fair die. Give the relative frequency and the corresponding theoretical
probability of each of the outcomes and compare them.
\begin{quote}
With 36 rolls, the following numbers are produced:\\
\[\{2,6,3,3,4,5,2,1,1,2,1,1,5,5,4,3,5,4,1,1,4,3,1,1,3,3,2,6,3,1,2,5,5,2,4,4\}\]
\begin{center}
\begin{tabular}{c | c | c}
Dice Number & Frequency & Theoretical Probability\\
\hline
1 & 9 & ${1/4}$\\

2 & 6 & ${1/6}$\\

3 & 7 & ${7/36}$\\

4 & 6 & ${1/6}$\\

5 & 6 & ${1/6}$\\

6 & 2 & ${1/18}$\\
\end{tabular}
\end{center}
\end{quote}

\item Simulate 96 rolls of a pair of dice where the sum is observed. Give the relative frequency and the
corresponding theoretical probability of each of the outcomes. Make a table showing your results. Repeat
the experiment 6 times and consider the total as if 576 rolls were simulated.
\begin{quote}
Rolling two die 96 times produces the matrix
\[
\left[
\begin{array}{c c c c c c c c}
7 & 3 & 6 & 12 & 9 & 7 & 2 & 6\\
10 & 7 & 9 & 6 & 10 & 9 & 7 & 7\\
5 & 8 & 4 & 7 & 3 & 11 & 7 & 11\\
10 & 6 & 10 & 8 & 4 & 8 & 4 & 9\\
10 & 9 & 6 & 11 & 7 & 7 & 9 & 7\\
5 & 8 & 5 & 10 & 7 & 4 & 12 & 6\\
5 & 10 & 6 & 8 & 9 & 9 & 12 & 4\\
2 & 7 & 5 & 9 & 6 & 4 & 7 & 5\\
9 & 6 & 4 & 10 & 6 & 6 & 6 & 11\\
8 & 7 & 6 & 8 & 9 & 9 & 8 & 9\\
5 & 6 & 9 & 12 & 5 & 8 & 7 & 7\\
8 & 6 & 12 & 3 & 10 & 8 & 6 & 6
\end{array}
\right]
\implies
\begin{array}{c | c | c}
\text{Value} & \text{Frequency} & \text{Probability}\\
\hline
1 & 0 & 0\\

2 & 2 & {1/48}\\

3 & 3 & {1/32}\\

4 & 7 & {7/96}\\

5 & 8 & {1/12}\\

6 & 17 & {17/96}\\

7 & 16 & {1/6}\\

8 & 11 & {11/96}\\

9 & 14 & {7/48}\\

10 & 9 & {3/ 32}\\

11 & 4 & {1/ 24}\\

12 & 5 & {5/96}\\
\end{array}
\]
Running the experiment 6 times produces
\begin{center}
\begin{tabular}{c | c | c | c | c | c | c | c}
Value & Run 1 & Run 2 & Run 3 & Run 4 & Run 5 & Run 6 & Total\\
\hline
1 & 0 & 0 & 0 & 0 & 0 & 0 & 0\\
2 & 1.5 & 1 & 1 & 1 & 1 & 2.5 & 8\\
3 & 2.33 & 3 & 1.66 & 2.33 & 3.33 & 1.66 & 14.33\\
4 & 2.5 & 1.75 & 1.75 & 3 & 1.5 & 1.75 & 12.25\\
5 & 1.6 & 2.2 & 2.6 & 1.6 & 2 & 2.2 & 12.2\\
6 & 2.5 & 3.16 & 2.33 & 2 & 1 & 1.5 & 12.5\\
7 & 2 & 2.286 & 2.142 & 2 & 2.571 & 2.571 & 13.571\\
8 & 2.5 & 1.25 & 1.5 & 1.75 & 2 & 1.625 & 10.625\\
9 & 0.33 & 1.33 & 0.88 & 1.44 & 1.11 & 1.11 & 6.22\\
10 & 1 & 0.4 & 1.1 & 0.7 & 0.8 & 1 & 5\\
11 & 0.27 & 0.18 & 0.54 & 0.54 & 0.63 & 0.45 & 2.63\\
12 & 0.25 & 0.33 & 0.25 & 0.0833 & 0.25 & 0.25 & 1.4166\\
\end{tabular}
\end{center}
\end{quote}

\newpage
\item Simulate 10 free-throws for Kobe Bryant, whose free-throw average in 2008 was 81\%. How many
of the shots were successful.
\begin{quote}
We can let 1 represent a successful free-throw, and 2 represent a miss. Running one test yields
\begin{center}
\begin{tabular}{c | c}
Shot & Result\\
1 & 1\\
2 & 1\\
3 & 1\\
4 & 1\\
5 & 1\\
6 & 1\\
7 & 1\\
8 & 2\\
9 & 2\\
10 & 1
\end{tabular}
\end{center}
From the table, we can see that he made 8 shots and missed two.
\end{quote}

\item A baseball player is a 0.331 hitter. Simulate 10 at-bats for this player and tell how many hits he
gets.
\begin{quote}
We can let 0 represent a miss, and 1 represent a hit.
\begin{center}
\begin{tabular}{c | c}
Shot & Result\\
\hline
1 & 1\\
2 & 0\\
3 & 1\\
4 & 1\\
5 & 0\\
6 & 0\\
7 & 1\\
8 & 0\\
9 & 0\\
10 & 1
\end{tabular}
\end{center}
Based on this run, the batter hits 50\% of the time.
\end{quote}

\newpage
\item A student who has not studied for a 10 question multiple choice test with 4 choices among the answers
(a, b, c, d) for each question, decides to simulate such a test and answer the questions according to a
simulation in which each choice has the same probability. Assume the correct answers are
\[\text{(a, b, b, c, d, d, a, c, b, a).}\]
What is the student's score?
\begin{center}
\begin{tabular}{c | c | c}
Student & Answer & Correct\\
\hline
a & a & Yes\\
d & b & No\\
a & b & No\\
a & c & No\\
b & d & No\\
c & d & No\\
a & a & Yes\\
d & c & No\\
d & b & No\\
a & a & No\\
\hline
& & \textbf{Score:} 20\%
\end{tabular}
\end{center}

\item In sampling 4 balls at random from an urn containing 30 balls, \textit{without replacing} after each
draw, we consider the balls as numbered 1 to 30. In selecting random whole number from 1 to 30, we ignore
any number that has already been selected and continue the selection until we obtain a sample size of 4.
Assume there are 20 red balls and 10 green balls in the urn. Draw 10 samples of size 4 and tabulate the
number of red balls in each sample. Compare your results with the theoretical probability.
\begin{quote}

\end{quote}

\newpage
\item Students are queued up at the registrars office with the registration windows open at 8. There are
four open windows; students approach the first open window as they advance to the front of the queue.
Assume that 10\% of the students require 5 minutes of service time, 30\% require 7 minutes, 40\% require 10
minutes, and 20\% require 15 minutes. Simulate the service of the first 20 students in a random queue. Show
the schedule of service at the four windows (A, B, C, D), and determine how long it takes to process these
students and give the average time from 8 to leaving the service window.
\begin{quote}
One result of running the simulation yields
\[5,10,7,7,10,10,5,15,15,7,15,15,10,10,7,7,10,7,15,15\]
as the queue. Assuming that students prefer A to B to C to D, we have the following process
\begin{center}
\begin{tabular}{c | c | c | c}
A & B & C & D\\
\hline
5 & 10 & 7 & 7\\
10 & 15 & 10 & 5\\
7 & 15 & 15 & 15\\
15 & 7 & 10 & 10\\
7 & 15 & 7 & 10\\
\hline
44 & 62 & 49 & 47
\end{tabular}
\end{center}
The average time is 8.8 minutes for A, 12.4 for B, 9.8 for C, and 9.4 for D. The overall average time
is 10.1 minutes.
\end{quote}

\item Simulate the following bank queue, and give the average time needed to process each customer.
There are four bank tellers, and 40\% of customers need 3 minutes, 50\% need 5, and 10\% need 8. Assume
there are 20 customers in line.
\begin{quote}
One run of the simulation yields
\[3,5,3,3,5,5,3,5,5,3,5,5,5,5,3,3,5,3,5,8\]
as the queue. On possible outcome of the tellers is
\begin{center}
\begin{tabular}{c | c | c | c}
Teller 1 & Teller 2 & Teller 3 & Teller 4\\
\hline
3 & 5 & 3 & 3\\
5 & 3 & 5 & 5\\
5 & 3 & 5 & 5\\
5 & 5 & 3 & 3\\
8 & 5 & 3 & 5\\
\hline
26 & 21 & 19 & 21
\end{tabular}
\end{center}
The average time is 5.2 for Teller 1, 4.2 for Teller 2, 3.8 for Teller 3, and 4.2 for Teller 4. The
overall average is 4.35 minutes for each customer.
\end{quote}

\newpage
\item A gas station with four self-serve pumps has determined that 80\% of all customers completely fill
their gas tanks and the remaining 20\% fill their tank with a fixed dollar amount's worth of fuel. Suppose
that it takes an average of 5 minutes for a complete fill up and 3 minutes for a partial fill up, and that
for an hour customers arrive steadily. Simulate this process for 30 customers.
\begin{quote}
One outcome of the simulation is
\[5,5,5,5,5,5,5,3,3,5,3,3,5,5,5,5,5,5,3,3,5,5,3,3,5,5,5,5,5,3.\]
This yields the setup:
\begin{center}
\begin{tabular}{c|c|c|c}
Pump 1 & Pump 2 & Pump 3 & Pump 4\\
\hline
5 & 5 & 5 & 5\\
5 & 5 & 5 & 3\\
5 & 3 & 3 & 3\\
5 & 3 & 3 & 5\\
3 & 5 & 5 & 3\\
5 & 5 & 5 & 5\\
  & 5 & 3 &  \\
\hline
28 & 31 & 29 & 26
\end{tabular}
\end{center}
The overall average time is 4.4 minutes per customer.
\end{quote}

\item Simulate 108 rolls of three dice and show the frequency of each possible sum of the faces.
\begin{quote}
Here is the outcome of one such run
\[
\left[
\begin{array}{c c c c c c c c c}
12 & 6 & 12 & 7 & 10 & 13 & 13 & 5 & 13\\
8 & 10 & 10 & 14 & 11 & 5 & 8 & 14 & 9\\
12 & 5 & 10 & 6 & 14 & 5 & 6 & 11 & 16\\
14 & 13 & 12 & 7 & 9 & 11 & 13 & 12 & 7\\
15 & 8 & 11 & 13 & 8 & 10 & 7 & 14 & 13\\
12 & 8 & 12 & 10 & 7 & 15 & 13 & 12 & 13\\
15 & 12 & 11 & 9 & 7 & 8 & 11 & 4 & 14\\
7 & 12 & 14 & 10 & 16 & 6 & 12 & 12 & 7\\
14 & 9 & 14 & 3 & 14 & 14 & 8 & 10 & 10\\
8 & 9 & 12 & 5 & 9 & 8 & 13 & 9 & 9\\
5 & 18 & 13 & 14 & 9 & 11 & 7 & 12 & 15\\
11 & 16 & 14 & 16 & 8 & 6 & 12 & 10 & 10\\
\end{array}
\right]
\implies
\begin{array}{c | c}
\text{Dice Number} & \text{Frequency}\\
\hline
3 & 1\\
4 & 1\\
5 & 6\\
6 & 5\\
7 & 9\\
8 & 10\\
9 & 9\\
10 & 11\\
11 & 8\\
12 & 15\\
13 & 11\\
14 & 13\\
15 & 4\\
16 & 4\\
17 & 0\\
18 & 1\\
\end{array}
\]
\end{quote}
\end{enumerate}
\end{document}
