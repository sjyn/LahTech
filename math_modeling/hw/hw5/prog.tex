% page 176, 8,10,12,14,17,18
\documentclass{hw}

\begin{document}

\begin{enumerate}
\item Why does l'H\^opital's rule apply to $\lim_{x\to 0}{V^{x} - 1 \over x}$, and precisely why is this
limit equal to $\lim_{x\to0}{V^{x}\ln{V} - 0\over 1}$?
\begin{quote}
l'H\^opital's rule applies since
\[
\lim_{x\to 0}{V^{x} - 1 \over x} = {V^{0} - 1\over0} = {1-1\over0} = {0\over0}.
\]
Using l'H\^opital's rule yields
\[
\lim_{x\to0}{V^{x}\ln{V} - 0\over 1}.
\]
\end{quote}

\item Use the Gompertz equation in the form ${dV\over dt} = (a-b\ln{V})V$ to explain why
$lim_{t\to\infty}V(t) = e^{a/b}$.
\begin{quote}
We can solve ${dV\over dt} = (a-b\ln{V})V$ for $V(t)$. We get
\[
V(t) = e^{{a\over b}-\left({a\over b}-\ln{V_{0}}\right)e^{-bt}}.
\]
If we take the limit as $t\to\infty$, we get
\begin{align*}
\lim_{t\to\infty}V(t) &= e^{{a\over b}-\left({a\over b}-\ln{V_{0}}\right)(0)}\\
&= e^{{a\over b}}.
\end{align*}
\end{quote}

\item In solving the Gompertz differential equation, we assumed that
$\int{1\over u}\du = \ln{u}+C$ rather than the more formally correct answer
$\int{1\over u}\du = \ln{|u|}+C$. Were we safe in ignoring the absolute value signs?
\begin{quote}
It was okay to ignore the absolute value signs since $u$ will always be positive. When solving the Gompertz
equation, we had that $u=Ce^{-bt}$. The term $e^{-bt}$ is always positive, and our constant $C$ was
determined by $V_{0}$, which is always positive. Therefore, $u > 0$, and $|u| = u$.
\end{quote}

\item Show that the Gompertz curve has a single point of inflection at the time when
$\ln{V} = {a\over b} - 1$.
\begin{quote}
The Gompertz equation has the form
\[
{dV\over dt} = aV-bV\ln{V}.
\]
If we differentiate this again, we get
\[
{d^2V\over dt^2} = a-b-b\ln{V}.
\]
Setting this equal to 0 yields
\begin{align*}
0 &= a-b-b\ln{V}\\
a &= b+b\ln{V}\\
{a\over b} &= 1 + \ln{V}\\
\ln{V} &= {a\over b} - 1.
\end{align*}
\end{quote}

\item With the estimated parameter values for the Gompertz model of chicken growth, show that the predicted
long-range limit to the size is 4.476.
\begin{quote}
From the model, we know that $g(t) = 4.47e^{-3.41e^{-0.251t}}$. If we take the limit as $t\to\infty$, we
have
\[
\lim_{t\to\infty}g(t) = 4.47e^{-3.41e^{0}} = 4.47.
\]
\end{quote}

\item Carry out the details of fitting the logistic model to the chicken weight data to show that
$d = 3.155450907$ and $a = 0.4124054532$.
\begin{quote}

\end{quote}
\end{enumerate}

\end{document}
