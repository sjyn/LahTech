% page 610 3,4,5,8,9,10,12-18
\documentclass{hw}
\begin{document}
\makeheader{Chapter 19}

\begin{enumerate}
\setcounter{enumi}{4}
\item
For $30\%$: 72.9\\
For $50\%$: 83.4\\
For $67\%$: 97.0\\
Mean: $84.433\%$\\
Standard Deviation: $12.08318$

\setcounter{enumi}{7}
\item For small values of $\%$-similar wanted, only a few ticks happen before everyone is happy. When
$\%$-similar wanted is between $25\%$ and $26\%$, the $\%$-similar is around $60\%$.

\item In both cases, the model never reaches a state of equilibrium. However, when the value is $70\%$,
it tends to be more segregated. When the value is $80\%$, it tends towards chaos.

\setcounter{enumi}{13}
\item In this case, the Schelling model predicts that for low enough percent similar wanted, the
population
will settle out.

\item The Schelling model predicts that the neighborhood will constantly be changing unless the population
has a very low percent wanted.

\item The Schelling model predicts that the neighborhood will constantly be changing unless the population
has a very low percent wanted.

\setcounter{enumi}{17}
\item The model does not change very drastically by using the Von-Neumann neighborhoods.
\end{enumerate}
\end{document}
