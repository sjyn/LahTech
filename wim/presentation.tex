\documentclass{beamer}
\mode<presentation>
\usetheme{boxes}
\usepackage{amsmath}
\AtBeginSection[]{
\begin{frame}<beamer>
\frametitle{Summary}
\tableofcontents[currentsection]
\end{frame}
}
\title{The Count Distinct Problem}
\author{Steven Rosendahl}

\begin{document}
\begin{frame}
\titlepage
\end{frame}

\section{The Problem}
\begin{frame}{The Question}
\pause
\begin{itemize}
\item What is the problem?
\pause
\item Image we have a set called $\mathbb{V}$ that contains a billion elements of the same type.
\pause
\item How many \textit{unique} elements are in $\mathbb{V}$?
\end{itemize}
\end{frame}

\begin{frame}{Questions}
\begin{enumerate}
\pause
\item \underline{Pok\'emon Problem:} How many unique Pok\'emon will a player encounter in a given playthrough of all the games?
\pause
\item \underline{Facebook Problem:} How many unique application installs are made a day on Facebook?
\pause
\item \underline{Twitter Problem:} How many unique hashtags are made a day on Twitter?
\pause
\end{enumerate}
We will use $\mathbb{S}$ to represent the set of all the data, and $\mathbb{V}$ to represent the set of unique elements. 
\end{frame}

\section{The Hash Table}
\begin{frame}{Hashing}
\begin{itemize}
\pause
\item What is \textit{hashing}?
\pause
\item Applying a function $h(x)$ to every element in $\mathbb{S}$, and storing the result in $\mathbb{V}$.
\pause
\item Ideally, $h(x)$ is
\begin{enumerate}
\pause 
\item Onto (surjective)
\pause
\item One-to-one (injective)
\end{enumerate}
\pause
\item We can ignore the duplicate values in $\mathbb{V}$.
\end{itemize}
\end{frame}

\begin{frame}{Solving the Pok\'emon Problem}
\begin{itemize}
\pause
\item How can we solve the Pok\'emon Problem using a hash?
\pause
\item Create a hash function that turns a given Pok\'emon into a numerical value
\pause
\item Store the result in $\mathbb{V}$ if it is not already there.
\pause
\item The hash function:
\begin{enumerate}
\item Sum up the ASCII value of each character in a Pok\'emon's name. Call this $n$.
\pause
\item Add $n$ to the Pok\'emon's corresponding National Pok\'edex number. Call this $m$.
\pause
\item Find $m\ mod\ 721$.
\pause
\end{enumerate}
\end{itemize}
\end{frame}

\begin{frame}{Problems With The Hash Table}
\begin{itemize}
\pause
\item Memory Intensive
\begin{itemize}
\item Pok\'emon problem only dealt with a set $\mathbb{S}$ of size 6000
\pause
\item Twitter Problem deals with $\mathbb{S}$ of size 200,000,000.
\pause
\item Collisions and collision policies also add to the amount of memory required.
\end{itemize}
\end{itemize}
\end{frame}

\section{The HyperLogLog}
\begin{frame}{The Algorithm}
\begin{enumerate}
\pause
\item Create a 
\end{enumerate}
\end{frame}
\end{document}