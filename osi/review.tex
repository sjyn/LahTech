\documentclass{article}
\usepackage{amssymb,amsmath}
\usepackage{mathtools}
\usepackage[table]{xcolor}
\usepackage{array}
\usepackage{multicol}
\usepackage{hyperref}
\usepackage{arydshln}
\topmargin=-0.3in
\textheight=9.2in
\textwidth=168mm
\oddsidemargin=-0.2in
\evensidemargin=-0.2in

\newcommand{\generateHeader}{
\hline
Process Number & 1 & 2 & 3 & 4 & 5 & 6 & 7 & 8 & 9 & 10 & 11 & 12 & 13 & 14 & 15 & 16 & 17 & 18 & 19 & 20\\
\hline
}

\newcommand{\bc}{\cellcolor{blue!60}}
\newcommand{\gc}{\cellcolor{black!50}}
\newcommand{\wc}{\cellcolor{white!}}
\newcommand{\blackc}{\cellcolor{black!}}

\newcolumntype{C}[1]{>{\centering\let\newline\\\arraybackslash\hspace{0pt}}m{#1}}

\newcommand{\blackrow}{
\blackc & \blackc & \blackc & \blackc & \blackc & \blackc & \blackc & \blackc & \blackc & \blackc & \blackc & \blackc & \blackc & \blackc & \blackc & \blackc & \blackc & \blackc & \blackc & \blackc & \blackc\\
\hline
}

\newcommand{\Example}{\textit{\textbf{Example: }}}
\title{Exam Review}
\date{}
\begin{document}
\maketitle

\newcommand{\specialcell}[2][c]{%
  \begin{tabular}[#1]{@{}c@{}}#2\end{tabular}}


\section{Memory Partitioning}

\indent The following diagram shows a partition of memory.
If three separate requests for memory of 40M, 20M, and 10M are made, determine
the spots in memory where the requests will each be placed, and indicate the
beginning address of the request based on the following policies:
\begin{enumerate}
\item First Fit
\item Best Fit
\item Next Fit
\item Worst Fit
\end{enumerate}
The grey blocks represent already filled spaces in memory.

\vspace{0.5cm}
\begin{center}
\begin{tabular}{|C{1cm} C{1cm} C{1cm} C{1cm} C{1cm} C{1cm} C{1cm} C{1cm} C{1cm} C{1cm} C{1cm} C{1cm}|}
\hline
20 & 20 & 40 & 60 & 20 & 10 & 60 & 40 & 20 & 30 & 40 & 40\\
\hline
\gc & \wc & \gc & \wc & \gc & \wc & \gc & \wc & \gc & \wc & \gc & \wc\\
\gc & \wc & \gc & \wc & \gc & \wc & \gc & \wc & \gc & \wc & \gc & \wc\\
\hline
\blackc & \blackc & \blackc & \blackc & \blackc & \blackc & \blackc & \blackc & \blackc & \blackc & \blackc & \blackc\\
\gc & \wc & \gc & \wc & \gc & \wc & \gc & \wc & \gc & \wc & \gc & \wc\\
\gc & \wc & \gc & \wc & \gc & \wc & \gc & \wc & \gc & \wc & \gc & \wc\\
\hline
\blackc & \blackc & \blackc & \blackc & \blackc & \blackc & \blackc & \blackc & \blackc & \blackc & \blackc & \blackc\\
\gc & \wc & \gc & \wc & \gc & \wc & \gc & \wc & \gc & \wc & \gc & \wc\\
\gc & \wc & \gc & \wc & \gc & \wc & \gc & \wc & \gc & \wc & \gc & \wc\\
\hline
\blackc & \blackc & \blackc & \blackc & \blackc & \blackc & \blackc & \blackc & \blackc & \blackc & \blackc & \blackc\\
\gc & \wc & \gc & \wc & \gc & \wc & \gc & \wc & \gc & \wc & \gc & \wc\\
\gc & \wc & \gc & \wc & \gc & \wc & \gc & \wc & \gc & \wc & \gc & \wc\\
\hline
\blackc & \blackc & \blackc & \blackc & \blackc & \blackc & \blackc & \blackc & \blackc & \blackc & \blackc & \blackc\\
\gc & \wc & \gc & \wc & \gc & \wc & \gc & \wc & \gc & \wc & \gc & \wc\\
\gc & \wc & \gc & \wc & \gc & \wc & \gc & \wc & \gc & \wc & \gc & \wc\\
\hline
\end{tabular}
\end{center}
\newpage


\section{Memory Allocation}

\indent A 1Mb block of memory is allocated using the buddy method.
Show the resulting partitions in the following diagram if they recieve the
following memory requests:
\[
A = 70K\qquad\qquad B=35K\qquad\qquad C=80K\qquad\qquad D=60K
\]
The memory is requested/released as follows:
\begin{multicols}{2}
\begin{enumerate}
\item Request A
\item Request B
\item Request C
\item Return A
\item Request D
\item Return B
\item Return D
\item Return C
\end{enumerate}
\end{multicols}

\vspace{0.5cm}
\begin{center}
\begin{tabular}{|c c c c c c c c c c c c c c c c|c|}
\hline
\multicolumn{16}{|c|}{1024} & Initial\\
\hline
\multicolumn{2}{|c|}{A} & \multicolumn{2}{|c|}{128} & \multicolumn{4}{|c|}{256} & \multicolumn{8}{|c|}{512} & Request A\\
\hline
\multicolumn{2}{|c|}{A} & \multicolumn{1}{|c|}{B} & \multicolumn{1}{|c|}{64} & \multicolumn{4}{|c|}{256} & \multicolumn{8}{|c|}{512} & Request B\\
\hline
\multicolumn{2}{|c|}{A} & \multicolumn{1}{|c|}{B} & \multicolumn{1}{|c|}{64} & \multicolumn{2}{|c|}{C} & \multicolumn{2}{|c|}{128} & \multicolumn{8}{|c|}{512} & Request C\\
\hline
\multicolumn{2}{|c|}{128} & \multicolumn{1}{|c|}{B} & \multicolumn{1}{|c|}{64} & \multicolumn{2}{|c|}{C} & \multicolumn{2}{|c|}{128} & \multicolumn{8}{|c|}{512} & Return A\\
\hline
\multicolumn{1}{|c|}{D} & \multicolumn{1}{|c|}{64} & \multicolumn{1}{|c|}{B} & \multicolumn{1}{|c|}{64} & \multicolumn{2}{|c|}{C} & \multicolumn{2}{|c|}{128} & \multicolumn{8}{|c|}{512} & Request D\\
\hline
\multicolumn{1}{|c|}{D} & \multicolumn{1}{|c|}{64} & \multicolumn{2}{|c|}{128} & \multicolumn{2}{|c|}{C} & \multicolumn{2}{|c|}{128} & \multicolumn{8}{|c|}{512} & Return B\\
\hline
\multicolumn{4}{|c|}{256} & \multicolumn{2}{|c|}{C} & \multicolumn{2}{|c|}{128} & \multicolumn{8}{|c|}{512} & Return D\\
\hline
\multicolumn{16}{|c|}{1024} & Return C\\
\hline
\end{tabular}
\end{center}
\newpage



\section{Paging I}
Given the following table, determine which physical address, if any, the following virtual address would correspond.
\[
1103\qquad\qquad\qquad 2325\qquad\qquad\qquad 5544
\]

\vspace{0.5cm}
\begin{center}
\begin{tabular}{|c|c|c|c|c|}
\hline
VPN & Valid Bit & Reference Bit & Modify Bit & PFN\\
\hline
0 & 1 & 1 & 0 & 4\\
\hline
1 & 1 & 1 & 1 & 7\\
\hline
2 & 0 & 0 & 0 & -\\
\hline
3 & 1 & 0 & 0 & 2\\
\hline
4 & 0 & 0 & 0 & -\\
\hline
5 & 1 & 0 & 1 & 0\\
\hline
\end{tabular}
\end{center}
\begin{minipage}[t]{0.3\textwidth}
\begin{align*}
1103 &= (1 * 1024) + 79\\
&\implies\text{ PFN: 7}\\
\text{Address} &= (7 * 1024) + 79\\
&= 7247
\end{align*}
\end{minipage}
\begin{minipage}[t]{0.3\textwidth}
\begin{align*}
2325 &= (2 * 1024) + 277\\
&\implies\text{ PFN: -}\\
&\implies\text{PAGE FAULT}
\end{align*}
\end{minipage}
\begin{minipage}[t]{0.3\textwidth}
\begin{align*}
5544 &= (5 * 1024) + 424\\
&\implies\text{ PFN: 0}\\
\text{Address} &= (0 * 1024) + 424\\
&= 79
\end{align*}
\end{minipage}
\newpage



\section{Paging II}
Given the following table, assume there is a page fault at 4.
Determine which page will be replaced using the following policies:
\begin{enumerate}
\item FIFO
\item LRU
\item Clock
\item Optimal (Assume the remaning sequence is 4, 0, 0, 0, 2, 4, 2, 1, 0, 3, 2)
\end{enumerate}

\vspace{0.5cm}
\begin{center}
\begin{tabular}{|c|c|c|c|c|}
\hline
VPN & Time Loaded & Reference Time & R-Bit & M-Bit\\
\hline
2 & 60 & 161 & 0 & 1\\
\hline
1 & 130 & 160 & 1 & 0\\
\hline
0 & 26 & 162 & 1 & 0\\
\hline
3 & 20 & 163 & 1 & 1\\
\hline
\end{tabular}

\vspace{0.5cm}
\begin{tabular}{|l|c c c c c c c c c c|}
\hline
VPN & 3 & 0 & 2 & 1 & $\cdots$ & 1 & 2 & 0 & 3 & 4\\
\hline
VPN In Memory & \specialcell[t]{3} & \specialcell[t]{3\\ 0} & \specialcell[t]{3\\ 0\\ 2} & \specialcell[t]{3\\ 0\\ 2\\ 1} & \specialcell{$\cdots$}& \specialcell[t]{3\\ 0\\ 2\\ 1} & \specialcell[t]{3\\ 0\\ 2\\ 1} & \specialcell[t]{3\\ 0\\ 2\\ 1} & \specialcell[t]{3\\ 0\\ 2\\ 1} & \specialcell[t]{4\\ 0\\ 2\\ 1}\\
\hline
Time & 20 & 26 & 60 & 130 & $\cdots$ & 160 & 161 & 162 & 163 & 164\\
\hline
\multicolumn{11}{c}{\blackc}\\
\hline
VPN & 3 & 0 & 2 & 1 & $\cdots$ & 1 & 2 & 0 & 3 & 4\\
\hline
VPN In Memory & \specialcell[t]{3} & \specialcell[t]{3\\ 0} & \specialcell[t]{3\\ 0\\ 2} & \specialcell[t]{3\\ 0\\ 2\\ 1} & \specialcell{$\cdots$}& \specialcell[t]{3\\ 0\\ 2\\ 1} & \specialcell[t]{3\\ 0\\ 2\\ 1} & \specialcell[t]{3\\ 0\\ 2\\ 1} & \specialcell[t]{3\\ 0\\ 2\\ 1} & \specialcell[t]{3\\ 0\\ 2\\ 4}\\
\hline
Time & 20 & 26 & 60 & 130 & $\cdots$ & 160 & 161 & 162 & 163 & 164\\
\hline
\multicolumn{11}{c}{\blackc}\\
\hline
VPN & 3 & 0 & 2 & 1 & $\cdots$ & 1 & 2 & 0 & 3 & 4\\
\hline
VPN In Memory & \specialcell[t]{3} & \specialcell[t]{3\\ 0} & \specialcell[t]{3\\ 0\\ 2} & \specialcell[t]{3\\ 0\\ 2\\ 1} & \specialcell{$\cdots$}& \specialcell[t]{3\\ 0\\ 2\\ 1} & \specialcell[t]{3\\ 0\\ 2\\ 1} & \specialcell[t]{3\\ 0\\ 2\\ 1} & \specialcell[t]{3\\ 0\\ 2\\ 1} & \specialcell[t]{3\\ 0\\ 4*\\ 1*}\\
\hline
Time & 20 & 26 & 60 & 130 & $\cdots$ & 160 & 161 & 162 & 163 & 164\\
\hline
\multicolumn{11}{c}{\blackc}\\
\hline
VPN & 1 & 2 & 0 & 3 & 4 & \multicolumn{5}{c|}{0 0 0 2 4 2 1 0 3 2 $\dots$}\\
\hline
VPN In Memory & \specialcell[t]{3\\ 0\\ 2\\ 1} & \specialcell[t]{3\\ 0\\ 2\\ 1} & \specialcell[t]{3\\ 0\\ 2\\ 1} & \specialcell[t]{3\\ 0\\ 2\\ 1} & \specialcell[t]{4\\ 0\\ 2\\ 1} & \multicolumn{5}{c|}{}\\
\hline
Time & 160 & 161 & 162 & 163 & 164 & \multicolumn{5}{c|}{$\dots$}\\
\hline
\end{tabular}
\end{center}
\newpage




\section{Processor Scheduling}
Determine the processor scheduling times of the following policies.
\begin{enumerate}
\item First Come First Serve
\item Round Robin (q = 1)
\item Round Robin (q = 4)
\item Shortest Process Next
\item Shortest Remaining Time
\item Highest Response Ratio Next (Response Ratio $= \frac{\text{wait time} + \text{service time}}{\text{service time}}$)
\item Feedback (q = 1)
\item Feedback (q = $2^{i}$)
\end{enumerate}
\begin{center}
\begin{tabular}{|c|c|c|}
\hline
Process Name & Arrival Time & Service Time\\
\hline
1 & 0 & 3\\
\hline
2 & 1 & 5\\
\hline
3 & 3 & 2\\
\hline
4 & 9 & 5\\
\hline
5 & 12 & 5\\
\hline
\end{tabular}

\vspace{0.5cm}

\begin{tabular}{|c|c|c|c|c|c|c|c|c|c|c|c|c|c|c|c|c|c|c|c|c|}
\generateHeader
1    &\bc    &\bc    &\bc    &    &    &    &    &    &    &    &    &    &    &    &    &    &    &    &    &\\
\hline
2    &    &    &    &\bc    &\bc    &\bc    &\bc    &\bc    &    &    &    &    &    &    &    &    &    &    &    &\\
\hline
3     &     &     &    &    &    &    &    &    &\bc    &\bc    &    &    &    &    &    &    &    &    &    &\\
\hline
4    &    &    &    &    &    &    &    &    &    &    &\bc    &\bc    &\bc    &\bc    &\bc    &    &    &    &    &\\
\hline
5    &    &    &    &    &    &    &    &    &    &    &    &    &    &    &    &\bc    &\bc    &\bc    &\bc    &\bc\\
\hline
\blackrow
1    &\bc    &    &\bc    &    &    &\bc    &    &    &    &    &    &    &    &    &    &    &    &    &    &\\
\hline
2    &    &\bc    &    &\bc    &    &    &\bc    &    &\bc    &    &\bc    &    &    &    &    &    &    &    &    &\\
\hline
3    &    &    &    &    &\bc    &    &    &\bc    &    &    &    &    &    &    &    &    &    &    &    &\\
\hline
4    &    &    &    &    &    &    &    &    &    &\bc    &    &\bc    &    &\bc    &    &\bc    &    &\bc    &    &\\
\hline
5    &    &    &    &    &    &    &    &    &    &    &    &    &\bc    &    &\bc    &    &\bc    &    &\bc    &\bc\\
\hline
\blackrow
1    &\bc    &\bc    &\bc    &    &    &    &    &    &    &    &    &    &    &    &    &    &    &    &    &\\
\hline
2    &    &    &    &\bc    &\bc    &\bc    &\bc    &    &    &\bc    &    &    &    &    &    &    &    &    &    &\\
\hline
3    &    &    &    &    &    &    &    &\bc    &\bc    &    &    &    &    &    &    &    &    &    &    &\\
\hline
4    &    &    &    &    &    &    &    &    &    &    &\bc    &\bc    &\bc    &\bc    &    &    &    &    &\bc    &\\
\hline
5    &    &    &    &    &    &    &    &    &    &    &    &    &    &    &\bc    &\bc    &\bc    &\bc    &    &\bc\\
\hline
\blackrow
1    &\bc    &\bc    &\bc    &    &    &    &    &    &    &    &    &    &    &    &    &    &    &    &    &\\
\hline
2    &    &    &    &    &    &\bc    &\bc    &\bc    &\bc    &\bc    &    &    &    &    &    &    &    &    &    &\\
\hline
3    &    &    &    &\bc    &\bc    &    &    &    &    &    &    &    &    &    &    &    &    &    &    &\\
\hline
4    &    &    &    &    &    &    &    &    &    &    &\bc    &\bc    &\bc    &\bc    &\bc    &    &    &    &    &\\
\hline
5    &    &    &    &    &    &    &    &    &    &    &    &    &    &    &    &\bc    &\bc    &\bc    &\bc    &\bc\\
\hline
\blackrow
1    &\bc    &\bc    &\bc    &    &    &    &    &    &    &    &    &    &    &    &    &    &    &    &    &\\
\hline
2    &    &    &    &    &    &\bc    &\bc    &\bc    &\bc    &\bc    &    &    &    &    &    &    &    &    &    &\\
\hline
3    &    &    &    &\bc    &\bc    &    &    &    &    &    &    &    &    &    &    &    &    &    &    &\\
\hline
4    &    &    &    &    &    &    &    &    &    &    &\bc    &\bc    &\bc    &\bc    &\bc    &    &    &    &    &\\
\hline
5    &    &    &    &    &    &    &    &    &    &    &    &    &    &    &    &\bc    &\bc    &\bc    &\bc    &\bc\\
\hline
\end{tabular}
\newpage
\begin{tabular}{|c|c|c|c|c|c|c|c|c|c|c|c|c|c|c|c|c|c|c|c|c|}
\generateHeader
1    &\bc&\bc&\bc&    &    &    &    &    &    &    &    &    &    &    &    &    &    &    &    &\\
\hline
2    &    &    &    &\bc&\bc&\bc&\bc&\bc&    &    &    &    &    &    &    &    &    &    &    &\\
\hline
3    &    &    &    &    &    &    &    &    &\bc&\bc&    &    &    &    &    &    &    &    &    &\\
\hline
4    &    &    &    &    &    &    &    &    &    &    &\bc&\bc&\bc&\bc&\bc&    &    &    &    &\\
\hline
5    &    &    &    &    &    &    &    &    &    &    &    &    &    &    &    &\bc&\bc&\bc&\bc&\bc\\
\hline
\blackrow
1    &\bc    &    &\bc    &    &    &    &\bc    &    &    &    &    &    &    &    &    &    &    &    &    &\\
\hline
2    &    &\bc    &    &    &\bc    &    &    &\bc    &\bc    &    &    &    &    &    &    &    &    &\bc    &    &\\
\hline
3    &    &    &    &\bc    &    &\bc    &    &    &    &    &    &    &    &    &    &    &    &    &    &\\
\hline
4    &    &    &    &    &    &    &    &    &    &\bc    &\bc    &\bc    &    &    &    &\bc    &    &    &\bc    &\\
\hline
5    &    &    &    &    &    &    &    &    &    &    &    &    &\bc    &\bc    &\bc    &    &\bc    &    &    &\bc\\
\hline
\blackrow
1    &\bc    &    &\bc    &\bc    &    &    &    &    &    &    &    &    &    &    &    &    &    &    &    &\\
\hline
2    &    &\bc    &    &    &    &\bc    &\bc    &    &\bc    &\bc    &    &    &    &    &    &    &    &    &    &\\
\hline
3    &    &    &    &    &\bc    &    &    &\bc    &    &    &    &    &    &    &    &    &    &    &    &\\
\hline
4    &    &    &    &    &    &    &    &    &    &    &\bc    &\bc    &\bc    &    &\bc    &\bc    &    &    &    &\\
\hline
5    &    &    &    &    &    &    &    &    &    &    &    &    &    &\bc    &    &    &\bc    &\bc    &\bc    &\bc\\
\hline
\end{tabular}
\end{center}
\newpage


\section{Multiprocessor Scheduling I}
Determine the runtimes of the following processes based on the Fixed Priority
($A\to B \to C$), and Earliest Deadline policies.

\vspace{0.5cm}
\begin{center}
\begin{tabular}{|c|c|c|c|}
\hline
Process & Arrival Time & Execution Time & Ending Deadline\\
\hline
$A_{1}$ & 0 & 10 & 20\\
$A_{2}$ & 20 & 10 & 40\\
$A_{3}$ & 40 & 10 & 60\\
$A_{4}$ & 60 & 10 & 80\\
$A_{5}$ & 80 & 10 & 100\\
\hline
$B_{1}$ & 0 & 10 & 50\\
$B_{2}$ & 50 & 10 & 100\\
\hline
$C_{1}$ & 0 & 15 & 50\\
$C_{2}$ & 50 & 15 & 100\\
\hline
\end{tabular}

\vspace{0.5cm}
\begin{tabular}{c|c:c:c:c:c:c:c:c:c:c:c:c:c:c:c:c:c:c:c:c:}
&  & 10 & & 20 & & 30 & & 40 & & 50 & & 60 & & 70 & & 80 & & 90 & & 100\\
\hline
FP & $A_{1}$ & $A_{1}$ & $B_{1}$ & $B_{1}$ & $A_{2}$ & $A_{2}$ & $C_{1}$ & $C_{1}$ & $A_{3}$ & $A_{3}$ & $B_{2}$ & $B_{2}$ & $A_{4}$ & $A_{4}$ & $C_{2}$ & $C_{2}$ & $A_{5}$ & $A_{5}$ & $C_{2}$ & \\
\hline
ED & $A_{1}$ & $A_{1}$ & $B_{1}$ & $B_{1}$ & $A_{2}$ & $A_{2}$ & $C_{1}$ & $C_{1}$ & $C_{1}$ & $A_{3}$ & $A_{3}$ & $B_{2}$ & $B_{2}$ & $A_{4}$ & $A_{4}$ & $C_{2}$ & $C_{2}$ & $C_{2}$ & $A_{4}$ & $A_{4}$\\
\hline
\end{tabular}
\end{center}
\newpage



\section{Multiprocessor Scheduling II}
Display the result of the following periodic tasks under the Earliest Starting Deadline,
First Come First Serve, and Earliest Starting Deadline with Unenforced Idle Times policies.

\vspace{0.5cm}
\begin{center}
\begin{tabular}{|c|c|c|c|}
\hline
Process & Arrival Time & Execution Time & Starting Deadline\\
\hline
A & 10 & 20 & 100\\
B & 20 & 20 & 20\\
C & 40 & 20 & 60\\
D & 50 & 20 & 80\\
E & 60 & 20 & 70\\
\hline
\end{tabular}

\vspace{0.5cm}
\scalebox{0.8}{
\begin{tabular}{c|c:c:c:c:c:c:c:c:c:c:c:c:c:c:c:c:c:c:c:c:c:c:c:c:}
& 5 & 10 & 15 & 20 & 25 & 30 & 35 & 40 & 45 & 50 & 55 & 60 & 65 & 70 & 75 & 80 & 85 & 90 & 95 & 100 & 105 & 110 & 115 & 120\\
\hline
ESD & & A & A & A & A & & C & C & C & C & E & E & E & E & D & D & D & D & & & & & &\\
\hline
ESDU & & & & & B & B & B & B & C & C & C & C & E & E & E & E & D & D & D & D & A & A & A & A\\
\hline
FCFS & & & A & A & A & A & & C & C & C & C & D & D & D & D & & & & & & & & &\\
\hline
\end{tabular}
}
\end{center}
\newpage



\section{I/O Operations}
% \begin{enumerate}
(a) Perform a FIFO, SSTF, SCAN, and C-SCAN on the following sequence of disk track requests
in order to calculate the average seek length:
\begin{center}
27, 129, 110, 186, 147, 41, 10, 64, 120
\end{center}
Assume that the disk starts at address 100, and the head is moving in the direction
of decreasing track number.
% \end{enumerate}

\vspace{0.5cm}
\begin{minipage}[t]{0.25\textwidth}
FIFO:\\\\
\begin{tabular}{c|c}
& 100\\
\hline
27 & 73\\
129 & 102\\
110 & 19\\
186 & 76\\
147 & 39\\
41 & 106\\
10 & 31\\
64 & 54\\
120 & 56\\
\hline
Average: & 61.8
\end{tabular}
\end{minipage}
\begin{minipage}[t]{0.25\textwidth}
SSTF:\\\\
\begin{tabular}{c|c}
& 100\\
\hline
110 & 10\\
120 & 10\\
129 & 9\\
147 & 18\\
186 & 39\\
64 & 122\\
41 & 23\\
27 & 14\\
10 & 17\\
\hline
Average: & 29.1
\end{tabular}
\end{minipage}
\begin{minipage}[t]{0.25\textwidth}
SCAN:\\\\
\begin{tabular}{c|c}
& 100\\
\hline
64 & 36\\
41 & 23\\
27 & 14\\
10 & 17\\
110 & 100\\
120 & 10\\
129 & 9\\
147 & 18\\
186 & 39\\
\hline
Average: & 29.6
\end{tabular}
\end{minipage}
\begin{minipage}[t]{0.25\textwidth}
C-SCAN:\\\\
\begin{tabular}{c|c}
& 100\\
\hline
64 & 36\\
41 & 23\\
27 & 14\\
10 & 17\\
186 & 176\\
147 & 39\\
129 & 18\\
120 & 9\\
110 & 10\\
\hline
Average: & 38
\end{tabular}
\end{minipage}

\vspace{0.5cm}
\noindent (b) Perform a FIFO, SSTF, SCAN, and C-SCAN on the following sequence of disk track requests
in order to calculate the average seek length:
\begin{center}
55, 58, 39, 18, 90, 160, 150, 30, 184
\end{center}
Assume that the disk starts at address 100, and the head is moving in the direction
of increasing track number.

\vspace{0.5cm}
\begin{minipage}[t]{0.25\textwidth}
FIFO:\\\\
\begin{tabular}{c|c}
& 100\\
\hline
55 & 45\\
58 & 3\\
39 & 19\\
18 & 21\\
90 & 72\\
160 & 70\\
150 & 10\\
30 & 120\\
184 & 154\\
\hline
Average: & 58.1
\end{tabular}
\end{minipage}
\begin{minipage}[t]{0.25\textwidth}
SSTF:\\\\
\begin{tabular}{c|c}
& 100\\
\hline
90 & 10\\
58 & 32\\
55 & 3\\
39 & 16\\
30 & 9\\
18 & 12\\
150 & 132\\
160 & 10\\
184 & 24\\
\hline
Average: & 27.5
\end{tabular}
\end{minipage}
\begin{minipage}[t]{0.25\textwidth}
SCAN:\\\\
\begin{tabular}{c|c}
& 100\\
\hline
150 & 50\\
160 & 10\\
184 & 24\\
90 & 94\\
58 & 32\\
55 & 3\\
39 & 16\\
30 & 9\\
18 & 12\\
\hline
Average: & 27.7
\end{tabular}
\end{minipage}
\begin{minipage}[t]{0.25\textwidth}
C-SCAN:\\\\
\begin{tabular}{c|c}
& 100\\
\hline
150 & 50\\
160 & 10\\
184 & 24\\
18 & 166\\
30 & 12\\
39 & 9\\
55 & 16\\
58 & 3\\
90 & 32\\
\hline
Average: & 35.7
\end{tabular}
\end{minipage}
\newpage



\section{UNIX Inodes}
Consider a UNIX file system represented by inodes.
Assume that there are 12 direct block pointers and a singly, doubly, and triply
indirect pointer in each inode.
Further, assume that the system block size is and the disk sector size are both
8 Kb.
If the disk block pointer is 32 bits, with 8 bits to identify the physical disk
and 24 bits to identify the physical block, then
\begin{enumerate}
\item What is the max file size supported by the system?\\\\
Since there are 12 block pointers in the inode, and each block is 8 Kb, we have
12 * 8 Kb, or 96 Kb, directly addressable.
The indirect pointer points to an entire block of pointers.
Since each block is 8 Kb, and each pointer is 32 bits, there are 1024 pointers
in the block; each pointer points to an 8 Kb block.
This means the single pointers take up 1024 * 8 Kb, or 8192 Kb.
The double pointer points to a pointer that points to a block.
There are 1024 of those, so we have 1024 * 1024 * 8 Kb, or 8,388,608 Kb.
The triple pointer will yield 1024 * 1024 * 1024 * 3 Kb, or 8,589,934,592 Kb.
This yields a total size of 8,598,331,488 Kb.
\item What is the max file system partition supported by the system?\\\\
There are 24 bits assigned to idetify the physical block, so the max file system
partition supported is $2^{24} * 8$, or 134,217,728 Kb.
\item How many disk accesses are required to access the byte in position 13,423,956?\\\\
The location in question is located in the third pointer, so it would take 4 disk
accesses to access the location.
\end{enumerate}
\end{document}


%\blackrow
%1    &    &    &    &    &    &    &    &    &    &    &    &    &    &    &    &    &    &    &    &\\
%\hline
%2    &    &    &    &    &    &    &    &    &    &    &    &    &    &    &    &    &    &    &    &\\
%\hline
%3    &    &    &    &    &    &    &    &    &    &    &    &    &    &    &    &    &    &    &    &\\
%\hline
%4    &    &    &    &    &    &    &    &    &    &    &    &    &    &    &    &    &    &    &    &\\
%\hline
%5    &    &    &    &    &    &    &    &    &    &    &    &    &    &    &    &    &    &    &    &\\
%\hline
