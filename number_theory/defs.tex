\documentclass{hw}
\title{Theorems and Definitions}
\date{}
\author{}

\newcommand{\Define}[1]{\item\underline{#1}:}
\newcommand{\prm}{p}
\newcommand{\set}[1]{\mathcal{#1}}
\newcommand{\setA}{\mathcal{A}}
\newcommand{\setB}{\mathcal{B}}
\newcommand{\setU}{\set{U}}

\begin{document}
\maketitle

\section{Definitions}

\subsection{Sets and Logic}
\begin{enumerate}
\Define{Logical Operator} A logical operator is a symbol that acts on a logical statement. The operators act as follows
\begin{center}
\begin{tabular}{|c|c|c|c|c|c|}
\hline
& & Negation & Disjunction & Conjunction & Implication\\
\hline
P & Q & $\neg$ P & P $\wedge$ Q & P $\vee$ Q & P $\implies$ Q \\
\hline
1 & 1 & 0 & 1 & 1 & 1\\
1 & 0 & 0 & 0 & 1 & 0\\
0 & 1 & 1 & 0 & 1 & 1\\
0 & 0 & 1 & 0 & 0 & 1\\
\hline
\end{tabular}
\end{center}
A statement that is always true (i.e. P $\vee\ \neg$P) is called a Tautology. A statement that is always false (i.e. P $\wedge\ \neg$P) is called a contradiction.\\
An implication P $\implies$ Q is made of two parts: the hypothesis (P) and the conclusion (Q).\\
Additionally, statements can be quantified using the two quantifiers, the universal quantifier, for all ($\forall$) and the existential quantifier, there exists ($\exists$).
\Define{Set} A set is a collection of objects. The objects in a set are called elements.
\Define{Empty Set} The empty set, symbolized by $\emptyset$ is the set with no elements.
\Define{Subset} A set $\setA$ is a subset of a set $\setB$ (noted $\setA\subset\setB$) if for all $x\in\setA$, $x\in\setB$.
\Define{Power Set} The power set of a set $\setA$ (noted $\set{P}(\setA)$) is the set of all subsets of $\setA$.
\Define{The Universal Set} The universal set, $\setU$ is the set of which all sets are subsets.
\Define{Intersection} For sets $\setA$ and $\setB$, the intersection ($\setA\cap\setB$) is the set of elements in both $\setA$ and $\setB$.
\[\setA\cap\setB = \{x|x\in\setA\wedge x\in\setB\}.\]
To say that $\setA$ and $\setB$ have a trivial intersection means that $\setA\cap\setB=\emptyset$. This is equivalent to saying $\setA$ and $\setB$ are disjoint.
\Define{Union} For sets $\setA$ and $\setB$, the union ($\setA\cup\setB$) is the set of elements in either $\setA$ or $\setB$.
\[\setA\cup\setB = \{x|x\in\setA\vee x\in\setB\}.\]
\Define{Set Difference} The set difference of a set $\setA$ and a set $\setB$ ($\setA-\setB$) is the set of all elements in $\setA$ that are not in $\setB$.
\[\setA-\setB=\{x|x\in\setA\wedge x\notin\setB\}.\]
The complement of a set $\setA$ in regards to $\setU$ is $\setU-\setA$.
\Define{Cartesian Product} For sets $\setA$ and $\setB$, the cartesian product ($\setA\times\setB$) is defined to be the set of ordered pairs $(a,b)$ such that $a\in\setA$ and $b\in\setB$.
\[\setA\times\setB=\{(a,b)|a\in\setA\wedge b\in\setB\}.\]
\Define{Axiom} An axiom is a statement whose truth value is accepted without proof.
\Define{Theorem} A theorem is a mathematical statement whose truth value can be verified through proof.
\Define{Lemma} A Lemma is a mathematical result that is used to prove other results.
\Define{Corollary} A corollary is a mathematical result that follows from another result.
\end{enumerate}

\newpage
\subsection{Number Theory}
\begin{enumerate}
\Define{Division} To say that $a$ divides $b$ ($a\mid b$) implies that $\exists x\in\ZZ$ such that $b=ax$.
\Define{Relation} A relation $\mathcal{R}$ from $\setA$ to $\setB$ is a subset of $\setA\times\setB$.
$\setA$ is related to $\setB$ ($a\mathcal{R}b$) if $(a,b)\in\mathcal{R}$ for all $a\in\setA$ and
$b\in\setB$. The domain of $\mathcal{R}$ is the set $\{x\ |\ (x,y)\in\mathcal{R}\}$. The range of
$\mathcal{R}$ is the set $\{y\ |\ (x,y)\in\mathcal{R}\}$. If $\mathcal{R}$ has an inverse
$\mathcal{R}^{-1}$, then it is $\{(y,x)\ |\ (x,y)\in\mathcal{R}\}$. A relation between two elements
$a\in\setA$ and $b\in\setB$ is denoted by $a\sim b$.
\Define{Reflexive} A relation $\mathcal{R}$ is reflexive if $x\mathcal{R}x$ for all $x\in\setA$.
\Define{Symmetric} A relation $\mathcal{R}$ is symmetric if $x\mathcal{R}y$ and $y\mathcal{R}x$ for all
$x\in\setA$ and $y\in\setB$.
\Define{Transitive} A relation $\mathcal{R}$ is transitive if $x\mathcal{R}y$ and $y\mathcal{R}z$ implies
$x\mathcal{R}z$ for all $x\in\setA$, $y\in\setB$, and $z\in\set{C}$.
\Define{Equivalence Relation} An equivalence relation is a relation that is reflexive, symmetric, and
transitive.
\Define{Equivalence Class} For a non-empty set $\setA$ containing elements $a$ and $b$, the equivalence
class of $a$, noted $[a]$ is the set $\{b\ |\ b\sim a\}$.
\Define{Partition} A partition of a non-empty set $\setA$ is the set of subsets where
\begin{enumerate}
\item The union of all sets in the partition of $\setA$ is $\setA$
\item The intersection of any two different sets in the partition of $\setA$ is not equivalent to
$\emptyset$.
\end{enumerate}
\Define{Function} A function $f$ from $\setA\to\setB$ is a relation from $\setA$ to $\setB$ that satisfies
\begin{enumerate}
\item $(a,b)\wedge(a,c)\in f\implies b=c$.
\item $\forall a\in\setA, \exists b\in\setB$ such that $(a,b)\in f$.
\end{enumerate}
\Define{Image} The image of a function $f:\setA\to\setB$ is $\{f(a)\ |\ a\in\setA\}$.
\Define{Inverse Image} For a function $f:\setA\to\setB$ and sets $\setB$ and $\set{D}$ where
$\set{D}\subset\setB$, the inverse image of $\set{D}$, $f^{-1}(\set{D})$, is defined to be
$\{a\in\setA\ |\ f(a)\in\set{D}\}$.
\Define{Injection} A function $f$ is injective if $\forall (a,b)\in\set{X},\ f(a)=f(b)\implies a=b$.
\Define{Surjection} A function $f$ is surjective if $\forall y\in\set{Y},\ \exists x\in\set{X}$ such that
$f(x)=y$.
\Define{Bijection} A function $f$ is a bijection if it is an injection and a surjection.
\end{enumerate}

\newpage
\section{Theorems and Important Ideas}
\begin{enumerate}
\Define{Division Algorithm} Suppose $a,b\in\ZZ$ and assume $b>0$. Then $\exists q,r\in\ZZ$ such that
$a=qb+r$ and $0\leq r\leq b$. Moreover, $q$ and $r$ are the only integers satisfying this property.
The integer $q$ is commonly called the quotient and $r$ is commonly called the remainder.
\begin{quote}
\textit{\textbf{Proof}}: Let $\set{S}=\{a-nb\ |\ n\in\ZZ\}$. We know that
\[
\lim_{n\to-\infty}(a-nb)\to+\infty,
\]
so $\set{S}$ must contain at least one positive integer. Let $r$ be the smallest element of $\set{S}$
that is greater than zero, called the remainder. Then there is a $q\in\ZZ$ such that $r=a-qb\geq0$, or
$a=qb+r$ for $r\geq0$. We have shown that this $r$ exists.\\
Assume $r\geq b$. Then $r-b\geq0$, and we know $r=a-qb$, so $a-qb-b\geq 0$. Then $a-b(q+1)\in\set{S}$
is greater than zero but less than $r$, which is a contradiction since $r$ is the smallest positive
integer in $\set{S}$. Therefore $a=qr+b$ and $0\leq r\leq b$.\\
Assume $a=bq+r$ and $a=bg+h$ where $0\leq r\leq b$ and $0\leq h\leq g$.
\end{quote}
\end{enumerate}
\end{document}
