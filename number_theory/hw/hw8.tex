\documentclass{hw}

\begin{document}
\makeheader{8}

\begin{enumerate}
\item Determine whether each of the following polynomials has a zero in the given $\ZZ_{p}$. Either
find a zero or prove that no zero exists.
\begin{enumerate}
\item $f(x)=x^3+x^2+x+1$ in $\ZZ_{2}$.
\begin{quote}
We can factor $x^3+x^2+x+1$ into $(x^{2}+1)(x+1)$. The only two options for a root in $\ZZ_{2}$ are
1 and 0. If we plug in zero we get $(0+1)(0+1)\equiv1\mod2$, so 1 is not a zero of the polynomial.
If we try 1, we get $(1+1)(1+1)\equiv(2)(2)\equiv(0)(0)\equiv0\mod2$, so 1 is a zero of the
polynomial.
\end{quote}
\item $f(x)=x^3+x^2+x+1$ in $\ZZ_{3}$.
\begin{quote}
$f(x)$ factors into $(x^{2}+1)(x+1)\equiv(x^{2}-2)(x-2)\mod3$. We have one zero at $x=2$. If we test
0, we get $(0+1)(0+1)\equiv1\mod3$, so 0 is not a zero of the polynomial, and if we test 1 we get
$(-1)(-1)\equiv1\mod3$, so 1 is not a zero either.
\end{quote}
\item $f(x)=x^{2}+2x+3$ in $\ZZ_{5}$.
\begin{quote}
The polynomial $f(x)=x^{2}+2x+3$ cannot be factored in $\ZZ_{5}$. Therefore, it has no zeros since
we cannot express it in the form $g(x)(x-\alpha)$.
\end{quote}
\end{enumerate}

\item Let $f(x)=x^{2}+1$. Prove that $f$ has a zero in $\ZZ_{5}$, but not in $\ZZ_{7}$.
\begin{quote}
If we consider the polynomial in $\ZZ_{5}$, we have that
\[x^{2}+1\equiv x^{2}-4\equiv(x+2)(x-2)\equiv(x-3)(x-2)\mod5.\]
Since $f$ was reducible in $\ZZ_{5}$, we know it has at least one zero, and in this case it has
two zeros, $x=3$ and $x=5$. In $\ZZ_{7}$, we have
\[x^{2}+1\equiv x^{2}-6\mod7.\]
The polynomial $x^{2}-6$ is irreducible in $\ZZ_{7}$, so there are no roots.
\end{quote}

\item Suppose that $p$ is prime, $k$ is a non-zero element of $\ZZ_{p}$, and $f_{k}(x)=x^{2}-k$.
\begin{enumerate}
\item Prove that $s$ is a zero of $f_{k}$ if and only if $-s$ is also a zero of $f_{k}$.
\begin{quote}
$\Rightarrow)$ Suppose $s$ is a zero of $f_{k}$. Then $s^{2}-k\equiv0\mod p$. If $-s$ is not a zero,
then we have $(-s)^{2}-k\ \cancel{\equiv}\ 0\mod p$, so $s^{2}-k\ \cancel{\equiv}\ 0\mod p$, which
is a contradiction.\\\\
$\Leftarrow)$ Suppose $-s$ is a zero of $f_{k}$. Then $(-s)^{2}-k\equiv0\mod p$. Then
$s^{2}-k\equiv0\mod p$, so $s$ is a zero of $f_{k}$.
\end{quote}
\item Further assuming that $p>2$ prove that $f_{k}$ either has no zeros in $\ZZ_{p}$ or exactly
two zeros in $\ZZ_{p}$.
\begin{quote}
Suppose there was only one zero in $\ZZ_{p}$. We know this cannot be the case, since by $(a)$ we
showed that if $s$ is a zero, then $-s$ is also a zero. If we suppose there are more than 2 zeros,
then we can consider another zero, $\sigma$. By $(a)$, we know that $-\sigma$ is also a zero. Since
$deg(f)=2$, we know that there are at most 2 zeros in $\ZZ_{p}$. There cannot be just one zero, so
there must be only two zeros, or no zeros.
% We
% know that, without loss of generality, that if we consider the first root $\sigma$, then we can
% say
% \begin{align*}
% f_{k}(x)&=(x-\sigma)g(x)\\
% x^{2}-k&=(x-\sigma)g(x)\\
% {x^{2}-k\over x-\sigma}&=g(x).
% \end{align*}
% If we perform the division, we find that
% \[g(x)=x+\sigma+{\sigma^{2}-k\over x-\sigma}=x+\sigma,\]
% meaning that
\end{quote}
\end{enumerate}

\item Let $p$ be a prime with $p>2$ and assume that $a,b,c,r\in\ZZ_{p}$ with $a\nequiv0\mod p$.
Further, we define the polynomial $f(x)=ax^{2}+bx+c\in\ZZ_{p}[x]$.
\begin{enumerate}
\item Prove that $r$ is a zero of $f$ in $\ZZ_{p}$ if and only if
$(2ar+b)^{2}\equiv b^{2}-4ac\mod p$.
\begin{quote}
$\Rightarrow)$ Since $r$ is a zero of $f(x)$ and $f$ is a quadratic polynomial, we can create the
expression
\begin{align*}
ar^{2}+br+c&\equiv0\mod p\\
ar^{2}+br&\equiv-c\mod p.
\end{align*}
Consider the term $(2ar+b)^{2}$. Then
\begin{align*}
(2ar+b)^{2}&= 4a^{2}r^{2}+4bar+b^{2}\\
&=4a(ar^{2}+br)+b^{2}\\
&=4a(-c)+b^{2}\\
&=b^{2}-4ac\\
&\equiv b^{2}-4ac\mod p
\end{align*}
\noindent\\
$\Leftarrow)$ Suppose $(2ar+b)^{2}\equiv b^{2}-4ac\mod p$. Then
\begin{align*}
&4a^{2}r^{2}+4arb+b^{2}\equiv b^{2}-4ac\mod p\\
&4a^{2}r^{2}+4arb\equiv-4ac\mod p\\
&4a^{2}r^{2}+4arb+4ac\equiv0\mod p\\
&4a(ar^{2}+br+c)\equiv0\mod p.
\end{align*}
Since we are in $\ZZ_{p}$, we know all elements have an inverse. We can multiply both sides of the
congruence by $a^{-1}4^{-1}$, which gives us $ar^{2}+br+c\equiv0\mod p$. Therefore, $r$ is a root
of $f$.
\end{quote}
\item A point $y\in\ZZ_{p}$ is called a \textit{perfect square} if there exists $z\in\ZZ_{p}$
such that $z^{2}=y$.
\begin{enumerate}
\item If $f$ has at least one zero in $\ZZ_{p}$, prove that $b^{2}-4ac$ is a perfect square.
\begin{quote}
Suppose $r$ is a zero of $f$. Then by $(a)$ we have
\[
(2ar+b)^{2}\equiv b^{2}-4ac\mod p,
\]
which implies $b^{2}-4ac$ is a perfect square.
\end{quote}
\item If $b^{2}-4ac\equiv0\mod p$, prove that $f$ has a uniqe zero in $f$. Find a formula for that
zero in terms of $a$ and $b$.
\begin{quote}
Suppose $b^{2}-4ac\equiv0\mod p$. We know by $(a)$ that $r$ is root, and therefore
$(2ar+b)^2\equiv b^2-4ac \mod p$. Suppose $\rho$ is a root of $f$. then
$(2a\rho+b)^2\equiv b^2-4ac \mod p$, so $(2ar+b)^2\equiv(2a\rho+b)^2$. Since $b^2-4ac$ is a perfect
square, then $2ar+b\equiv2a\rho+b$. Then $2ar\equiv2a\rho$, and multiplying by $2^{-1}a^{-1}$ yields
$r\equiv\rho$. To find an expression for $r$, we know that $(2ar+b)^2=0$, so if we multiply by
$(2ar+b)^{-1}$, we have $2ar+b=0$. Then $r=-b2^{-1}a^{-2}$.
\end{quote}
\item If $b^{2}-4ac\nequiv0\mod p$ and $b^{2}-4ac$ is a perfect square, prove that $f$ has exactly
two distinct zeros in $\ZZ_{p}$.
\begin{quote}
We know that we have the equivalence $(2ar+b)^2\equiv b^2-4ac$. Let $x=(2ar+b)^2$ and
$k=b^2-4ac$. Then we can rearrange the congruence to say $x^2-k\equiv0\mod p$. We know
$x^2-k$ has either 0 or 2 zeros. Since $k$ is a perfect square, we can let $k=j^2$ for some
$j\in\ZZ_{p}$. Then $x^2-k=x^2-j^2=(x+j)(x-j)$, which has two distinct roots in $\ZZ_{p}$.
\end{quote}
\end{enumerate}
\end{enumerate}

\item Suppose that $g(x)=x^2+1\in\ZZ_{3}[x]$ and let $\phi$ be a zero of $g$.
\begin{enumerate}
\item Prove that $\phi\notin\ZZ_{3}$.
\begin{quote}

\end{quote}
\item Define the set $\FF_{9}=\{ 0,1,2,\phi,\phi+1,\phi+2,2\phi,2\phi+1,2\phi+2 \}$. Assuming that
the multiplication and addition in $\FF_{9}$ obeys the distributive law, prove that every element of
$\FF_{9}$ has a multiplicative inverse.
\begin{quote}

\end{quote}
\end{enumerate}

\item If $p>2$ is prime, prove that $\ZZ_{p}$ contains exactly $(p+1)/2$ perfect squares.
\begin{quote}

\end{quote}
\end{enumerate}
\end{document}
