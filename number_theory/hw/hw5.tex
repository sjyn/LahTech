\documentclass{hw}

\begin{document}
\begin{enumerate}
\item Use modular arithmetic to determine the day of the week 1000 days after Friday, March 11, 2016.
\begin{quote}
\[
1000 = (142)(7) + 6 \implies 1000\equiv6\text{ mod }7
\]
It will be 6 days after Friday, which is Thursday.
\end{quote}

\item Use modular arithmetic to determine the day of the week on March 11, 2036.
\begin{quote}
There are $(36-16)(365)+5=7305$ days between March 11, 2016 and March 11, 2036. We know
$7305\equiv4\text{ mod }7$, so it will be $4$ days after Friday, which is Tuesday.
\end{quote}

\item Can modular arithmetic be used to help us determine what month it is exactly 5000 days after Friday,
March 11, 2016?  If so, what month is it?  Otherwise, explain why modular arithmetic provides no help
in solving this problem.
\begin{quote}

\end{quote}

\item Without using a calculator, find the remainder when $8^{10}$ is divided by 11 and when $5^9$
is divided by 13.

\begin{minipage}[b]{0.5\textwidth}
\begin{align*}
10 &= 5 \cdot 2\\
8^{10} &= 8^{5\cdot2}\\
64^5&\equiv2^5\text{ mod }11\\
&\equiv32\text{ mod }11\\
&\equiv1\text{ mod }11
\end{align*}
The remainder is 1.
\end{minipage}
\begin{minipage}[b]{0.5\textwidth}
\begin{align*}
5^9 &= 5^3\cdot5^6\\
5^9 &\equiv 125\cdot5^6\text{ mod }13\\
&\equiv -8\cdot5^6\text{ mod }13\\
&\equiv 5\cdot5^6\text{ mod }13\\
&\equiv 5\cdot125\cdot125\text{ mod }13\\
&\equiv 125\text{ mod }13\\
&\equiv 5\text{ mod }13
\end{align*}
The remainder is 5.
\end{minipage}

\item If $a\in\ZZ$, prove that $a^3\not\equiv2\text{ mod }4$.
\begin{quote}
Since $a\in\ZZ$, we have that $a\equiv0\text{ mod }4$, $a\equiv1\text{ mod }4$, $a\equiv2\text{ mod }4$, or
$a\equiv3\text{ mod }4$.
If $a\equiv0\text{ mod }4$, then $a^3\equiv0^3\text{ mod }4$, which is equivalent to $0\text{ mod }4$.
If $a\equiv1\text{ mod }4$, then $a^3\equiv1^3\text{ mod }4$, which is equivalent to $1\text{ mod }4$.
If $a\equiv2\text{ mod }4$, then $a^3\equiv2^3\text{ mod }4,$ which is equivalent to
$8\text{ mod }4$,or $0\text{ mod }4$.
If $a\equiv3\text{ mod }4$, then $a^3\equiv3^3\text{ mod }4$, which is equivalent to
$27\text{ mod }4$, or $1\text{ mod }4$.
Therefore, it is impossible that $a^3\equiv2\text{ mod }4$.
\end{quote}
$\triangle$

\item Suppose that $m,n\in\ZZ$, with $m,n\geq2$ and write $l=lcm(m,n)$.
\begin{enumerate}
\item If $a$ and $b$ are integers such that $a\equiv b\text{ mod }l$, prove that
$a\equiv b\text{ mod }m$ and $a\equiv b\text{ mod }n$.
\begin{quote}
Suppose $a\equiv b\text{ mod }l$. Then $l\mid(a-b)$, or $a-b=lj,\ j\in\ZZ$. Since $l=lcm(m,n)$, we know that
both the prime factorization of $m$ and $n$ appear in the prime factorization of $l$. Suppose, without loss
of generality, that we factor out the prime factorization of
$m=p_{1}^{\alpha_{1}}p_{2}^{\alpha_{2}}\dots p_{k}^{\alpha_{k}},\ k\in\ZZ$ out of the prime factorization of
$l = p_{1}^{\gamma_{1}}p_{2}^{\gamma_{2}}\dots p_{q}^{\gamma_{q}},\ q\in\ZZ$. Then we are left with
$l=m\cdot{l\over m}$, where ${l\over m}\in\ZZ$. Then we have $a-b=m\cdot{l\over m}\cdot j$, which is $m$
times an integer $r$. Therefore $a-b=mr$, so $m\mid(a-b)$, and $a\equiv b\text{ mod }m$.
\end{quote}
$\triangle$

\item If $a$ and $b$ are integers such that $a\equiv b\text{ mod }m$ and $a\equiv b\text{ mod }n$, can
we conclude that $a\equiv b\text{ mod }l$? Either prove your answer or provide a counterexample.
\end{enumerate}

\item If $a,b,n\in\ZZ$, we adopt the notation that
\[
a+n\ZZ = \{a+nz\ :\ z\in\ZZ\}.
\]
If $(a+n\ZZ)\cap(b+n\ZZ)\neq\emptyset$, prove that $a+n\ZZ = b+n\ZZ$.
\end{enumerate}
\end{document}
