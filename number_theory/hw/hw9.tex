\documentclass{hw}

\begin{document}
\makeheader{9}

\begin{enumerate}
\item Directly calculate $\displaystyle\sum\limits_{d|12}\phi(d)$ and verify that you obtain 12 as your
answer.
\begin{align*}
\sum_{d|12}\phi(d) &= \phi(1) + \phi(2) + \phi(3) + \phi(4) + \phi(6) + \phi(12)\\
&= 1 + 1 + 2 + 2 + 2 + 4\\
&= 12
\end{align*}

\item Suppose that $p_{1},p_{2},\dots,p_{N}$ are distinct primes. Prove that
\[
{\phi(p_{1}p_{2}\dots p_{N}) \over p_{1}p_{2}\dots p_{N}}
=
\prod_{n=1}^{N}\left(1-{1\over p_{n}}\right)
\]
\begin{align*}
{\phi(p_{1}p_{2}\dots p_{N}) \over p_{1}p_{2}\dots p_{N}} &=
{\phi(p_{1})\phi(p_{2})\dots \phi(p_{N}) \over p_{1}p_{2}\dots p_{N}}\\
&= \prod_{n=1}^{N}{\phi(p_{n})\over p_{n}}\\
&= \prod_{n=1}^{N}{p_{n} -1 \over p_{n}}\\
&= \prod_{n=1}^{N}\left(1-{1\over p_{n}}\right)
\end{align*}

\item Find a value of $n$ such that $\phi(n)/n < 1/4$. What do you think is a good strategy for
choosing $n$ so that $\phi(n)/n$ is close to zero?
\begin{quote}
One value for which this holds true is $n=210$. $\phi(210)$ is 48, and $48/210 = 8/35 < 1/4$. One
strategy for finding these numbers would be to
\end{quote}

\item Suppose that $p$ is prime and $m$ and $n$ are non-negative integers.
\begin{enumerate}
\item Prove that $\phi(p^{m+n})\geq\phi(p^{m})\phi(p^{n})$.
\begin{quote}
We can consider $\phi(p^{m+n})$. If we let $m+n=j$, then we have $\phi(p^{j})$, which can be
expressed as $p^{j}-p^{j-1}$. If we consider $\phi(p^{m})\phi(p^{n})$, we have
\begin{align*}
\phi(p^{m})\phi(p^{n})&= (p^{m}-p^{m-1})(p^{n}-p^{n-1})\\
&= p^{m+n} - 2p^{m+n-1} + p^{m+n-2}\\
&= p^{j}-2p^{j-1}+p^{j-2}.
\end{align*}
If we compare the two values, we get
\begin{align*}
p^{j}-p^{j-1} &\stackrel{?}{\geq} p^{j}-2p^{j-1}+p^{j-2}\\
p^{j-1} &\geq p^{j-2}.
\end{align*}
We know this is true since $m,n>0$, so $j>0$.
\end{quote}

\item Under what additional assumptions on $m$ and $n$ do we obtain
$\phi(p^{m+n})=\phi(p^{m})\phi(p^{n})$?
\end{enumerate}


\end{enumerate}
\end{document}
