\documentclass{hw}

\begin{document}
\makeheader{9}

\begin{enumerate}
\item Directly calculate $\displaystyle\sum\limits_{d|12}\phi(d)$ and verify that you obtain 12 as your
answer.
\begin{align*}
\sum_{d|12}\phi(d) &= \phi(1) + \phi(2) + \phi(3) + \phi(4) + \phi(6) + \phi(12)\\
&= 1 + 1 + 2 + 2 + 2 + 4\\
&= 12
\end{align*}

\item Suppose that $p_{1},p_{2},\dots,p_{N}$ are distinct primes. Prove that
\[
{\phi(p_{1}p_{2}\dots p_{N}) \over p_{1}p_{2}\dots p_{N}}
=
\prod_{n=1}^{N}\left(1-{1\over p_{n}}\right)
\]
\begin{align*}
{\phi(p_{1}p_{2}\dots p_{N}) \over p_{1}p_{2}\dots p_{N}} &=
{\phi(p_{1})\phi(p_{2})\dots \phi(p_{N}) \over p_{1}p_{2}\dots p_{N}}\\
&= \prod_{n=1}^{N}{\phi(p_{n})\over p_{n}}\\
&= \prod_{n=1}^{N}{p_{n} -1 \over p_{n}}\\
&= \prod_{n=1}^{N}\left(1-{1\over p_{n}}\right)
\end{align*}

\item Find a value of $n$ such that $\phi(n)/n < 1/4$. What do you think is a good strategy for
choosing $n$ so that $\phi(n)/n$ is close to zero?
\begin{quote}
One value for which this holds true is $n=210$. $\phi(210)$ is 48, and $48/210 = 8/35 < 1/4$. One
strategy for finding these numbers would be to
\end{quote}

\item Suppose that $p$ is prime and $m$ and $n$ are non-negative integers.
\begin{enumerate}
\item Prove that $\phi(p^{m+n})\geq\phi(p^{m})\phi(p^{n})$.
\begin{quote}
We can consider $\phi(p^{m+n})$. If we let $m+n=j$, then we have $\phi(p^{j})$, which can be
expressed as $p^{j}-p^{j-1}$. If we consider $\phi(p^{m})\phi(p^{n})$, we have
\begin{align*}
\phi(p^{m})\phi(p^{n})&= (p^{m}-p^{m-1})(p^{n}-p^{n-1})\\
&= p^{m+n} - 2p^{m+n-1} + p^{m+n-2}\\
&= p^{j}-2p^{j-1}+p^{j-2}.
\end{align*}
If we compare the two values, we get
\begin{align*}
p^{j}-p^{j-1} &\stackrel{?}{\geq} p^{j}-2p^{j-1}+p^{j-2}\\
p^{j-1} &\geq p^{j-2}.
\end{align*}
We know this is true since $m,n>0$, so $j>0$.
\end{quote}

\item Under what additional assumptions on $m$ and $n$ do we obtain
$\phi(p^{m+n})=\phi(p^{m})\phi(p^{n})$?
\begin{quote}
If either $m$ or $n$, but not both $m$ and $n$ are zero, then we have
$\phi(p^{m+0})=\phi(p^{m})\phi(p^{0})=\phi(p^{m})$ or
$\phi(p^{0+n})=\phi(p^{0})\phi(p^{n})=\phi(p^{n})$.
\end{quote}
\end{enumerate}

\item Suppose that $a$ and $b$ are positive integers.
\begin{enumerate}
\item Prove that $\phi(ab)\geq\phi(a)\phi(b)$.
\begin{quote}
Suppose that $a$ and $b$ are not relatively prime and consider the product $\phi(a)\phi(b)$. Then
\begin{align*}
\phi(a)\phi(b) &<
\phi(p_{1}^{\alpha_{1}}p_{2}^{\alpha_{2}}\dots p_{r}^{\alpha_{r}})
\phi(p_{1}^{\beta_{1}}p_{2}^{\beta_{2}}\dots p_{r}^{\beta_{r}})\\
&= \phi(p_{1}^{\alpha_{1}})\phi(p_{2}^{\alpha_{2}})\dots \phi(p_{r}^{\alpha_{r}})
\phi(p_{1}^{\beta_{1}})\phi(p_{2}^{\beta_{2}})\phi(p_{r}^{\beta_{r}})\\
&= \prod_{i=1}^{r}\left(p_{i}^{\alpha_{i}}-p_{i}^{\alpha_{i}-1}\right)
\prod_{i=1}^{r}\left(p_{i}^{\beta_{i}}-p_{i}^{\beta_{i}-1}\right)\\
&= \prod_{i=1}^{r}p_{i}^{\alpha_{i}}\left(1-{1\over p_{i}}\right)
\prod_{i=1}^{r}p_{i}^{\beta_{i}}\left(1-{1\over p_{i}}\right)\\
&= ab\prod_{i=1}^{r}\left(1-{1\over p_{i}}\right)^{2}.
\end{align*}
We can also consider the function $\phi(ab)$.
\begin{align*}
\phi(ab) &= \phi(p_{1}^{\alpha_{1}}p_{2}^{\alpha_{2}}\dots p_{r}^{\alpha_{r}}
p_{1}^{\beta_{1}}p_{2}^{\beta_{2}}\dots p_{r}^{\beta_{r}})\\
&= \phi(p_{1}^{\alpha_{1}+\beta_{1}}p_{2}^{\alpha_{2}+\beta_{2}}\dots p_{r}^{\alpha_{r}+\beta_{r}})\\
&= \phi(p_{1}^{\alpha_{1}+\beta_{1}})\phi(p_{2}^{\alpha_{2}+\beta_{2}})\dots
\phi(p_{r}^{\alpha_{r}+\beta_{r}})\\
&= \prod_{i=1}^{r}p_{i}^{\alpha_{i}+\beta_{i}}-p_{i}^{\alpha_{i}+\beta_{i}}{1\over p}\\
&= \prod_{i=1}^{r}p_{i}^{\alpha_{i}+\beta_{i}}\left(1-{1\over p_{i}}\right)\\
&= ab\prod_{i=1}^{r}\left(1-{1\over p_{i}}\right).
\end{align*}
Since
\[
\phi(ab)=ab\prod_{i=1}^{r}\left(1-{1\over p_{i}}\right) >
ab\prod_{i=1}^{r}\left(1-{1\over p_{i}}\right)^{2}>\phi(a)\phi(b),
\]
we have
\[
\phi(ab) > \phi(a)\phi(b).
\]
If $gcd(a,b)=1$, then we can say $\phi(ab) = \phi(a)\phi(b)$. Therefore $\phi(ab)\geq\phi(a)\phi(b)$.
\end{quote}

\item Prove that $\phi(ab) = \phi(a)\phi(b)$ if and only if $gcd(a,b)=1$.
\begin{quote}
$\Rightarrow)$ Suppose $\phi(ab) = \phi(a)\phi(b)$ implies $gcd(a,b)>1$. We know from $(a)$ that
when $a$ and $b$ are not relatively prime, $\phi(ab) > \phi(a)\phi(b)$, which is a direct contradiction
to $\phi(ab) = \phi(a)\phi(b)$. Therefore $\phi(ab) = \phi(a)\phi(b)$ implies $gcd(a,b)=1$.
\noindent\\\\
$\Leftarrow)$ Suppose $gcd(a,b)=1$. Then $a$ and $b$ are relatively prime, so
$\phi(ab)=\phi(a)\phi(b)$ by the theorem.
\end{quote}
\end{enumerate}

\item Suppose that $n$ is a positive integer and $k$ is any integer.
\begin{enumerate}
\item Prove that $gcd(n,k)=1$ if and only if $gcd(n,n-k)=1$.
\begin{quote}
$\Leftarrow)$ Suppose that $gcd(n,k)>1$. Then $gcd(n,k)=q$, so $q\mid n$ and
$q\mid k$, $n=qa$ and $k=qb$ for some $a,b\in\ZZ$. Then $n-k=pa-pb=p(a-b)$, so $p\mid n-k$, which
implies that $gcd(n,n-k)>1$. Therefore, by contrapositive, $gcd(n,k)=1$ implies $gcd(n,n-k)=1$.
\noindent\\\\
$\Rightarrow)$ Suppose $gcd(n,n-k)>1$. Then there exists $p\in\ZZ$ such that $p\mid n$ and $p\mid k$.
Then $n=pa$ and $k=pb$ for some $a,b\in\ZZ$. We can express $n-k=pa-pb=p(a-b)$, so $p\mid(n-k)$ and
$p\mid n$, and $gcd(n,n-k)>1$.
\end{quote}
\item Prove that $\phi(n)$ is an even integer for all $n\geq3$.
\begin{quote}
By definition, the totient function counts the number of units in $\ZZ_{n}$. Suppose that
$n\geq3$. If we take any element $k$ from $\ZZ_{n}$, we know by $(a)$ that if $k$ is relatively
prime to $n$, then $n-k$ is relatively prime to $n$. Therefore, if $k$ is a unit in $\ZZ_{n}$, then
$n-k$ is also a unit in $\ZZ_{n}$. If $n$ is odd, then we know that there will not be any situation
where $n-k=k$, since that would imply that $n=2k$, or $n$ is even which is a contradiction. Then for
every unit $k$ in $\ZZ_{n}$, we can find another unit $n-k$ also in $\ZZ_{n}$, which means there are
an even number of units in $\ZZ_{n}$, so $\phi(n)\in\{ 2j\ |\ j\in\ZZ \}$ when $n$ is odd. If $n$
is even, however, there may be a unit $k$ in $\ZZ_{n}$ such that $k=n-k$. If this is the case,
then this unit will be the same as $n/2$. Since it is a unit, then $gcd(n/2,n)=1$. We saw that
$n$ was even, so we can say $n=2l,\ l\in\ZZ$. Then $gcd(2l/2,2l)=gcd(l,2l)\neq1$, so there cannot
be a unit $k$ in an even $\ZZ_{n}$ such that $k=n-k$.
\end{quote}
\end{enumerate}

\item If $n$ is a positive integer prove that $\phi(n)=2$ if and only if $n\in\{ 3,4,6\}$.
\begin{quote}
$\Rightarrow)$
\noindent\\\\
$\Leftarrow)$ Suppose $n\in\{ 3,4,6\}$. Then $\phi(3)=2$, $\phi(4)=2$, and $\phi(6)=2$.
\end{quote}
\end{enumerate}
\end{document}
