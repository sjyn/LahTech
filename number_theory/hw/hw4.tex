\documentclass{hw}

\begin{document}
\makeheader{4}

\begin{enumerate}
\item If $p$ is prime and $p\mid a^k$, prove that $p\mid a$. Conclude that $p^k\mid a^k$.
\begin{quote}
Since $p\mid a^k$, we have that $p\mid aa^{k-1}$. Then $p\mid a$ or $p\mid a^{k-1}$, but not
both. If we assume that $p\mid a^{k-1}$, we can say that $p\mid aa^{k-2}$. If $p\mid a^{k-2}$, then we have
that $p\mid aa^{k-3}$. If we repeat this process, we will ultimately have that $p\mid aa^0$. Then $p\mid a$
or $p\mid a^0$. If $p\mid a$, then we are done. If $p\mid a^0$, then $p\mid 1$, which
is a contradiction. Therefore, $p$ must divide $a$.\\\\
Assume $p\mid a$. Then $a=pm$ for some $m\in\ZZ$. Raising both sides to the $k$ power gives $a^k=p^km^k$.
Since $m^k\in\ZZ$, we can say that $m^k=j$. Then $a^k=p^kj$, so $p^k\mid a^k$.
\end{quote}
$\triangle$



\item Suppose that $a$ and $b$ are positive integers.
\begin{enumerate}
\item If $a^2\mid b^2$ prove that $a\mid b$. (Hint: Assuming that $a^2m = b^2$, use unique factorization to
prove that $m$ is a perfect square.)
\begin{quote}
Assume that $a^2\mid b^2$. Then $b^2=a^2m$ for some $m\in\ZZ$. By the Fundamental Theorem of Arithmetic,
we have that
$(p_{1}^{\alpha_1}p_{2}^{\alpha_2}\dots p_{j}^{\alpha_j})^2 =
(p_{1}^{\beta_1}p_{2}^{\beta_2}\dots p_{k}^{\beta_k})^2m$ for $j,k\in\ZZ$ and $\alpha,\beta\in\NN$.
Then we can say that
\[
{
(p_{1}^{\beta_1}p_{2}^{\beta_2}\dots p_{k}^{\beta_k})^2
\over
(p_{1}^{\alpha_1}p_{2}^{\alpha_2}\dots p_{j}^{\alpha_j})^2
}
= m\in\ZZ.
\]
Therefore, $m$ is a perfect square, so we can take its square root. Let $n=\sqrt{m}$. Then
$\sqrt{a^2m}=\sqrt{b^2}$, or $an=b$, so $a|b$.
\end{quote}

\item If $n$ is a positive integer such that $a^n \mid b^n$ can we conclude that $a \mid b$? Either prove
your answer or provide a counterexample.
\begin{quote}
By the same argument used above, we can say
\[
{b^n\over a^n} =
{
(p_{1}^{\beta_1}p_{2}^{\beta_2}\dots p_{k}^{\beta_k})^n
\over
(p_{1}^{\alpha_1}p_{2}^{\alpha_2}\dots p_{j}^{\alpha_j})^n
}
= m\in\ZZ.
\]
Therefore, $m$ is a perfect $n^{\text{th}}$ power, so we can say $j=\sqrt[n]{m}$. Then
$\sqrt[n]{b^n}=\sqrt[n]{a^nm}$ simplifies to $b=aj$. Therefore, $a|b$.
\end{quote}
$\triangle$
\end{enumerate}



\item Suppose that $a$ and $b$ are positive integers and $p_1, p_2,\dots, p_n$ are primes such that
\[
a = p_{1}^{\alpha_1}p_{2}^{\alpha_2}\dots p_{n}^{\alpha_n}
\qquad\text{and}\qquad
b = p_{1}^{\beta_1}p_{2}^{\beta_2}\dots p_{n}^{\beta_n},
\]
where $\alpha_{i}$ and $\beta_{i}$ are non-negative (possibly equal to 0) integers. Prove that
\[
lcm(a,b)=p_{1}^{max(\alpha_{1},\beta_{1})}p_{2}^{max(\alpha_{2},\beta_{2})}\dots
p_{n}^{max(\alpha_{n},\beta_{n})}.
\]
\begin{quote}
Let $L = lcm(a,b)$. We have by definition that $a\mid L$ and $b\mid L$. If we consider the case where
$a\mid L$, then we have that $L = am$ for some $m\in\ZZ$. We can arrange this as ${L\over a}$, which is
\[
{L\over p_{1}^{\alpha_1}p_{2}^{\alpha_2}\dots p_{n}^{\alpha_n}}
=
{
p_{1}^{\gamma_1}p_{2}^{\gamma_2}\dots p_{n}^{\gamma_n}
\over
p_{1}^{\alpha_1}p_{2}^{\alpha_2}\dots p_{n}^{\alpha_n}
}
\text{ by the FTA.}
\]
If we assume that $\gamma_{i} < \alpha_{i}$ for some $i\leq n$, then $L < a$, and ${L\over a}\notin\ZZ$,
which is a contradiction; we must have that $\gamma_{i} \geq \alpha_{i}$. Similarly, we can express
$b\mid L$ as
\[
{
p_{1}^{\gamma_1}p_{2}^{\gamma_2}\dots p_{n}^{\gamma_n}
\over
p_{1}^{\beta_1}p_{2}^{\beta_2}\dots p_{n}^{\beta_n}
}
= q\in\ZZ.
\]
By the same argument, we must have that we must have that $\gamma_{i} \geq \beta_{i}$. If we take
any arbitrary $\alpha_{i}$ such that $\alpha_{i} < \beta_{i}$, then letting
$\gamma_{i} = \alpha_{i}$ would cause ${L\over b}\notin\ZZ$; we also have that taking
$\gamma_{i} = \beta_{i}$ when $\beta_{i} < \alpha_{i}$ will cause ${L\over a}\notin\ZZ$. Therefore,
$\gamma_{i}$ must be the maximum of $\alpha_{i},\beta_{i}$ for all $i$. We can conclude that
the prime factorization of $L$ must be
$p_{1}^{max(\alpha_{1},\beta_{1})}p_{2}^{max(\alpha_{2},\beta_{2})}\dots
p_{n}^{max(\alpha_{n},\beta_{n})}$.
\end{quote}
$\triangle$



\item Determine whether each of the following statements is true or false. If true, prove it. If false,
provide a counterexample.
\begin{enumerate}
\item If $gcd(a, p^2) = p$ then $gcd(a^2, p^2) = p^2$.
\begin{quote}
We know that $p\mid a$ by the definition of the $gcd$. We can say that $a = pn$ for some $n\in\ZZ$.
Then $a^2=p^2n^2$, so $p^2\mid a^2$. Since $p^2$ is the greatest thing that divides $p^2$ and
$a > p$ so $a^2>p^2$, $gcd(a^2, p^2) = p^2$.
\end{quote}

\item If $gcd(a, p^2) = p$ and $gcd(b, p^2) = p^2$ then $gcd(ab, p^4) = p^3$.
\begin{quote}
If we take the case where $a=p$ and $b=p^3$, then we have $gcd(p, p^2) = p$ and $gcd(p^3, p^2) = p^2$.
However, $gcd(p^4, p^4) \neq p^3$.
\end{quote}

\item If $gcd(a,p^2)=p$ then $gcd(a+p,p^2)=p$.
\begin{quote}
If we take $a,p=2$, then we have that $gcd(2,4)=2$, but $gcd(4,4)\neq 2$.
\end{quote}
\end{enumerate}



\item Prove that every prime $p\neq 3$ has the form $3q + 1$ or $3q + 2$ for some integer $q$. Moreover,
prove that there are infinitely many primes of the form $3q + 2$.
\begin{quote}
If we consider a prime number $p\neq 3$, we will get a remainder of either 1 or 2 when we divide it by
three; if we get a remainder of 0, then that number was a multiple of 3 and therefore not prime. We
also know that all prime numbers above 3 are odd. Therefore, by the division algorithm, we have that
$p=3q+1$ or $p=3q+2$.\\\\
Assume that there are finitely many primes of the form $3q + 2$. Then we can say
\begin{align*}
p_1 &= 3q_1 + 2\\
p_2 &= 3q_2 + 2\\
\dots& \\
p_n &= 3q_n + 2\\
\end{align*}
We will let $m = 3p_1p_2\dots p_k-1$, which we can express as $3p_1p_2\dots p_k-3+2$. By factoring
we have that $m=3(p_1p_2\dots p_k-1)+2$. Then we have a prime $p = 3q+2$ that divides $m$. Since there
are finitely many primes of this form, we have that $p=p_i$ for some $i \leq n$. Without loss of
generality, we will let $i=1$. Then we have
\begin{align*}
1 &= 3(p_1p_2\dots p_n)-m\\
&= p_1(3p_2p_3\dots p_n - {m\over p_1})\\
&= p_1j\text{ for some } j\in\ZZ
\end{align*}
Therefore, $p_1\mid1$, which is a contradiction.
\end{quote}
$\triangle$



\item Two primes $p$ and $q$ with $p<q$ are called twin primes if $p+2=q$. If $p$ and $q$ are twin primes
with $3 < p < q$, prove that $6 \mid p + 1$.
\begin{quote}
We can take any three consecutive numbers starting with a prime $p$ where $p+2=q$ is also prime. We know
that any prime number greater than 3 is even, so it is divisible by two. We also know that if we take
three consecutive numbers, one of them has to be divisible by 3, since it will either have a remainder of
0, 1, or 2. Since $p$ is prime and $p+2$ is prime, then $p+1$ must be divisible by three. Therefore
$p+1$ must be divisible by 6, or $6|p+1$.
\end{quote}
\end{enumerate}
\end{document}
