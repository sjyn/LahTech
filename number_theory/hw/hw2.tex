\documentclass{hw}

\begin{document}
\makeheader{2}

\begin{enumerate}
\item Calculate $gcd(5, 7)$ and find two pairs of integers $(u, v)$ such that $gcd(5, 7) = 5u + 7v$.
\begin{quote}
By the definition of the $gcd$, we have that $gcd(5,7) = 1$.
\begin{gather*}
5(3) + 7(-2) = 1\\
5(10) + 7(-7) = 1
\end{gather*}
\end{quote}

\item Show that $c|a$ and $c|b$ if and only if $c | gcd(a,b)$.
\begin{quote}

Assume $c|a$ and $c|b$. Then $c$ is a common factor of $a$ and $b$. We know that $gcd(a,b)$ is also
a common factor of both $a$ and $b$. Since both $gcd(a,b)$ and $c$ are common factors of $a$ and
$b$, then they must also be common factors of each other. Therefore, $c|gcd(a,b)$.\\

Let $d = gcb(a,b)$, and assume that $c|d$. Then $d = cn$ for some $n\in\ZZ$. By the definition of
$gcd$, we know that $d|a$. Therefore, $cn|a$, which means that $a = cnm$ for some $m\in\ZZ$. Then
$a = cj$, where $j = nm\in\ZZ$. Therefore $c|a$. We also know that $d|b$ by the definition of
$gcd$. By the same argument, we know that $c|b$ as well. Therefore $c|b$ and $c|a$.
\end{quote}
$\triangle$

\item Suppose $a_{1},a_{2},\dots,a_{n}$ are integers not all equal to 0. We define
$gcd(a_{1},a_{2},...,a_{n})$ to be the largest integer which divides $a_{k}$ for all $1 \leq k \leq
n$. Prove that $gcd(a_{1},a_{2},\dots,a_{n}) = gcd(gcd(a_{1},a_{2}),a_{3},...,a_{n})$.
\begin{quote}
Let $gcd(a_{1},a_{2},\dots,a_{n}) = d$ and $gcd(gcd(a_{1},a_{2}),a_{3},\dots,a_{n}) = \alpha$. We
know that, by the definition of $gcd$, that $d|a_{k}$ for $1 \leq k \leq n$. We know that $d|a_{1}$
and $d|a_{2}$, so it must also divide $gcd(a_{1},a_{2})$. Therefore, $d|gcd(gcd(a_{1},a_{2})
,a_{3},\dots,a_{n})$, so $d|\alpha$. We also know that $\alpha|gcd(a_{1},a_{2})$, and that
$\alpha|a_{k}$ for $2 < k \leq n$. Since $\alpha$ divides $gcd(a_{1},a_{2})$, it must divide
$a_{1}$ and $a_{2}$. Therefore $\alpha|a_{k}$ for $1 \leq k \leq n$, so
$\alpha|gcd(a_{1},a_{2},\dots,a_{n})$, or $\alpha|d$. We know that if $d|\alpha$ and
$\alpha|d$, then $|d| = |\alpha|$. In this case, both $\alpha$ and $d$ are positive,
so we know that $\alpha = d$.
\end{quote}
$\triangle$

\item Use your answer to the previous problem, along with the Euclidian Algorithm, to determine
$gcd(1092,1155,2002)$ (It's possible that you'll need a calculator to do the arithmetic on this
problem).
\begin{align*}
1155 &= 1(1092) + 63\\
1092 &= 17(63) + 21\\
63 &= 3(21) + 0\\
\therefore\gcd{1155}{1092} &= \gcd{21}{0} = 21\\
2002 &= 95(21) + 7\\
21 &= 3(7) + 0\\
\therefore\gcd{21}{2002} &= \gcd{7}{0} = 7
\end{align*}

\item Let $a_{1},a_{2},\dots,a_{n} \in \NN$ and consider the two following definitions:
\begin{itemize}
\item We say that $a_{1},a_{2},\dots,a_{n}$ are relatively prime if
$gcd(a_{1},a_{2},\dots,a_{n}) = 1$.
\item We say that $a_{1},a_{2},\dots,a_{n}$ are pairwise relatively prime if
$gcd(a_{i},a_{j}) = 1$ for all $i\neq j$.
\end{itemize}
\begin{enumerate}
\item If $a_{1}, a_{2},\dots,a_{n}$ are pairwise relatively prime can we conclude that
$a_{1}, a_{2},\dots,a_{n}$ are relatively prime? Either prove your answer or give a counterexample.
\begin{quote}
Let $(a_{n}) = (a_{1},a_{2},\dots,a_{n})$, and assume $(a_{n})$ is pairwise relatively prime.
Then for any $i,j \leq n$ we have that $\gcd{a_{i},a_{j}} = 1$. This means that if we take any
to distinct elements in $(a_{n})$, they will have no common factors except for 1. As a result, the
greatest common factor is 1, or $gcd((a_{n})) = 1$, which is the definition of begin relatively
prime.
\end{quote}
$\triangle$
\item If $a_{1}, a_{2},\dots,a_{n}$ are relatively prime can we conclude that $a_{1},
a_{2},\dots,a_{n}$ are pairwise relatively prime? Either prove your answer or give a
counterexample.
\begin{quote}
Consider $(a_{n}) = (2,3,4)$. We know that $gcd((a_{n})) = 1$, but $gcd(2,4)\neq 1$.
Therefore, we can conclude that $(a_{n})$ being relatively prime does not necessarily
mean that $(a_{n})$ is pairwise relatively prime.
\end{quote}
$\triangle$
\end{enumerate}

\item Suppose that $a | c$ and $b | c$. Do we necessarily have that $ab | c$? Either prove your
answer or give a counterexample.
\begin{quote}
Let $a,b = 4$ and $c = 8$. We know that $4|8$, but $(4\cdot4)\cancel{|}8$. Therefore $ab$ does
not divide $c$.
\end{quote}
$\triangle$

\item If $a|c$ and $b|c$ prove that $\lcm{a}{b}|c$. Conclude that if $a$ and $b$ are relatively
prime, then $ab|c$.
\begin{quote}
We know that since $a|c$ and $b|c$, $c$ is a common factor of both $a$ and $b$. If we assume
that $c$ is the smallest common factor of both $a$ and $b$, then $c = \lcm{a}{b}$, and $c|c$.
Otherwise, let $\delta = \lcm{a}{b}$. Then, by the definition of the $lcm$, we know that
$\delta$ is a common factor of $a$ and $b$, and that $\delta < c$. Since both $c$ and $\delta$
are common factors of $a$ and $b$, it follows that $\delta|c$.\\

Since $a$ and $b$ are relatively prime, $\gcd{a}{b} = 1$. We also know that $ab =
\lcm{a}{b}\gcd{a}{b}$, which means that $ab = \lcm{a}{b}$. Therefore, $ab|c$.
\end{quote}
$\triangle$
\end{enumerate}
\end{document}
