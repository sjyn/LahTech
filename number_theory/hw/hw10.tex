\documentclass{hw}

\begin{document}
\makeheader{10}
\begin{enumerate}
\item Let $\phi$ denote Euler's Function.
\begin{enumerate}
\item Calculate $\phi(22)$.
\[
\phi(22) = \phi(11)\phi(2)=10\cdot1 = 10.
\]
\item Does there exist $l\in\ZZ$ such that $a^{7l}\equiv a\mod 22$ for all units $a\in\ZZ_{22}$? If
so, find $l$. Otherwise, explain why no such $l$ exists.
\begin{quote}
We know that we can find an $l$ if $7l\equiv1\mod\phi(22)$. Since $gcd(7,10)=1$, there is an inverse
of 7 in $\ZZ_{10}$, which is 3. Therefore, $a^{21}\equiv a\mod22$ for all units $a\in\ZZ_{22}$.
\end{quote}
\end{enumerate}

\item Suppose that $a\in\ZZ$ is such that $gcd(a,15)=1$. If $gcd(k,8)=1$, prove that
$a^{k^{2}}\equiv a\mod15$.
\begin{quote}
Suppose $gcd(k,8)=1$, and define the set $U_{8}$ to be the set of units in $\ZZ_{8}$. We want to find
$k^{-1}$ such that $k^{-1}k\equiv1\mod\phi(15)$, which allows us to say $a^{k^{-1}k}\equiv a\mod15$.
We know that $U_{8}=\{ 1,3,5,7 \}$. Since $gcd(k,8)=1$, $k\in U_{8}$. Suppose that $k=1$. Then
$k^{-1}=1$, so $a^{k^{2}}\equiv a^{1}\equiv a\mod15$. If $k=3$, $k^{-1}=3$, so $a^{k^{2}}\equiv
a^{9}\equiv a\mod15$. If $k=5$, then $k^{-1}=5$, and $a^{k^{2}}\equiv a^{25}\equiv a\mod15$. If $k=7$,
then $k^{-1}=7$, so $a^{k^{2}}\equiv a^{49}\equiv a\mod15$.
\end{quote}

\item Let $\Omega$ denote the prime counting map and $\lambda$ the Liouville function. Prove that
$\lambda(x)=\lambda(y)$ if and only if $\Omega(x)\equiv\Omega(y)\mod2$.
\begin{quote}
$\Rightarrow)$ Suppose $\lambda(x)=\lambda(y)$. We know
\[
\lambda(a)=
\begin{cases}
1,\ \ \qquad\Omega(a)\in\{2k|k\in\ZZ\}\\
-1,\qquad\Omega(a)\in\{2k+1|k\in\ZZ\}
\end{cases}.
\]
Therefore, the only way $\lambda(x)=\lambda(y)$ is if the parity of $\Omega(x)$ is the same
as the parity of $\Omega(y)$. Assume that $\Omega(x),\Omega(y)\in\{ 2k|k\in\ZZ\}$. Then
$\Omega(x)\equiv\Omega(y)\equiv0\mod2$. Assume $\Omega(x),\Omega(y)\in\{ 2k+1|k\in\ZZ\}$. Then
$\Omega(x)\equiv\Omega(y)\equiv1\mod2$. Therefore, if $\lambda(x)=\lambda(y)$,
$\Omega(x)\equiv\Omega(y)\mod2$.\\\\
$\Leftarrow)$ Suppose that $\Omega(x)\equiv\Omega(y)\mod2$. We know that $\lambda(a)$ is defined
to be $(-1)^{\Omega(a)}$, so suppose without loss of generality that
$\Omega(x),\Omega(y)\in\{ 2k|k\in\ZZ\}$. Then $\lambda(x)=(-1)^{2k}=1$, and $\lambda(y)=(-1)^{2k}=1$,
so $\lambda(x)=\lambda(y)$.
\end{quote}

\item Let $i=\sqrt{-1}\in\CC$ and define $\rho:\NN\to\CC$ by $\rho(x)=i^{\Omega(x)}$. Prove that
$\rho(x)=\rho(y)$ if and only if $\Omega(x)\equiv\Omega(y)\mod4$.
\begin{quote}
$\Rightarrow)$ Suppose $\rho(x)=\rho(y)$. We know that $i$ to a power is either $1,-1,i,$ or $-i$.
If $\rho(x)$ is to equal $\rho(y)$, then both must equal one of the four variations of $i$ to a
power. Suppose $\rho(x)=\rho(y)=1$. Then $i^{\Omega(x)}=i^{\Omega(y)}=1$, so $\Omega(x)$ and
$\Omega(y)$ must be a multiple of 4. Then $\Omega(x)\equiv\Omega(y)\equiv0\mod4$. If
$\rho(x)=\rho(y)=-1$, then $\Omega(x)$ and $\Omega(y)$ must be a multiple of 2 but not a multiple
of 4, since being a multiple of 4 would cause $\rho(x)=\rho(y)=1$. If $\Omega(x)$ and $\Omega(y)$
are multiples of 2 and not multiples of 4, then $\Omega(x)\equiv\Omega(y)\equiv2\mod4$. Suppose that
$\rho(x)=\rho(y)=i$. Then $i^{\Omega(x)}=i^{\Omega(y)}=i$. We know that $\Omega(x)$ and $\Omega(y)$
must be in the set $\{ 4k+1|k\in\ZZ\}$, or $\{ 4k+3|k\in\ZZ\}$. Suppose
$\Omega(x),\Omega(y)\in\{ 4k+1|k\in\ZZ\}$. Then $i^{\Omega(x)}=i^{\Omega(y)}=i^{4k+1}=i^{4k}i=i$.
In this case, we have $\Omega(x)\equiv\Omega(y)\equiv1\mod4$. Finally, suppose that
$\Omega(x),\Omega(y)\in\{ 4k+3|k\in\ZZ\}$. Then $i^{\Omega(x)}=i^{\Omega(y)}=i^{4k+3}=i^{4k}(-i)=-i$.
In this case, $\Omega(x)\equiv\Omega(y)\equiv3\mod4$. Therefore, if $\rho(x)=\rho(y)$,
$\Omega(x)\equiv\Omega(y)\mod4$.\\\\

$\Leftarrow)$ Suppose $\Omega(x)\equiv\Omega(y)\mod4$. Then $\Omega(x),\Omega(y)\in\{0,1,2,3\}$.
If $\Omega(y)=0$, then $\Omega(x)=0$, and $\rho(x)=i^{0}=1$ and $\rho(y)=i^{0}=1$, so $\rho(x)=\rho(y)$.
If $\Omega(x)=\Omega(y)=1$, then  $\rho(x)=i=\rho(y)$. If $\Omega(x)=\Omega(y)=2$, then
$\rho(x)=-1=\rho(y)$. If $\Omega(x)=\Omega(y)=3$, then $\rho(x)=-i=\rho(y)$.
\end{quote}

\item Suppose that $f:\NN\to\CC$ is a function such that $f(xy)=f(x)+f(y)$ for all $x,y\in\NN$. If
$f(p)=1$ for all primes $p$, prove that $f(x)=\Omega(x)$ for all $x\in\NN$.
\begin{quote}
We know that we can express $x$ and $y$ as
\begin{gather*}
x=p_{1}^{\alpha_{1}}p_{2}^{\alpha_{2}}+\dots+p_{n}^{\alpha_{n}}\\
y=p_{1}^{\beta_{1}}p_{2}^{\beta_{2}}+\dots+p_{n}^{\beta_{n}}.
\end{gather*}
We can express $f(xy)-f(y)$ as $f(x)+f(y)-f(y)=f(x)$, and we also know
\begin{align*}
f(xy)&=f(p_{1}^{\alpha_{1}+\beta_{1}}p_{2}^{\alpha_{2}+\beta_{2}}+\dots+p_{n}^{\alpha_{n}+\beta_{n}})\\
%
&= f(p_{1}^{\alpha_{1}}) + f(p_{1}^{\beta_{1}}) + f(p_{2}^{\alpha_{2}}) + f(p_{2}^{\beta_{2}})+
\dots + f(p_{n}^{\alpha_{n}}) + f(p_{n}^{\beta_{n}})\\
%
f(xy)-f(y)&=f(x)\\
&=f(p_{1}^{\alpha_{1}}p_{2}^{\alpha_{2}}\dots p_{n}^{\alpha_{n}})\\
%
&=f(p_{1}^{\alpha_{1}})+f(p_{2}^{\alpha_{2}})+\dots+f(p_{n}^{\alpha_{n}})\\
%
&=\sum_{k=1}^{\alpha_{1}}f(p_{1}) + \sum_{k=1}^{\alpha_{2}}f(p_{2}) + \dots +
\sum_{k=1}^{\alpha_{n}}f(p_{n})\\
%
&= \sum_{k=1}^{\alpha_{1}}1 + \sum_{k=1}^{\alpha_{2}}1\dots + \sum_{k=1}^{\alpha_{n}}1\\
&= \alpha_{1} + \alpha_{2} + \dots + \alpha_{n}\\
&=\Omega(x)
\end{align*}
\end{quote}

\item For an ordered pair $(a,b)\in\NN\times\NN$ we define the \textit{Generalized Prime Counting Map}
by \[\overline{\Omega}(a,b)=\Omega(a)-\Omega(b).\] If $(a,b),(c,d)\in\NN\times\NN$ are such that
$ad=bc$, prove that $\overline{\Omega}(a,b)=\overline{\Omega}(c,d)$.
\begin{quote}
Suppose $ad=bc$. We can write out the prime factorizations of $a,b,c$, and $d$ as
\begin{align*}
a &= p_{1}^{\alpha_{1}}p_{2}^{\alpha_{2}}\dots p_{n}^{\alpha_{n}}\\
b &= p_{1}^{\beta_{1}}p_{2}^{\beta_{2}}\dots p_{n}^{\beta_{n}}\\
c &= p_{1}^{\gamma_{1}}p_{2}^{\gamma_{2}}\dots p_{n}^{\gamma_{n}}\\
d &= p_{1}^{\delta_{1}}p_{2}^{\delta_{2}}\dots p_{n}^{\delta_{n}}.
\end{align*}
Then
\begin{align*}
ad&=p_{1}^{\alpha_{1}+\delta_{1}}p_{2}^{\alpha_{2}+\delta_{2}}\dots p_{n}^{\alpha_{n}+\delta_{n}}\\
bc&=p_{1}^{\beta_{1}+\gamma_{1}}p_{2}^{\beta_{2}+\gamma_{2}}\dots p_{n}^{\beta_{n}+\gamma_{n}}.
\end{align*}
We can consider $\overline{\Omega}(ad,bc)=\Omega(ad)-\Omega(bc)=0$. We know
$\Omega(ad)-\Omega(bc)=0$ since $ad=bc$ implies they will have the same prime factorization up to
permutation, so the powers of the prime factors of $ad$ will be equivalent to the powers of the prime
factors of $bc$. We now have
\begin{align*}
\Omega(ad)-\Omega(bc)&=(\alpha_{1}+\delta_{1})+(\alpha_{2}+\delta_{2})+\dots+(\alpha_{n}+\delta_{n})
- (\beta_{1}+\gamma_{1})-(\beta_{2}+\gamma_{2})-\dots-(\beta_{n}+\gamma_{n})\\
&= (\alpha_{1}+\alpha_{2}+\dots+\alpha_{n})-(\beta_{1}+\beta_{2}+\dots+\beta_{n})+
(\delta_{1}+\delta_{2}+\dots+\delta_{n})-(\gamma_{1}+\gamma_{2}+\dots+\gamma_{n})\\
&=\Omega(a)-\Omega(b)-\Omega(c)+\Omega(d)\\
&= 0.
\end{align*}
Then $\Omega(a)-\Omega(b)=\Omega(c)-\Omega(d)$, which by definition is
$\overline{\Omega}(a,b)=\overline{\Omega}(c,d)$.
\end{quote}

\newpage
\item Let $\mu$ be the M\"{o}bius Function.
\begin{enumerate}
\item If $p$ and $q$ are distinct primes, prove that $\mu(pq)=1$.
\begin{quote}
Since $\mu$ is multiplicative, we have that $\mu(xy)=\mu(x)\mu(y)$ when $gcd(x,y)=1$. Since $p$ and
$q$ are distinct, we know that $\mu(pq)=\mu(p)\mu(q)$. By definition of the M\"{o}bius function, for
a prime $\rho$ we have
\[
\mu(\rho)=-\sum_{\substack{d\mid \rho\\d\neq\rho}}\mu(\rho)=-\mu(1)=-1,
\]
so $\mu(p)\mu(q)=(-1)(-1)=1$.
\end{quote}
\item If $p$ is prime, prove that $\mu(p^{2})=0$.
\begin{quote}
From $(a)$, we know that $\mu(\rho)=-1$ for some prime $\rho$. If we consider $p^2$, we know that the
only divisors are $1,p,$ and $p^2$. By the definition of the M\"{o}bius function, we have
\[
\mu(p^2)=-\sum_{\substack{d\mid p^2\\d\neq p^2}}\mu(d)=-(\mu(1) + \mu(p))=-1 +1 = 0.
\]
\end{quote}
\item If $p$ is prime and $n$ is a positive integer, find a formula for the value of $\mu(p^{n})$.
Prove your answer.
\begin{quote}
\[
\mu(p^n)=
\begin{dcases*}
-1,\qquad n=1\\
0,\ \ \qquad n>1
\end{dcases*}
\]
Consider the value of $\mu(p^{n})$. We have already shown that $\mu(p)=-1$ for any prime $p$ and
that $\mu(p^2)=0$ for any prime. We can suppose that $\mu(p^{n})=0$ for $n\geq2$, and
$\mu(p^{n})=-1$ for $n=1$. We want to show that $\mu(p^{n+1})=0$ for $n\geq2$. Then
\begin{align*}
\mu(p^{n+1})&=-\sum_{\substack{d\mid p^{n+1}\\ d\neq p^{n+1}}}\mu(d)\\
&= -(\mu(1)+\mu(p)+\mu(p^{2})+\mu(p^{3})+\dots+\mu(p^{n}))\\
&= -(1+(-1)+0+0+\dots+0)\text{ by the inductive hypothesis}\\
&= -(0+0+0+\dots+0)\\
&= -0\\
&= 0.
\end{align*}
\end{quote}
\end{enumerate}

% \item If $\phi$ denotes Euler's Function and $s>2$, prove that
% \[
% {\zeta(s-1)\over\zeta(s)}=\sum_{n=1}^{\infty}{\phi(n)\over n^{s}}.
% \]
\end{enumerate}
\end{document}
