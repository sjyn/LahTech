\documentclass{hw}

\begin{document}
\makeheader{3}

\begin{enumerate}
\item Suppose that $a,b\in\ZZ$ and define the set $J={ax+by|x,y\in\ZZ}$. Prove that $J=Z$ if and only if
$gcd(a, b) = 1$.
\begin{quote}
Assume $J=\ZZ$. We know that $1\in\ZZ$, so $1\in J$, and $1\in\{ax+by|x,y\in\ZZ\}$. Therefore
$1=ax+by$, so $gcd(a,b)=1$ by Bezout's Identity.\\\\
Assume $gcd(a,b)=1$. Then there exists $u,v\in\ZZ$ such that $1 = au+bv$ by Bezuot's Identity. We know
that any $n\in\ZZ$, we can represent it as $1\cdot n = a\cdot un + b\cdot vn$. Let $x=un$ and $y=vn$.
Then we have $n = ax+by$ where $ax+by\in\ZZ$. Since $n$ is any arbitrary integer, we can say that
$ax+by$ is any arbitrary integer for any $x,y\in\ZZ$, or $\{ax+by|x,y\in\ZZ\}=\ZZ$
\end{quote}
$\triangle$

\item We define the \textit{Fibonacci Sequence} to be the sequence of integers
${x_{0}, x_{1}, x_{2},\dots}$
satisfying the properties
\[
x_{0} = 0,\quad x_{1} = 1,\quad\text{and}\quad x_{n} = x_{n-1}+x_{n-2}\text{ for all } n\geq2.
\]
Prove that $gcd(x_{n},x_{n-1}) = 1$ for all $n \geq 1$. (Hint: Try using induction on $n$.)
\begin{quote}
For the base case, $n=2$, we have that
\begin{align*}
x_{2} &= x_{1} + x_{0}\\
&= 1 + 0\\
&= 1\\
x_{1} &= 1\\
gcd&(x_{1},x_{2})= 1.
\end{align*}
We can assume that $gcd(x_{n},x_{n-1}) = 1$ for $n\geq2$. We now need to show that
$gcd(x_{n+1},x_{n}) = 1$ for $n\geq1$. We also have that $x_{n+1}=x_{n}+x_{n-1}$,
so $gcd(x_{n+1},x_{n}) = gcd(x_{n}+x_{n-1},x_{n})$. By Bezout's Identity, we know that
$gcd(x_{n}+x_{n-1},x_{n}) = (x_{n}+x_{n-1})u + x_{n}v$ for some $u,v\in\ZZ$. Then
\begin{align*}
gcd(x_{n}+x_{n-1},x_{n}) &= x_{n}u+x_{n-1}u + x_{n}v\\
&= x_{n-1}u + x_{n}(u+v)\\
&= x_{n-1}u + x_{n}w\\
&= gcd(x_{n-1},x_{n})\\
&= 1\text{ by the induction hypothesis}.
\end{align*}
Therefore, $gcd(x_{n+1},x_{n}) = 1$.
\end{quote}
$\triangle$

\item Let $a, b, x$ be positive integers with $x \geq 2$ and set $d = gcd(a, b)$.
\begin{enumerate}
\item Prove that $x^{d} - 1$ divides $gcd(x^{a} - 1, x^{b} - 1)$.
\begin{quote}
We know that $d = gcd(a, b)$, so $d|a$ and $d|b$. We have $a=dn$ and $b=dm$. Then $x^{a}-1=x^{dn}-1$ and
$x^{b}-1=x^{dm}-1$. We know that $x^{dn}=(x^{d})^{n}$, so we can rewrite $x^{dn}-1$ as $(x^{d})^{n}-1$.
Now we have that $(x^{d})^{n}-1 = (x^{d}-1)(x^{dn-1}+x^{dn-2}+\cdots+1)$. Recall that
$(x^{d})^{n}-1 = x^{a} - 1$, so we have $x^{a} - 1 = (x^{d}-1)j$, where
$j=(x^{dn-1}+x^{dn-2}+\cdots+1)\in\ZZ$. Therefore, $(x^{a} - 1)|(x^{d}-1)$. We can apply the same
argument for $x^{b}-1$, and we will find that $(x^{b} - 1)|(x^{d}-1)$. Since $(x^{d}-1)$ divides both
$(x^{b}-1) \text{ and } (x^{a}-1)$, it divides $gcd(x^{a} - 1, x^{b} - 1)$.
\end{quote}
\item Prove that $x^{d}-1$ is a multiple of $gcd(x^{a}-1,x^{b}-1)$ and conclude that
$x^{d}-1 = gcd(x^{a}-1,x^{b}-1)$. (Hint: We know that there exist integers $u$ and $v$ such that
$d = au + bv$. Now show that there exist integers $\alpha$ and $\beta$ such that
$x^{d}-1 = \alpha(x^{a}-1) + \beta(x^{b}-1)$.)
\begin{quote}
Since we know $d=au+bv$ for $u,v\in\ZZ$, we can say
\begin{align*}
x^{d}-1 &= x^{au+bv}-1\\
&= x^{au}x^{bv}-1\\
&= x^{au}x^{bv}-1+x^{au}-x^{au}\\
&= x^{au}\cdot(x^{bv}-1)+1\cdot(x^{au}-1).
\end{align*}
If we let $a=^{au}$ and $b=1$, then we have $x^{d}-1 = a\cdot(x^{bv}-1)+b\cdot(x^{au}-1)$. Therefore,
by Bezout's Identity, $x^{d}-1 = gcd(x^{a}-1,x^{b}-1)$.
\end{quote}
$\triangle$
\end{enumerate}


\item Show that the equation $1495x + 50060y = 4$ has no solutions for $x,y\in\ZZ$.
\begin{quote}
\begin{align*}
50060 &= 1495(33) + 725\\
1495 &= 725(2) + 45\\
725 &= 45(16) + 5\\
45 &= 5(9) + 0\\
gcd(50060,1495) &= 5
\end{align*}
However, $5\cancel{|}4$, so there is no solution.
\end{quote}

\item Find all solutions to the equation $7x+4y=1$ for $x,y\in\ZZ$.
\begin{center}
\begin{minipage}[t]{0.4\textwidth}
\begin{align*}
7 &= 4\cdot1 + 3\\
4 &= 3\cdot1 + 1\\
3 &= 1\cdot3 + 0
\end{align*}
\end{minipage}
\begin{minipage}[t]{0.4\textwidth}
\begin{align*}
1 &= 4\cdot1 - 3\cdot1\\
&= 4\cdot1 - (7-4)\cdot1\\
&= 4\cdot1 -7\cdot1 + 4\cdot1\\
&= 4\cdot2-7\cdot1
\end{align*}
\end{minipage}
\end{center}
$x = 1+7n$ and $y=2+4n$ are solutions to the Diophantine equation.

\item Find all solutions to the equation $1485x + 1745y = 15$ for $x,y\in\ZZ$.
\begin{center}
\begin{minipage}[t]{0.4\textwidth}
\begin{align*}
1745 &= 1485(1) + 260\\
1485 &= 260(5) + 185\\
260 &= 185(1) + 75\\
185 &= 72(2) + 35\\
75 &= 35(2) + 5\\
35 &= 7(5) + 0\\
gcd&(1745,1485) = 5
\end{align*}
\end{minipage}
\begin{minipage}[t]{0.4\textwidth}
\begin{align*}
5 &= 75 - 35(2)\\
5 &= 75 - (185 - 75(2))(2)\\
&= 75(5) - 185(2)\\
5 &= (260-185(1))(5)-185(2)\\
&= 260(5)-187(7)\\
5 &= 260(5) - (1485 -260(5))(7)\\
&= 260(40) + 1485(-7)\\
5 &= (1745 - 1485(1))(40) + 1485(-7)\\
&= 1745(40) + 1485(-47)\\
15 &= 1745(120) + 1485(-141)
\end{align*}
\end{minipage}
\end{center}

$x = -141 + {349n\over3}$ and $y = 120 + 99n$ are solutions to the equation.

\item Suppose you have two small champagne glasses, one holding 8 ounces and another holding 5 ounces. Is
it possible to fill one of the glasses with exactly 1 ounce of champagne? If so, how can this be done? If
not, prove that it cannot be done.
\begin{quote}
We need to find a solution to the Diophantine Equation
\[
8x+5y=1.
\]
We know that $gcd(8,5)=1$, and $1|1$, so there is a solution. Working backwards through the Euclidian
Algorithm gives us
\begin{align*}
1 &= 3-2(1)\\
&= 3(2)-5\\
&= 8(2)+5(-3).
\end{align*}
Therefore, if we fill the first glass up twice and empty the second glass three times, we will end up with
1 ounce leftover.
\end{quote}
\end{enumerate}
\end{document}
