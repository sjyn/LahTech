\documentclass{hw}

\begin{document}
\makeheader{7}

\begin{enumerate}
\item If $p$ is prime and $x,y\in\ZZ_{p}$, prove that $(x-y)^{p}\equiv x^{p}-y^{p}\mod p$.
\begin{quote}
\begin{align*}
(x-y)^{p} &\equiv (x-y)\mod p \qquad\text{ by the corollary to FLT}\\
&\equiv (x\mod p) - (y\mod p)\\
&\equiv (x^{p}\mod p) - (y^{p}\mod p)\qquad\text{ by the corollary to FLT}\\
&\equiv x^{p} - y^{p} \mod p
\end{align*}
\end{quote}

\item Suppose that $p$ is prime and $k$ is a positive integer, and $x\in\ZZ_{p}$.
\begin{enumerate}
\item Prove that $x^{p^{k}}\equiv x\mod p$.

\item Prove that $x^{p^{k}-1}\equiv 1\mod p$ if and only if $\gcd{(x,p)}=1$.
\end{enumerate}

\item Let $p,n$, and $r$ be non-negative integers with $p$ prime. Further, assume that $r$ is the
remainder when $n$ is divided by $p-1$. Prove that $a^{n}\equiv a^{r}\mod p$.

\item Assume that $n$ is a positive integer. Prove that $1^{n}+2^{n}+3^{n}+4^{n}$ is divisible by
5 if and only if $n$ is not divisible by 4.

\item Determine whether there exists a solution to each of the following systems of congruences.  If
there is a solution, find all solutions to the system by writing the solution set as a single residue
class modulo $n$ for some $n\geq 2$. If there is no solution, prove that there is no solution.
\begin{enumerate}
\item $x\equiv5\mod7$\\
$x\equiv0\mod4$
\begin{align*}
x_{0}&=a_{1}c_{1}d_{1}+1_{2}c_{2}d_{2}\\
&= 5c_{1}d_{1} + 0c_{2}d_{2}\\
&= 5\cdot4\cdot d_{1}\\
&= 5\cdot4\cdot2\\
&= 40\\
&\equiv 12\mod28
\end{align*}

\item $x\equiv 5\mod7$\\
$x\equiv1\mod4$\\
$x\equiv0\mod5$
\begin{quote}
The $\gcd{(7,4,5)}=1$, so by the Chinese remainder theorem, we have a solution.
\begin{align*}
x_{0}&=a_{1}c_{1}d_{1}+a_{2}c_{2}d_{2}+a_{3}c_{3}d_{3}\\
&=5\cdot c_{1}\cdot d_{1} + 1\cdot c_{2}\cdot d_{2} + 0\\
&=5\cdot20\cdot d_{1}+1\cdot35\cdot d_{2}\\
&=5\cdot20\cdot-1 + 1\cdot35\cdot3\\
&=-100+105\\
&=5\\
&\equiv5\mod140
\end{align*}
\end{quote}

\item $x\equiv5\mod6$\\
$x\equiv2\mod4$
\begin{quote}
We cannot use Chinese remainder theorem here since 6 and 4 are not coprime. Suppose, however, that
there is a solution. We can form a new system by determining the prime factorization of 6.
\[
\begin{cases}
x\equiv5\mod2\\
x\equiv5\mod3\\
x\equiv2\mod4
\end{cases}
\to
\begin{cases}
x\equiv1\mod2\\
x\equiv2\mod3\\
x\equiv2\mod4
\end{cases}
\]
If this is the case, then $x\equiv1\mod2$ implies that the solution is odd, and $x\equiv2\mod4$ implies
the solution is even. This is not possible; therefore there is no solution.
\end{quote}

\item $3x\equiv1\mod10$\\
$5x\equiv2\mod7$
\begin{quote}
We cannot initially use Chinese remainder theorem to solve this problem. If we find $3^{-1}$ in
$\ZZ_{10}$, then we can multiply both sides of the congruence by that value to produce a new
congruence. We have $3x\equiv1\mod10$, which is satisfied by $x=7=3^{-1}$. We can rewrite this
congruence as $x\equiv7\mod10$. Similarly, we can find $5^{-1}$ in $\ZZ_{7}$. We have that
$5x\equiv1\mod7$, so $x=3=5^{-1}$. Multiplying both sides of the congruence yields the new system
\[
\begin{cases}
x\equiv7\mod10\\
x\equiv3\mod7
\end{cases}.
\]
We know this has a solution by the Chinese remainder theorem, since 10 and 7 are coprime.
\begin{align*}
x_{0}&=a_{1}c_{1}d_{1}+a_{2}c_{2}d_{2}\\
&=7\cdot7\cdot d_{1}+3\cdot10\cdot d_{2}\\
&=7\cdot7\cdot3+3\cdot10\cdot5\\
&=147+150\\
&=297\\
&\equiv13\mod70
\end{align*}
\end{quote}
\end{enumerate}

\item Find all solutions to the congruence $97x\equiv301\mod315$. It may be helpful to note that
$315=3^{2}\cdot4\cdot7$.
\begin{quote}
We can split this congruence into several parts.
\[
\begin{cases}
97x\equiv301\mod9\\
97x\equiv301\mod5\\
97x\equiv301\mod7
\end{cases}
\to
\begin{cases}
7x\equiv4\mod9\\
2x\equiv1\mod5\\
6x\equiv0\mod7
\end{cases}
\to
\begin{cases}
x\equiv7\mod9\\
x\equiv3\mod5\\
x\equiv0\mod7
\end{cases}
\]
By the Chinese remainder theorem, which we can use since 9, 7, and 5 are coprime, we have that
\begin{align*}
x_{0}&=7c_{1}d_{1}+3c_{2}d_{2}+0c_{3}d_{3}\\
&=7\cdot35\cdot d_{1}+3\cdot63\cdot d_{2} + 0\\
&=7\cdot35\cdot -1+3\cdot63\cdot 2 + 0\\
&=-245+378\\
&=133\\
&\equiv133\mod315.
\end{align*}
\end{quote}

\item Find all solutions to the congruence $x^{1000}\equiv1\mod10$.
\begin{quote}
By the prime factorization of 10, we have that
\[
5\mid(x^{1000}-1)\qquad\qquad\text{and}\qquad\qquad2\mid(x^{1000}-1).
\]
Since $2\mid(x^{1000}-1)$, we have that $x^{1000}\equiv1\mod2$, implying that $x$ is odd. This leaves
us with two possibilities in $\ZZ_{5}$, namely $\{1,3\}$. We can express 1000 as $5\cdot5\cdot5\cdot8$,
so we have
$((((x)^{8})^{5})^{5})^{5}\equiv(((x)^{8})^{5})^{5}\equiv((x)^{8})^{5}\equiv x^{8}\equiv1\mod5$.
We can express 8 as $4\cdot2$, so we have $(x^{2})^{4}\equiv1\mod5$, which means $x^{2}\equiv1\mod5$
by Fermat's Little Theorem. We know $x$ is either 1 or 3, so we can test the values. If we try 3, we
get that $9\equiv1\mod5$, which is not true, whereas if we try 1, we get $1\equiv1\mod5$, which is true.
Therefore, $x=1$.
\end{quote}
\end{enumerate}
\end{document}
