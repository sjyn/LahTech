\documentclass{hw}

\usepackage{cancel}

\begin{document}
\makeheader{6}

\begin{enumerate}
\item For each of the following linear congruences of the form $ax\equiv c\mod n$, determine whether a
solution exists. If so, find a formula for all solutions and determine how many solutions there are in
$\ZZ_{n}$.
\begin{enumerate}
\item $3x\equiv 5\mod7$
\begin{quote}
The $gcd$ of $a$ and $n$ is 1, so there is a solution since $1\mid5$, and there is only one solution.
If we solve the Diophantine equation $3x+7y=5$, we get a general solution for x as $x=-3+7n$. The
solution we want is 4, since all other $n$ give a solution equal to 4 in $\ZZ_{7}$.
\end{quote}
\item $4x\equiv 9\mod12$
\begin{quote}
This congruence has no solution since $gcd(4,12)=4$, but $4\nmid9$.
\end{quote}
\item $18x\equiv 27\mod45$
\begin{quote}
The $gcd(18,45)=9$, and $9\mid27$, so there is a solution, and in fact there are 9 solutions. If we
solve the Diophantine equation $18x_{0}+45y_{0}=27$, we get that $x_{0}=-1+5n$. The nine solutions can
be found by starting with $n=1$ to $n=9$, and are $4,9,14,19,24,29,34,39,44$.
\end{quote}
\end{enumerate}

\item Let $S$ denote the number of solution to the linear congruence $ax\equiv c\mod20$. Prove that
$S\in\{ 0,1,2,4,5,10,20\}$.
\begin{quote}
We know that there are either 0 or $gcd(a,20)$ solutions to the congruence $ax\equiv c\mod20$.
Suppose we have the set $A=\{0,1,2,3,4,5,6,7,8,9,10,11,12,13,14,15,16,17,18,19,20\}$. We know that for
these numbers, we will have a unique $gcd$ with 20, and in fact all possible $gcd's$ will be
represented in this set. If we find the $gcd$ of a relatively prime number $r$ and 20, then
$gcd(r,20)=1$, so it is possible to have only one solution. In addition, $gcd(1,20)=1$, so we can
eliminate this number as well. Now we can consider the set
$B=A\backslash\{3,7,9,11,13,17,19\}=\{0,2,4,5,6,8,10,12,14,15,16,18,20\}$. For both 0 and 20,
$gcd(0,20)=gcd(20,20)=20$, so it is possible to have 20 solutions. Now consider the set
$C=B\backslash\{0,20\}=\{2,4,5,6,8,10,12,14,15,16,18\}$. We can find all the elements of $C$ such that
$gcd(20,c)=2$, which are $\{2,6,14,18\}$, which leaves us with the set $D=\{4,5,8,10,12,15,16\}$.
Now we can consider all the elements with which give us a $gcd$ of 4, which are $\{4,8,12,16\}$.
This leaves us with the set $E=\{5,15\}$. If we take any element of $E$ and find $gcd(e,20)$, we
will get 5. Now we are only left with $gcd(10,20)=10$, which provides us with 10.
\end{quote}

\item Suppose that $a,n\in\ZZ$ with $n\geq3$ and $gcd(a,n)>1$. Prove that there exist at least two
non-zero points $c\in\ZZ_{n}$ such that $ax\equiv c\mod n$ has no solutions.
\begin{quote}
Suppose $ax\equiv c\mod n$. Then there are either $gcd(a,n)$ solutions, or 0 solutions. If there are
0 solutions, then there are $n$ number of non-solutions $c\in\ZZ_{n}$. Now suppose we have $gcd(a,n)$
solutions. We know that $gcd(a,n)>1$, so we can let $c_{1}=1$. In this case, the only thing that
divides $c_{1}$ is 1, but $gcd(a,n)>1$, so there is no solution. We can also choose $c_{2}=n-1$. In
this case, $n-1\equiv-1\mod n$, and the only thing that divides $-1$ is 1. Since $gcd(a,n)>1$, it does
not divide $-1$, and therefore there are no solutions.
\end{quote}

\item Determine whether each given point is a unit in the given $\ZZ_{n}$. If so, find its
multiplicative inverse. If not, explain why it fails to be a unit.
\begin{enumerate}
\item $3\in\ZZ_{6}$
\begin{quote}
$3\equiv1\mod6$ implies that $3x+6y=1$. However, $gcd(3,6)=3$, which does not divide 1. Therefore, 3
is not a unit.
\end{quote}
\item $7\in\ZZ_{12}$
\begin{quote}
The greatest common divisor of 7 and 12 is 1, so 7 is a unit in $\ZZ_{12}$. We have the relationship
that $7x\equiv1\mod12$, so $7x+12y=1$. One solution to this Diophantine equation is $x_{0}=-5$, so all
solutions can be expressed as $x=-5+12n,\ n\in\ZZ$. When $n=1$, $x=7$, which means that $7$ is its own
inverse.
\end{quote}
\item $13\in\ZZ_{18}$
\begin{quote}
$gcd(13,18)=1$, so 13 is a unit in $\ZZ_{18}$. We can form the relationship $13\equiv1\mod18$, so we
have that $13x+18y=1$. We have one solution, $x_{0}=7$, so all solutions $x$ can be expressed as
$x=7+18n,\ n\in\ZZ$. If we choose $n=0$, then we have $x=7$, so 7 is the inverse of 13.
\end{quote}
\end{enumerate}

\item Suppose that $p$ is prime.
\begin{enumerate}
\item Prove that the set $\{0,1,2,\dots,p^{2}-2,p^{2}-1\}$ contains exactly $p(p-1)$ elements which
are relatively prime to $p$. Conclude that $\ZZ_{p^{2}}$ contains exactly $p(p-1)$ units.
\begin{quote}
Suppose we consider the set containing all non-units of $\ZZ_{p^{2}}$. This set contains elements of
the form $\{0,p,2p,3p,\dots,p(p-1)\}$. We know this set contains $p$ elements, since it contains
$p-1$ multiples of $p$, and 0. If we subtract the number of elements in the $\ZZ_{p^{2}}$ from the
number of non-units, we get $p^{2}-p$, or $p(p-1)$.
\end{quote}
\item How many units does $\ZZ_{p^{n}}$ have? Prove your answer.
\begin{quote}
We can consider the set of non-units in $\ZZ_{p^{n}}$. This yields the set
\[\{0,p,2p,\dots,p^2,2p^{2},\dots,p^{3},\dots,p^{n-1}(p-1)\}.\] We know this set has size $p^{n-1}$,
since it contains all multiples of powers of $p$ up to the $n-1$ power. Again, we can subtract the
sizes of the entire set and the set of non-units, and we get $p^{n}-p^{n-1}=p^{n-1}(p-1)$.
\end{quote}
\end{enumerate}

\item Suppose $p$ and $q$ are distinct primes. Prove that $\ZZ_{pq}$ contains exactly $(p-1)(q-1)$
units.
\begin{quote}
We can consider the set of non-units of $\ZZ_{pq}$, which takes the form
\[\{0,p,q,\dots,np,mq\},\ n<p,\ m<q.\]
If we can determine the size of the set of non-units, then we can find the size of the units. We know
that the non-units contain all multiples of $p$, $np$ for some $n\in\ZZ$, and the set also contains
all multiples of $q$, $mq$ for some $m\in\ZZ$. We know we have up to $p$ multiples of $p$, and $q$
multiples of $q$, so there are $q+p$ multiples in total. However, when $n=m$, we have a duplicate
multiple, so the set of non-units contains $q+p-1$ elements. If we subtract the total size of the
set from the size of the set of non-units, we have $pq-p-q+1=(p-1)(q-1)$.
% Suppose $a$ is a unit of $\ZZ_{pq}$. We have that $ax\equiv1mod(pq)$, so $pq\mid(ax-1)$. By definition
% of prime division, we have that $p\mid(ax-1)$ and $q\mid(ax-1)$. It follows that $ax\equiv1\mod p$,
% so $gcd(a,p)=1$ for all $p\in\ZZ_{p}$. This means that every element of $\ZZ_{p}$ is a unit, so there
% are $p-1$ units. Similarly, we have that there are $q-1$ units in $\ZZ_{q}$. Therefore there are
% $(p-1)(q-1)$ units in $\ZZ_{pq}$.
\end{quote}
\end{enumerate}
\end{document}
