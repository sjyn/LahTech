\documentclass{article}
\usepackage{amsmath}
\usepackage{mathtools}
\usepackage{graphicx}
\usepackage{multicol}
\usepackage{array}
\usepackage{amssymb}
\topmargin=-0.3in
\textheight=9.2in
\textwidth=168mm
\oddsidemargin=-0.2in
\evensidemargin=-0.2in

\newcommand{\deff}[1]{\textbf{\textit{\underline{#1}}}}
\newcommand{\NN}{\mathbb{N}}
\newcommand{\ZZ}{\mathbb{Z}}
\newcommand{\QQ}{\mathbb{Q}}
\newcommand{\RR}{\mathbb{R}}
\newcommand{\PS}{\mathcal{P}}
\newcommand{\US}{\mathcal{U}}
\newcommand{\rel}{\mathcal{R}}

\newcommand{\seta}{\mathbb{A}}
\newcommand{\setb}{\mathbb{B}}
\newcommand{\setc}{\mathbb{C}}
\newcommand{\setd}{\mathbb{D}}

\title{Proofs Definitions}
\date{}
\begin{document}
\maketitle
\begin{enumerate}
\item \deff{Set:} A set is a collection of objects.
The items in a set are called elements.
\item \deff{The Empty Set:} The empty set, given by $\emptyset$, is the set with
no elements.
\item \deff{Subset:} Let $\seta$ and $\setb$ be sets. Then $\seta\subset\setb$ if
$\forall x\in\seta$, $x\in\setb$.
\item \deff{Power Set:} Let $\seta$ be a set. Then the Power Set of $\seta$,
noted as $\PS(\seta)$, is the set of all subsets of $\seta$.
\item \deff{The Universal Set:} The universal set $\US$ is the set of which all
sets are subsets.
\item \deff{Intersection:} Let $\seta$ and $\setb$ be sets. The intersection of
$\seta$ and $\setb$, given by $\seta\cap\setb$ is
$\{x|x\in\seta\land x\in\setb\}$.
\item \deff{Union:} Let $\seta$ and $\setb$ be sets. The union of
$\seta$ and $\setb$, given by $\seta\cup\setb$ is
$\{x|x\in\seta\lor x\in\setb\}$.
\item \deff{Trivial Intersection (Disjoint):} Let $\seta$ and $\setb$ be sets. If $\seta$
and $\setb$ have a trivial intersection, then $\seta\cap\setb = \emptyset$.
\item \deff{Set Difference:} Let $\seta$ and $\setb$ be sets. The set difference
of $\seta$ and $\setb$, noted as $\seta - \setb$, is
$\{x|x\in\seta\land x\notin\setb\}$.
\item \deff{Cartesian Product:} Let $\seta$ and $\setb$ be sets. Then the product
of $\seta$ and $\setb$, given by $\seta\times\setb$, is
$\{(a,b)|a\in\seta\land b\in\setb\}$.
\item \deff{Well Ordered:} A set $\seta$ is well ordered if for every non-empty
set $\setb\subset\seta$, $\setb$ has a least element.
\item \deff{Compliment:} The compliment of a set $\seta$ in regards to $\US$ is
the set difference of $\US$ and $\seta$.
\item \deff{Negation:} The negation of a statement $\rho$, given by $\neg\rho$,
is the statement that has the opposite truth values of $\rho$.
\item \deff{Disjunction:} The disjunction of statements $\rho$ and $\varphi$ is
the statement where either $\rho$ and $\varphi$ is true, or both are true, given
by $\rho\lor\varphi$.
\item \deff{Conjunction:} The conjunction of statements $\rho$ and $\varphi$ is
the statement where $\rho$ and $\varphi$ are both true, given by $\rho\land\varphi$.
\item \deff{Implication:} The implication of $\rho$ and $\varphi$, noted as
$\rho\implies\varphi$, is the statement where if $\rho$ then $\varphi$ is true.
\begin{center}
\begin{tabular}{c|c|c|c|c}
$\rho$ & $\varphi$ & $\rho\land\varphi$ & $\rho\lor\varphi$ & $\rho\implies\varphi$\\
\hline
0 & 0 & 0 & 0 & 1\\
0 & 1 & 0 & 1 & 1\\
1 & 0 & 0 & 1 & 0\\
1 & 1 & 1 & 1 & 1
\end{tabular}
\end{center}
\item \deff{Logically Equivalent:} Two statements $\rho$ and $\varphi$ are
logically equivalent, noted by $\rho\equiv\varphi$, if they have the same truth
values.
\item \deff{Tautology:} A statement $\rho$ is a tautology if it is true for all
possible truth values.
\item \deff{Contradiction:} A statement $\rho$ is a contradiction if it is false
for all possible truth values.
\item \deff{Universal Quantifier:} The universal quantifier, $\forall$, is a
quantifier that asserts that a given statement $\rho$ holds for all elements in
the specified domain.
\item \deff{Existential Quantifier:} The existential quantifier, $\exists$, is a
quantifier that assets that a given statement $\rho$ holds for at least one
element in the specified domain.
\item \deff{Axiom:} An axiom is a statement that is accepted as true without
proof.
\item \deff{Theorem:} A theorem is a statement that is proved to be true.
\item \deff{Lemma:} A lemma is a statement that serves as an intermediate step
in a proof.
\item \deff{Corollary:} A corollary is a statement that follows from an eariler
result.
\item \deff{Vacuous Statement:} A vacuous statement is a statement in which the
assumption is always false. The statement is always true.
\item \deff{Division:} Let $a,b\in\RR$. To say $a$ divides $b$, noted as $a|b$,
implies that $b=ax$ for some $x\in\ZZ$.
\item \deff{Induction:} For all $n\in\NN$, let $\rho(n)$ be a statement. If
$\rho_{0}$ is true and $\rho(n)\implies\rho(n + 1)$, then $\rho(n)$ is true for
all $n$.
\item \deff{Strong Induction:} For all $n\in\NN$, let $\rho(n)$ be a statement.
If $\rho_{0}$ is true and $\rho(i)\implies\rho(n+1)$ for all $i\in\NN$, then
$\rho(n)$ is true for all $n\in\NN$.
\item \deff{Relation:} A relation $\rel$ from set $\seta$ to set $\setb$ is a
subset of $\seta\times\setb$. Set $\seta$ is related to set $\setb$, noted as
$\seta\rel\setb$ if $(a,b)\in\rel$ for $a\in\seta$ and $b\in\setb$.
\item \deff{Inverse Relation:} Given a relation $\rel$ from $\seta$ to $\setb$,
$\rel$ inverse, noted as $\rel^{-1}$ or $\rel^{\text{opp}}$, is
$\{(b,a)|(a,b)\in\rel\}$.
\item \deff{Reflixivity:} A relation $\rel$ is reflexive on a set $\seta$ if
$a\rel a$ for all $a\in\seta$.
\item \deff{Symmetry:} A relation $\rel$ is symmetric on a set $\seta$ if
$a\rel\eta\implies\eta\rel a$ for all $a,\eta\in\rel$.
\item \deff{Transitivity:} A relation $\rel$ is transitive on a set $\seta$ if
$a\rel\eta$ and $\eta\rel\delta\implies a\rel\delta$ for $a,\eta,\delta\in\seta$.
\item \deff{Equivalence Relation:} A relation $\rel$ is an equivalence relation
if it is reflexive, symmetric, and transitive.
\item \deff{Equivalence Class:} Let $\seta$ be a non-empty set with elements
$a$ and $\eta$. The equivalence class of $a$, noted as $[a]$, is
$\{\eta|\eta\rel a\}$.
\item \deff{Partition:} Let $\seta$ be a non-empty set. A patition of $\seta$,
given by $\mathbb{P}$, is a set of subsets of $\seta$ where
$\cup_{i\in\mathbb{I}}\seta = \mathbb{P}$ and $\seta_{i}\cap\seta_{j}\neq\emptyset$
for all $i\neq j$.
\item \deff{Function:} A function $f$ from $\seta\to\setb$ is a relation from
$\seta\to\setb$ where if $(a,b)$ and $(a,c)$ are in $f$, then $b=c$, and
$\forall x\in\seta,\exists y\in\setb$ such that $(x,y)\in f$.
\item \deff{Image:} The image of a function $f:\seta\to\setb$ is
$\{f(a)|a\in\seta\}$.
\item \deff{Preimage:} Let $f$ be a function such that $f:\seta\to\setb$ and
$\setd\subset\setb$. Then the preimage of $\setd$, noted as $f^{-1}(\setd)$, is
$\{a\in\seta|f(a)\in\setd\}$.
\item \deff{Injection:} A function $f:\seta\to\setb$ is injective if
$f(a) = f(b)\implies a = b$ for all $a,b\in\seta$.
\item \deff{Surjection:} A function $f:\seta\to\setb$ is surjective if
$\forall b\in\setb, \exists a\in\seta$ such that $f(a) = b$.
\item \deff{Bijection:} A function $f:\seta\to\setb$ is bijective if it is
injective and surjective.
\item \deff{Composition:} Let $f:\seta\to\setb$ and $g:\setb\to\setd$ be
functions. Then the composition of $g$ and $f$, given by $g\circ f$, is
defined as $(g\circ f)(a) = g(f(a))$. 
\end{enumerate}
\end{document}
