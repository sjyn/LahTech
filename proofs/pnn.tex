\documentclass{article}

\usepackage{amssymb}
\usepackage{amsmath}
\usepackage{minted}

\topmargin=-0.3in
\textheight=9.2in
\textwidth=168mm
\oddsidemargin=-0.2in
\evensidemargin=-0.2in

\newcommand{\ZZ}{\mathbb{Z}}
\newcommand{\RR}{\mathbb{R}}
\newcommand{\QQ}{\mathbb{Q}}

\pagenumbering{gobble}

\title{3.0 Is Not An Integer}
\date{}
\begin{document}
\maketitle
\newpage

In the following document, we will show that 3.0 is not an integer, or $3.0 \notin \ZZ$. This result
stems from the fundametal idea, which we will prove, that $0.\overline{99} = 1.0$. The notation
$0.\overline{99}$ is a symbolic way to represent $0.99999...$ where the $9$ repeats \textbf{infinitely}.
It is clear that we can accept $0.\overline{99}$ as being an element of the real numbers $\RR$ and not an
element of the set of integers $\ZZ$ or rationals $\QQ$. We will now show that $0.\overline{99} = 1.0$.\\

\noindent\textbf{\textit{Theorem:}} $0.\overline{99} = 1.0$.\\\\
\indent\textbf{\textit{Proof:}} We can represent the number $0.\overline{99}$ as a geometric series
represented as
\begin{align*}
0.\overline{99} &= \frac{9}{10} + \frac{9}{100} + \frac{9}{1000} + \frac{9}{10000} + \cdots\\
&= \frac{9}{10} + \left(\frac{9}{10}\right)\left(\frac{1}{10}\right) +
\left(\frac{9}{10}\right)\left(\frac{1}{100}\right) +
\left(\frac{9}{10}\right)\left(\frac{1}{1000}\right) + \cdots\\
&= \frac{9}{10} + \left(\frac{9}{10}\right)\left(\frac{1}{10}\right)^{1} +
\left(\frac{9}{10}\right)\left(\frac{1}{10}\right)^{2} +
\left(\frac{9}{10}\right)\left(\frac{1}{10}\right)^{3} + \cdots\\
&= \sum_{n=1}^{\infty}\left(\frac{9}{10}\right)\left(\frac{1}{10}\right)^{n-1}.
\end{align*}
The general form for a geometric series is given by
\[
\sum_{n=1}^{\infty}ar^{n-1},
\]
Which leads us to assign the common ratio $r = \frac{1}{10}$ and $a = \frac{9}{10}$. By definition,
a geometric series converges to $\frac{a}{1 - r}$ if $|r| < 1$. This leads us to
\begin{align*}
0.\overline{99} &= \frac{\frac{9}{10}}{1 - \frac{1}{10}}\\
&= \frac{9}{10 -1}\\
&= \frac{9}{9}\\
&= 1.
\end{align*}
$\triangle$
\noindent\\

Now we can extend our argument to the original claim, that $3.0\notin\ZZ$.\\
\noindent\textbf{\textit{Theorem:}} $3.0\notin\ZZ$.\\\\
\indent\textbf{\textit{Proof:}} Assume that $3.0\in\ZZ$. Then $2.0 + 1 \in\ZZ$. We know as a fundamental
truth that if $x\in\RR$ and $y\notin\ZZ$, then $x + y\notin\ZZ$. We have already shown that
$1 = 0.\overline{99}$, so it follows that $2.0 + 1 = 2.0 + 0.\overline{99}$. $0.\overline{99}$ is not an
integer, and therefore $3.0 = 2.0 + 0.\overline{99}\notin\ZZ$.\\
$\triangle$
\newpage

Before we conclude, it should also be noted that from a practical standpoint, 3.0 cannot be considered
an integer. In this day and age, the study of programming is becoming a more and more necessary
skill for people to learn. Programming languages such as Java, C++, Ruby, and countless other
languages that are widely used would never assert that $3.0\in\ZZ$. For example, consider the following
Ruby code:
\begin{minted}{ruby}
def main
    if 3.0 == 3 then
        puts "3.0 = 3"
    else
        puts "3.0 != 3"
end
\end{minted}
This code will execute the else block of code. In the following Java code, the compiler would
automatically cast the value 3.0 to 3, since it was explicitly declared as an integer value:
\begin{minted}{java}
public class ThreePointOh {
    public static void main(String[] args){
        int tpo = 3.0;
        System.out.println(tpo); //Outputs 3, not 3.0
    }
}
\end{minted}
It is simply not practical to continue under the assumption that $3.0\in\ZZ$.
\end{document}
