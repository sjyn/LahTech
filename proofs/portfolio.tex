\documentclass{article}
\usepackage{amssymb,amsmath}
\usepackage{mathtools}
\usepackage{graphicx}
\usepackage{multicol}
\usepackage{array}
\topmargin=-0.3in
\textheight=9.2in
\textwidth=168mm
\oddsidemargin=-0.2in
\evensidemargin=-0.2in

\newcommand{\Proof}{\textit{\textbf{Proof: }}}
\newcommand{\NN}{\mathbb{N}}
\newcommand{\ZZ}{\mathbb{Z}}
\newcommand{\QQ}{\mathbb{Q}}

\begin{document}
\noindent\textit{\textbf{Steven Rosendahl}}\\
\textit{\textbf{Proofs Homework}}
\begin{enumerate}
\item
\begin{enumerate}
%%%QUESTION 1%%%
\item Let $a,b$ and $c$ be natural numbers with $a$ odd.
Prove that if $a|(b -c)$ and $a|(b+c)$, then $a|b$ and $a|c$.
\begin{quote}
\indent \Proof Let $a|(b-c)$ and $a|(b + c)$.
Then $b-c = ax$ and $b + c = ay$ for some $x,y\in \NN$.
We have that $b = ax + c$.
Then 
\begin{gather*}
ax + c + c = ay\\
ax + 2c = ay\\
2c = ay - ax\\
2c = a(y-x).
\end{gather*}
Since $a$ is odd, $y-x$ must be even since $2c$ is even.
Then $2 | (y-x)$, and we have $c = a\frac{y-x}{2}$, or $c = az$ for $z\in\NN$.
Therefore $a | c$.
If we let $c = b-ax$, we have
\begin{gather*}
b + b - ax = ay\\
2b - ax = ay\\
2b = ay + ax\\
2b = a(y + x)
\end{gather*}
Since $2b$ is even and $a$ is odd, $y-x$ must be even, or $2 | (y+x)$.
Therefore $b = a\frac{y+x}{2}$, or $b = aj$ for $j\in\NN$.
Therefore $a | b$.
\end{quote}
$\triangle$
\item Using a truth table, show that $\neg(P\wedge Q)$ and $(\neg P \vee \neg Q)$ are logically equivalent.\\
\begin{quote}
\begin{center}
\begin{tabular}{c | c | c | c}
$P$ & $Q$ & $\neg(P\wedge Q)$ & $(\neg P \vee \neg Q)$\\
\hline
0 & 0 & 1 & 1\\
0 & 1 & 1 & 1\\
1 & 0 & 1 & 1\\
1 & 1 & 0 & 0
\end{tabular}
\end{center}
\end{quote}
\end{enumerate}
\newpage
%%%QUESTION 2%%%
\item
\begin{enumerate}
\item Establish the following identity using induction.
\[
\sum_{i = 1}^{n}i^{3} = \left(\frac{n(n+1)}{2}\right)^{2}
\]
\begin{quote}
\Proof \\
\textit{\textbf{Base Case: }} $n = 1$
\begin{align*}
\sum_{i = 1}^{1}i^{3} &= \left(\frac{1(1+1)}{2}\right)^{2}\\
1 ^{3} &= \left(\frac{2}{2}\right)^{2}\\
1 &= 1
\end{align*}
\textit{\textbf{Assume: }} $$\sum_{i = 1}^{n}i^{3} = \left(\frac{n(n+1)}{2}\right)^{2}$$
\textit{\textbf{Prove: }}  
\[
\sum_{i = 1}^{n + 1}i^{3} = \left(\frac{(n + 1)(n+2)}{2}\right)^{2}
\]
\begin{align*}
\therefore \sum_{i = 1}^{n}i^{3} + \sum_{i = n + 1}^{n + 1}i^{3} &= \left(\frac{n(n+1)}{2}\right)^{2} + (n + 1)^{3}\ \ \text{By the Induction Hypothesis}\\
&= \frac{n^{2}(n+1)^{2}}{4} + (n+1)^{3}\\
&= \frac{n^{2}(n+1)^{2}}{4} + \frac{4(n+1)^{3}}{4}\\
&= \frac{n^{2}(n+1)^{2} + 4(n+1)^{3}}{4}\\
&= \frac{(n+1)^{2}(n^{2} + 4n + 4)}{4}\\
&= \frac{(n+1)^{2}(n+2)^{2}}{2^{2}}\\
&= \left(\frac{(n+1)(n+2)}{2}\right)^{2}
\end{align*}
\end{quote}
$\triangle$
\item Prove that if $n^{3}$ is odd, then $n$ is odd.
\begin{quote}
\Proof Assume the contrapositive: if $n$ is even, then $n^{3}$ is even.
Then $n = 2k$ for $k\in \ZZ$, which mean that $n^{3} = (2k)^{3}$.
$(2k)^{3} = 2(2^{2}k^{3})$ where $(2^{2}k^{3}) \in \ZZ$. Therefore $n^{3}$ is even.
\end{quote}
$\triangle$
\end{enumerate}
\newpage
%%%QUESTION 3%%%
\item
\begin{enumerate}
\item Using the definition, prove that $f: \QQ \to \QQ$ given by $f(x) = 3x+2$ is bijective.
\begin{quote}
\Proof Let $x,y \in \QQ$ such that $f(x) = f(y)$. Then
\begin{align*}
3x + 2 &= 3y + 2\\
3x &= 3y\\
x &= y
\end{align*}
Therefore $f$ is injective.\\
Let $y\in \QQ$ such that $y = \frac{x-2}{3}$. Then
\begin{align*}
f(y) &= 3\left(\frac{x-2}{3}\right) + 2\\
&= x - 2 + 2\\
&= x
\end{align*}
Therefore $f$ is surjective.\\
Therefore $f$ is bijective.
\end{quote}
$\triangle$
\item Define a relation on $\NN \times \NN$ by $(a,b) \sim (c,d)$ if $a + d = b + c$. 
Prove that $\sim$ is an equivalence relation.
\begin{quote}
\textbf{\textit{Symmetric:}} Let $(a,b) \sim (c,d)$. Then
\begin{align*}
a + d &= b + c\\
-b - c &= -a - d\\
(-1)(b + c) &= (-1)(a + d)\\
b + c &= a + d
\end{align*}
Therefore $(c,d) \sim (a,b)$, and $\sim$ is symmetric.\\\\
\textbf{\textit{Reflexive:}} Let $(a,b) \in\NN\times\NN$. 
If $(a,b) \sim (a,b)$, then $a + b = a + b$.
Therefore $(a,b) \sim (a,b)$, and $\sim$ is reflexive.\\\\
\textbf{\textit{Transitive:}} Let $(a,b) \sim (c,d)$ and $(c,d) \sim (e,f)$.
Then $a + d = b + c$ and $c + f = d + e$. 
Solving for $c$ yields $c = d + e - f$, and substituting gives us $a + d = b + d + e - f$.
Then $a + f = e + b$, and $(a,b) \sim (e,f)$.
Therefore, $\sim$ is transitive.\\\\
Therefore $\sim$ is an equivalence relation.
\end{quote}
$\triangle$
\end{enumerate}
%%%QUESTION 4%%%
\newpage
\item
\begin{enumerate}
\item Let $\mathcal{X}$ be a finite set with cardinality $n$.
Prove that the power set, $\mathcal{P}(\mathcal{X})$, has cardinality $2^{n}$.
\begin{quote}
\Proof\\
\textit{\textbf{Base Case:}} A set of size 1.\\
Let $\mathcal{X}$ be the set $\{x\}$.
Then $\mathcal{P}(\mathcal{X})$ is $\{\emptyset,\{x\}\}$, which has a cardinality of $2^{||\mathcal{X}||}$, or $2 ^{1}$.\\\\
\textit{\textbf{Assume:}} $||\mathcal{P}(\{x_{0}, x_{1}, x_{2}, \ldots, x_{n}\})|| = 2^{||\mathcal{X}||}$. \\\\
\textit{\textbf{Prove:}} $||\mathcal{P}(\{x_{0}, x_{1}, x_{2}, \ldots,x_{n}, x_{n+1}\})|| = 2^{||\mathcal{X}|| + 1}$.\\
We know that $\mathcal{P}$ is the set of all subsets of $\mathcal{X}$. 
If we count the number of subsets of $\{x_{0}, x_{1}, x_{2},\ldots,x_{n},x_{n+1}\}$, we know that the subset will either contain $x_{n+1}$, or it will not contain $x_{n+1}$.
If the subset $\gamma$ does not contain $x_{n+1}$, then $\gamma \subseteq \{x_{0},x_{1},x_{2},\ldots,x_{n}\}$, and there are $2^{||\mathcal{X}||}$ $\gamma$ by the induction hypothesis.
If the subset $\lambda$ contains $x_{n+1}$, then it is the result of some set $\gamma \cup \lambda$.
Since $\gamma \subseteq\{x_{0},x_{1},x_{2},\ldots,x_{n}\}$, we only need $\gamma \cup \{x_{n+1}\}$ to account for all possible sets.
Therefore $||\mathcal{P}(\gamma \cup \{x_{n+1}\})||$ is $||\mathcal{P}(\gamma)|| \cdot ||\mathcal{P}(\{x_{n+1}\})||$, or $2^{||\mathcal{X}||} \cdot 2^{||\{x_{n+1}\}||}$.
This is equivalent to $2^{||\mathcal{X}||}\cdot2^{1}$, or $2^{||\mathcal{X}|| + 1}$.
\end{quote}
$\triangle$
\item Let $n \geq 2$ be an integer.
Prove that $a \equiv b(\text{mod}\ n)$ is an equivalence relation on $\ZZ$.
\begin{quote}
Let $R$ be the relation $a\equiv b\ mod\ n$.\\
\textit{\textbf{Symmetric:}} Let $aRb$.
Then $a \equiv b(\text{mod}\ n)$, or $a - b = nk$ for $k\in\ZZ$.
It follows that
\begin{align*}
a &= nk + b\\
-nk &= b-a\\
nj &= b-a,\in\ZZ\\
\end{align*}
Therefore $b \equiv a\ mod\ n$, and $R$ is symmetric.\\\\
\textit{\textbf{Reflexive:}} Let $aRa$.
Then $a \equiv a\ mod\ n$.
It follows that $n|(a-a)$, or $n|0$.
Since $n \geq 2$, $n|0$, and $R$ is reflexive.\\\\
\textit{\textbf{Transitive:}} Let $aRb$ and $bRc$.
Then $a\equiv b\ mod\ n$ and $b\equiv c\ mod\ n$.
By definition, $a-b = nk,\ k\in\ZZ$ and $b - c = nj,\ j\in\ZZ$.
Then $a-b = a-nj-c = nk$.
It follows that $a-c = nk + nj$, or $a-c = n(k + j)$.
Therefore $a\equiv b\ mod\ n$, and $R$ is transitive\\\\
Therefore $R$ is an equivalence relation.
%\Proof Let $n \geq 2$ and $n | (b - a)$.
%Therefore $b - a  = nk,\ k\in\ZZ$.
%Then $b - a \equiv 0\ mod\ n$.
%It follows that $b \equiv a$, and since $mod\ n$ is an equivalence relation, $a \equiv b$ by symmetry.
\end{quote}
$\triangle$
\item Let $A$ and $B$ be sets.
Prove that $\overline{A \cup B} = \overline{A} \cap \overline{B}$.
\begin{quote}
\Proof\\ 
{$\overline{A \cup B} \subseteq \overline{A} \cap \overline{B}$:} 
Let $x\in\overline{A \cup B}$.
Then $x \notin A$ or $B$.
Since $x \notin A$, $x \in \overline{A}$.
Since $x \notin B$, $x \in \overline{B}$.
Therefore $x \in \overline{A}$ and $x\in\overline{B}$, or $x \in \overline{A} \cap \overline{B}$.\\\\
{$\overline{A \cup B} \supseteq \overline{A} \cap \overline{B}$:} 
Let $x\in\overline{A} \cap \overline{B}$.
Then $x \notin A$ and $x \notin B$.
Therefore $x \notin A \cup B$, or $x \in \overline{A \cup B}$.
\end{quote}
$\triangle$
\item Prove that $\sqrt{5}$ is irrational.
\begin{quote}
\Proof Let $p$ be a prime number, and assume that $\sqrt{p}$ is rational.
Then $\sqrt{p} = \frac{n}{m}$ for $n,m \in \NN$.
It follows that $n^{2} p = m^{2}$.
We know that there are two factors of $p$, namely 1 and $p$, and that a squared number will have an even number of prime factors, since it has double the prime factors as its root.
Then $n^{2}p$ will have an odd number of prime factors, since its prime factors are the prime factors of $n^{2}$ and the number $p$.
Since $n^{2}p = m^{2}$, $m^{2}$ must also have an odd number of prime factors.
However, $m^{2}$ has an even number of prime factors.
Therefore, by contradiction, the root of a prime number is irrational.
Therefore, since 5 is prime, $\sqrt{5}$ is irrational. $\triangle$
\end{quote}
\end{enumerate}
\end{enumerate}
\end{document}