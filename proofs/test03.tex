\documentclass{article}
\usepackage{amsmath, amssymb}
\usepackage{cancel}
\topmargin=-0.3in
\textheight=9.2in
\textwidth=168mm
\oddsidemargin=-0.2in
\evensidemargin=-0.2in
\pagenumbering{gobble}

\def\labelitemi{-}
\newcommand{\SLN}{\indent\textit{\textbf{Solution: }}}
\newcommand{\ZZ}{\mathbb{Z}}
\newcommand{\RR}{\mathbb{R}}
\newcommand{\QQ}{\mathbb{Q}}

\begin{document}
\noindent Steven Rosendahl\\
Proofs Exam 3
\vspace{1cm}
\begin{enumerate}
%%%QUESTION 1%%%
\item Prove that for all positive integers $N$,
\[
\sum_{k=1}^{n}\frac{1}{(k+2)(k+3)} = \frac{n}{3n + 9}
\]
\begin{quote}
\SLN
\noindent We first need to show a base case. If we let $n = 1$, we have
\begin{align*}
\sum_{k=1}^{1}\frac{1}{(k+2)(k+3)} &= \frac{1}{3(1) + 9}\\
\frac{1}{(1 + 2)(1 + 3)} &= \frac{1}{12}\\
\frac{1}{12} &= \frac{1}{12}.
\end{align*}
\noindent Now we can form our induction hypothesis, given by
\[\sum_{k=1}^{n}\frac{1}{(k+2)(k+3)} = \frac{n}{3n + 9}.\]
\noindent Now we need to prove:
\[
\sum_{k=1}^{n + 1}\frac{1}{(k+2)(k+3)} = \frac{n + 1}{3(n + 1) + 9}
\]
\begin{align*}
\therefore\ \sum_{k=1}^{n + 1}\frac{1}{(k+2)(k+3)} &= \sum_{k=1}^{n}\frac{1}{(k+2)(k+3)} + \sum_{k=n+1}^{n+1}\frac{1}{(k+2)(k+3)}\\
\text{By the induction hypothesis, }&= \frac{n}{3n+9} + \frac{1}{(n+3)(n+4)}\\
&= \frac{n(n+4) + 3}{3(n+3)(n+4)}\\
&= \frac{n^{2} + 4n + 3}{3(n+3)(n+4)}\\
&= \frac{(n+3)(n+1)}{3(n+3)(n+4)}\\
&= \frac{n+1}{3(n+4)}\\
&= \frac{n+1}{3(n + 1 + 3)}\\
&= \frac{n+1}{3(n+1) + 9}
\end{align*}
$\triangle$
\end{quote}
\newpage
%%%QUESTION 2%%%
\item Let $R$ be a relation on $\{1,2,3,4,5,6\}$ be defined by $xRy$ if and only if $x + y \leq 8$. Determine which of the following properties $R$ has: reflexive, symmetric, transitive. Give a proof for each property it satisfies, and give a counterexample for each of those properties it does not have.
\begin{quote}
\textit{\textbf{Reflexive: }} Let $x \in \{1,2,3,4,5,6\}$.
Then $xRx$ means that $x + x \leq 8$.
However $6 + 6 = 12 \nleq 8$. Therefore, $R$ is not reflexive.\\\\
\textit{\textbf{Symmetric: }} Let $x,y \in \{1,2,3,4,5,6\}$ such that $xRy$.
Then $x + y \leq 8$.
Since addition is commutative, $x + y = y + x$.
Then $x + y = y + x \leq 8$.
Therefore $yRx$, which means $R$ is symmetric.\\\\
\textit{\textbf{Transitive: }} Let $a,b,c \in  \{1,2,3,4,5,6\}$ such that $aRb$ and $bRc$.
Then $a + b \leq 8$, and $b + c \leq 8$.
However, if we let $a = 5$, $b = 2$, and $c = 6$, then $aRb$, since $5 + 2 = 7 \leq 8$, and $bRc$,
since $2 + 6 = 8 \leq 8$, but $a\cancel{R}c$, since $5 + 6 \nleq 8$.
\end{quote}
$\triangle$
%%%QUESTION 3%%%
\item Assume relation $R$ on $A = \{1, 2, 3, 4\}$ is an equivalence relation where
\[R = \{(1, 1), (2, 2), (\ \ ,\ \ ), (4, 4), (1, 2), (2, 4), (\ \ ,\ \ ), (\ \ ,\ \ ), (4, 2), (4, 1)\}.\]
Note that $R$ has exactly 3 blank entries.
\begin{enumerate}
\item Fill in the 3 missing entries of $R$.
\begin{quote}
\[R = \{(1, 1), (2, 2), (3, 3), (4, 4), (1, 2), (2, 4), (2, 1), (1, 4), (4, 2), (4, 1)\}.\]
\end{quote}
\item Clearly write all of the equivalence classes of $R$.
\begin{quote}
\begin{align*}
[1] &= [2] = [3] = \{1, 2, 4\}\\
[3] &= \{3\}
\end{align*}
\end{quote}
\item Write the partition associated with $R$.
\begin{quote}
\[P = \{[1],[3]\} = \{[2],[3]\} = \{[4],[3]\}\]
\end{quote}
\end{enumerate}
%%%QUESTION 4%%%
\item  Let $R$ be a relation from $A=\{1,2,3,4,5\}$ to $B=\{a,b,c,d\}$ given by
\[R=\{(1,d),(2,c),(3,a),(4,d),(5,b)\}.\]
\begin{enumerate}
\item Why is $R$ a function from $A$ to $B$?
\begin{quote}
$R$ is a function since every element in $A$ maps to an element in $B$, and no one element
in $A$ maps to two elements in $B$ at the same time.
\end{quote}
\item Is $R$ an injection from $A$ to $B$? Why?
\begin{quote}
$R$ is not an injection since both 1 and 4 map to $d$.
\end{quote}
\item Is $R$ a surjection from $A$ to $B$? Why?
\begin{quote}
$R$ is a surjection since every value in the codomain $B$ has a pre-image in $A$.
\end{quote}
\end{enumerate}
%%%QUESTION 5%%%
\newpage
\item
\begin{enumerate}
\item Show that the function $f:\ZZ \to \ZZ$ defined by $f(x)=2x+1$ is an injection.
\begin{quote}
\SLN Let $x,y \in \ZZ$ such that $f(x) = f(y)$.
Then $2x + 1 = 2y + 1$, or $2x = 2y$.
Therefore, $x = y$, which means $f$ is injective.
\end{quote}
$\triangle$
\item Show that the function $g:\RR \to \RR$ defined by $g(x) = 7x - 1$ is an surjection.
\begin{quote}
\SLN Let $y \in \RR$. Then $y = 7x - 1$, or $x = \frac{y + 1}{7}$.
We know this is in $\RR$ since it is either in $\QQ, \ZZ,\ \text{or}\  \RR$.
Substituting $x$ into $f$ gives us
\begin{align*}
f\left(\frac{y + 1}{7}\right) &= 7\left(\frac{y + 1}{7}\right) - 1\\
&= y + 1 - 1\\
&= y
\end{align*}
Therefore, $f$ is surjective.
\end{quote}
$\triangle$
\end{enumerate}
%%%QUESTION 6%%%
\item Let $f: \RR - \{1\} \to \RR - \{3\}$ be defined by $f(x) = \frac{3x}{x - 1}$.
\begin{enumerate}
\item Show $f(x)$ is a bijection.
\begin{quote}
\textit{\textbf{Injection: }} Let $x,y \in \RR - \{1\}$ such that $f(x) = f(y)$.
Then
\begin{align*}
\frac{3x}{x - 1} &= \frac{3y}{y - 1}\\
(3x)(y - 1) &= (3y)(x - 1)\\
3xy - 3x &= 3xy - 3y\\
-3x &= -3y\\
x &= y.
\end{align*}
Therefore, $f$ is an injection.\\\\
\textit{\textbf{Surjection: }} Let $y \in \RR - \{3\}$.
Then $y = \frac{3x}{x - 1}$, or $x = -\frac{y}{3-y}$, which is an element of $\RR - \{3\}$.
Substituting back into $f$ gives us
\begin{align*}
f\left(-\frac{y}{3-y}\right) &= \frac{3\left(-\frac{y}{3-y}\right)}{\left(-\frac{y}{3-y}\right) - 1}\\
&= \frac{-3y}{-y-(3-y)}\\
&= \frac{-3y}{-y - 3 + y}\\
&= \frac{-3y}{-3}\\
&= y.
\end{align*}
Therefore $f$ is a surjection.
\end{quote}
Therefore $f$ is a bijection.\\
$\triangle$
\item Find the inverse of $f(x)$.
\begin{quote}
Let $g(x) = -\frac{x}{3 - x}$.
Then
\hspace{-0.25\textwidth}
\begin{minipage}[t]{0.4\textwidth}
\begin{align*}
f\circ g(x) &= f(g(x))\\
&= f\left(-\frac{x}{3 - x}\right)\\
&= \frac{3\left(-\frac{x}{3-x}\right)}{\left(-\frac{x}{3-x}\right) - 1}\\
&= \frac{-3x}{-x-(3-x)}\\
&= \frac{-3x}{-x - 3 + x}\\
&= \frac{-3x}{-3}\\
&= x
\end{align*}
\end{minipage}
\begin{minipage}[t]{0.4\textwidth}
\begin{align*}
g \circ f(x) &= g(f(x))\\
&= g(\frac{3x}{x-1})\\
&= -\frac{\frac{3x}{x-1}}{3-\frac{3x}{x-1}}\\
&= \frac{-3x}{3(x-1)-3x}\\
&= \frac{-3x}{3x - 3 - 3x}\\
&= \frac{-3x}{-3}\\
&= x
\end{align*}
\end{minipage}
Therefore, $g(x) = f^{-1}(x)$.
\end{quote}
$\triangle$
\end{enumerate}
\end{enumerate}
\end{document}
