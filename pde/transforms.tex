\section{Transforms}
\hrule
\noindent \\\\
\indent Much like when we solve ODE's, we can use transforms to solve PDE's. When we use a Laplace
transform to solve an ODE, we are transforming the ODE into an algebraic equation that is much
easier to solve. When we transform a PDE, we get an ODE that we can solve using techniques that
we already know. In this section, we will look at two transforms: the Laplace Transform and the
Fourier Transform.\\
\subsection{The Laplace Transform}
\indent Let's begin by looking at an exaple use of a Laplace Transform to solve an ODE.\\\\
\noindent \textbf{\textit{Ex:}} Solve the following ODE using a Laplace Transform.
\[
\begin{cases*}
y'' - 5y' + 6y = 0\\
y(0) = 2\\
y'(0) = 2
\end{cases*}
\]
\indent \textbf{\textit{Solution:}} We should begin by recalling the definition of the Laplace operator:
\[
\Lap[f(t)] =\int_{0}^{\infty}f(t)e^{-st}dt
\]
We also know that the Laplace transforms of a derivative and a second derivative:
\[
\Lap[f'(t)] = s\Lap[f(t)] - f(0)\qquad\qquad \Lap[f''(t)] = s^{2}\Lap[f(t)]-sf(0)-f'(0)
\]
This is really all the information we need. Applying the transform to both sides of the equation yields
\begin{gather*}
\Lap[y'' - 5y' + 6y ] = \Lap[0]\\
s^{2}\gamma(s)-sy(0)-y'(0) - s\gamma(s)-y(0) = 0\\
s^{2}\gamma(s)-2s-2-s\gamma(s)-2 = 0\\
s^{2}\gamma(s)-s\gamma(s)-2s-4 = 0\\
\gamma(s) = \frac{2s - 8}{s^{2} -5s + 6}\\
\gamma(s) = \frac{4}{s-2}-\frac{2}{s-3}
\end{gather*}
We now need to transform our equation back to its ODE form; we do in fact have an inverse Laplace
operator, but it involves complex variables. If we consult a table of transforms, we will see that
$\gamma(s)$ becomes
\[
y(t) = 4e^{2t} +- 2e^{3t}
\]
And we are done.
\newpage


\indent Now we want to apply the Laplace transform to a PDE. The idea is the same, but our PDE involves
two variables: usually $x$, representative of position and $t$, representative of time. The Laplace
transform \textit{only} transforms with respect to our $t$ variable. Let's take a look at an
example:\\\\
\noindent \textbf{\textit{Ex:}} Solve the following PDE using a Laplace Transform.
\[
\begin{cases*}
u_{x} + u_{t} = x\\
u(0,t) = 0\\
u(x,0) = 0
\end{cases*}
\]
\indent \textbf{\textit{Solution:}} In this problem we are dealing with a first order PDE. Much
like an ODE, we simply need to determine what the Laplace transforms of $u_{x}$ and $u_{t}$ are.
We will let $\Lap[u(x,t)] = \Upsilon(x,s)$. Let's begin with $u_{t}(x,t)$.
\begin{align*}
\Lap[u_{t}] &= \int_{0}^{\infty}e^{-st}u_{t}(x,t)dt\\
&= e^{-st}u(x,t)\bigg|_{0}^{\infty} + s\int_{0}^{\infty}e^{-st}u(x,t)dt\\
&= s\Upsilon(x,s)-u(x,0)
\end{align*}
In a similar fashion, we could find $\Lap[u_{tt}(x,t)]$. We won't derive it here, but it ends up
looking very similar to our second derivative transform for our ODE:
\[
\Lap[u_{tt}(x,t)] = s^{2}\Upsilon(x,s)-su(x,0)-u_{t}(x,0)
\]
We also need to determine the transform for $u_{x}(x,t)$. Recall that the Laplace transform only
deals with the time variable, so our transforms for both $u_{x}(x,t)$ and $u_{xx}(x,t)$ become
$\Upsilon_{x}(x,s)$ and $\Upsilon_{xx}(x,s)$, respectivly. We now have all the information we need
to continue. Applying our transform yields
\begin{gather*}
\Lap[u_{x} + u_{t}] = \Lap[x]\\
\Upsilon'(x,s) + s\Upsilon(x,s)-u(x,0) = \frac{x}{s}\\
\Upsilon'(x,s) + s\Upsilon(x,s) =\frac{x}{s}
\end{gather*}
This is just a first order ODE. The only thing we are missing now is the initial condition, which
we can determine by transforming the initial condition on the PDE. In this case, $u(0,t) = 0$, so
the initial condition is still 0 after applying the transform. Now we have
\[
\begin{cases*}
\Upsilon'(x,s) + s\Upsilon(x,s) =\frac{x}{s}\\
\Upsilon(0,s) = 0
\end{cases*}
\]
We can solve this ODE by using the integrating factor method. The integrating factor in this case
is
\begin{align*}
\rho &= e^{\int s\mathop{dx}}\\
&= e^{sx}
\end{align*}
Our ODE now becomes
\[
\frac{d}{dx}(e^{sx}\Upsilon(x,t)) = \frac{e^{xs}x}{s}
\]
Integration, simplification, and applying our initial condition yields
\[
\Upsilon(x,t) = \frac{x}{s^{2}}-\frac{1}{s^{3}} + \frac{e^{-sx}}{s^{3}}
\]
Again, we won't dive into the complexity of the inverse Laplace transform here, but if we do take
the inverse, we get
\[
u(x,t) = xt - \frac{t^{2}}{2} - H(t-x)\frac{(t-x)^{2}}{2}
\]
The function $H(t-x)$ is the Heavyside Step Function, and is defined as
\[
H(x) =
\begin{cases*}
0,\quad x<0\\
1,\quad x\geq 0
\end{cases*}
\]
And we are done.
\newpage



\subsection{The Fourier Transform}
\indent The Fourier transform also allows us to take a PDE and transform it into an ODE. In this case,
however, we will be transforming with respect to the position variable. There are several variations to
the Fourier transform; we will look at the regular Fourier transform, the Fourier Sine transform, and the
Fourier Cosine transform.
\subsubsection{The Full Fourier Transfrom}
The Fourier transform of a function $f(x)$ is given by
\[
\Four[f(x)] = \frac{1}{2\pi}\int_{-\infty}^{\infty}e^{-i\omega x}f(x)\mathop{dx}
\]
Just like the Laplace transform, we will need to determine the Fourier transform for the derivatives
of functions as well. Recall that with the Laplace transfrom, $\Lap[u_{x}]$ became $\Upsilon_{x}$.
Using the same argument, we know that
\begin{gather*}
\Four[u_{t}(x,t)] = \Lambda_{t}(\omega,t)\\
\Four[u_{tt}(x,t)] = \Lambda_{tt}(\omega,t)
\end{gather*}
When we apply the Fourier transfrom to our $x$ derivatives, we get
\begin{gather*}
\Four[u_{x}(x,t)] = i\omega\Lambda(\omega,t)\\
\Four[u_{xx}(x,t)] = -\omega\Lambda(\omega,t)
\end{gather*}
Let's consider the following example:\\\\
\noindent \textbf{\textit{Ex:}} Solve the following PDE using a Fourier Transform.
\[
\begin{cases*}
2u_{x} + 3u_{t} = 0\\
u(x,0) = 3
\end{cases*}
\]
\indent \textbf{\textit{Solution:}} We are again dealing with a first order PDE here. We will
begin by taking the Fourier transform of both sides of the PDE. The Fourier transform of 0 is
still 0, so we get
\begin{gather*}
\Four[2u_{x} + 3u_{t}] = \Four[0]\\
2i\omega\Lambda(\omega,t) + 3\Lambda_{t}(\omega,t) = 0
\end{gather*}
This is just a first order ODE with respect to $t$. We also need to transform our inital condition.
Doing so yields
\[
\Four[3] = 3\delta(\omega)
\]
The delta function is defined as the derivative of the Heavyside Step Function. We now have an
ODE represented by
\[
\begin{cases*}
2i\omega\Lambda(\omega,t) + 3\Lambda_{t}(\omega,t) = 0\\
\Lambda(\omega,0) = 3\delta(\omega)
\end{cases*}
\]
Solving this ODE and applying the initial condition yields
\[
\Lambda(\omega,t) = 3\delta(\omega)e^{-\frac{2}{3}i\omega t}
\]
From this step, we would transform the equation back to the $x,t$ coordinate system. We won't do that
here, so we are done.
\newpage

\subsubsection{The Fourier Sine Transform}
