\section{Mathematica and Partial Differential Equations}
\hrule
\noindent \\\\
We will briefly talk about solving PDE's in Mathematica in this section. We can solve PDE's with
the tools available to solve ODE's. Let's consider an example equation:\\\\

\noindent \textbf{\textit{Ex:}} Solve the following PDE
\[
u_{t} + 2u_{x} = 0
\]
\indent \textbf{\textit{Solution:}} We can define both $u_{x}$ and $u_{t}$ in Mathematica by using
the D command.
\begin{minted}{mathematica}
 In[1]:= ClearAll["Global"];
         pde = D[u[x,t], t] + 2 D[u[x,t], x] == 0;
\end{minted}
Here, we have assigned our PDE to the Mathematica variable pde. Note that we never gave a definition
to the function $u(x,t)$. It's not necessary; Mathematica understands what we mean. To solve the
equation, we will use the DSolve command, just as we would with an ODE.
\begin{minted}{mathematica}
 In[2]:= solution = DSolve[pde, u[x,t], {x,t}]
Out[2]:= {{u[x,t] -> C[1][(1/2)(2t-x)]}}
\end{minted}
Mathematica represents the arbitrary function that we get when solving with C[1].
If we want to assign the solution for later use, we can create a new function and assign it to the
solution that we just found.
\begin{minted}{mathematica}
 In[3]:= f[x_,t_] := u[x,t] /. solution[[1]];
\end{minted}
The line above assigns u[x,t] to the function f[x,t]. We can also apply an initial condition by
providing DSolve with a list of equations rather than just the PDE.
\begin{minted}{mathematica}
 In[4]:= solution = DSolve[{pde, u[0,t] == Sin[t]},u[x,t],{x,t}]
Out[4]:= {{u[x,t] -> Sin[(1/2)(2t-x)]}}
\end{minted}
Here we have provided an initial condition of $\sin{t}$. Again, we could assign the result of Out[4] to
a function that we could use later. It may benefit us, for example, to plot the result.
\begin{minted}{mathematica}
 In[5]:= Plot3D[u[x,t]/.solution,{x,-4,4},{t,-4,4}];
\end{minted}
We've supressed the output here, but the result would be the graph of the function in the 3D plane.
