\section{The Cauchy Problem}
\hrule
\noindent \\\\
\indent Up until now, we have always been given our PDE's with boundary conditions and initial values; but imagine that we did not have a boundary? The traditional method that we have been using would fail at this point. Since we would not be able to find any eigenvalues, we would not be able to form a series solution by using the \textit{Separation of Variables} technique. Let's consider the familiar heat equation:
\[
\begin{cases}
\xi_{t} - k\xi_{xx} = 0\\
\xi(x,0) = f(x)
\end{cases}
\]
Through much careful derivation and theory, we can say that the general solution for this problem will be
\[
\xi(x,t) = \frac{1}{\sqrt{4 \pi k t}}\int_{-\infty}^{\infty}e^{-\frac{(x-y)^{2}}{4kt}}f(y)dy.
\]
The question now should be ``How can we solve this?'' The first step we will take is letting
\[
u = \frac{y-x}{\sqrt{4kt}}.
\]
We are going to perform a \textit{u substitution} to try and put this integral into a nicer form. This yields
\begin{align*}
y &= x + u\sqrt{4kt}\\
dy &= \sqrt{4kt}\ du.
\end{align*}
This changes our original problem into
\[
\xi(x,t) = \frac{1}{\sqrt{\pi}}\int_{-\infty}^{\infty}e^{-u^{2}}f(x+u\sqrt{4kt})du.
\]
Let's look at an example\\\\
\noindent \textbf{\textit{Ex:}} Solve the following PDE
\[
\begin{cases}
\mu_{t} - k\mu_{xx} = 0\\
\mu(x,0) = \cos{x}
\end{cases}
\]
\indent \textbf{\textit{Solution:}} From above, we know what our general solution will look like. All we need to do is substitute in our initial condition. We have the following:
\begin{align*}
\mu(x,t) &= \frac{1}{\sqrt{\pi}}\int_{-\infty}^{\infty}e^{-u^{2}}\cos{(x + u\sqrt{4kt})}du\\
&= \frac{1}{\sqrt{\pi}}\int_{-\infty}^{\infty}e^{-u^{2}}\left[\cos{(x)}\cos{(u\sqrt{4kt})} - \sin{(x)}\sin{(u\sqrt{4kt})}\right]du\\
&=  \frac{2}{\sqrt{\pi}}\cos{(x)}\int_{0}^{\infty}e^{-u^{2}}\cos{(u\sqrt{4kt})}du.
\end{align*}
You may be wondering what happened to the $\sin$ terms. Recall that $\sin$ is an odd function, so it will ultimately be $0$ on the interval $[-\infty,\infty]$; because of this fact, we can remove it now. We also use the face that the $\cos$ is even, so it is okay to halve the interval and multiply by two. Now we need to evaluate this integral. We will substitute $\alpha = \sqrt{4kt}$, and assign a function, $\Upsilon(\alpha)$, to the integral. This gives us the following system:
\[
\begin{cases}
\Upsilon(\alpha) = \int_{0}^{\infty}e^{-u^{2}}\cos{(\alpha u)}du\\
\Upsilon(0) = \int_{0}^{\infty}e^{-u^{2}}du\\
\end{cases}
\]
Believe it or not, we actually know the answer to $\Upsilon(0)$. To find it, we can assign a function to the integral in a similar fashion as we have already done. Calling it $\Psi$, we have
\begin{align*}
\Psi &= \int_{0}^{\infty}e^{-u^{2}}du\\
\Psi^{2} &= \int_{0}^{\infty}e^{-u^{2}}du \int_{0}^{\infty}e^{-v^{2}}dv\\
\Psi^{2} &= \int_{0}^{\infty}\int_{0}^{\infty}e^{-u^{2}-v^{2}}dudv\\
\Psi^{2} &= \int_{0}^{\pi/2}\int_{0}^{\infty}re^{-r^{2}}drd\theta\\
\Psi^{2} &= \frac{\pi}{2}\left[-\frac{1}{2}e^{-r^{2}}\right]\Bigg |_{r=0}^{r=\infty}\\
\Psi^{2} &= \frac{\pi}{4}\\
\Psi &= \frac{\sqrt{\pi}}{2}.
\end{align*}
So we know that $\Upsilon(0) = \frac{\sqrt{\pi}}{2}$. Plugging this into our system gives us
\[
\begin{cases}
\Upsilon(\alpha) = \int_{0}^{\infty}e^{-u^{2}}\cos{(\alpha u)}du\\
\Upsilon(0) = \frac{\sqrt{\pi}}{2}\\
\end{cases}
\]
Now, we will take the derivative of $\Upsilon$. This gives us
\[
\Upsilon'(\alpha) = -\int_{0}^{\infty}ue^{-u^{2}}\sin{(\alpha u)}du.
\]
Integration by parts gives us
\[
\frac{1}{2}e^{-u^{2}}\sin{(\alpha u)}\Big |_{0}^{\infty} - \alpha\int_{0}^{\infty}\frac{1}{2}e^{-u^{2}}\cos{(\alpha u)}du.
\]
The value from $0$ to $\infty$ is zero, since the $\sin$ function odd. This leaves us with following relationship:
\[
\Upsilon'(\alpha) = -\frac{\alpha}{2}\Upsilon(\alpha) \alpha.
\]
We can solve this as an ODE. It's separable, so we get
\begin{align*}
\frac{d\Upsilon}{\Upsilon} &= -\frac{\alpha}{2}d\alpha\\
\ln{(\Upsilon)} &= -\frac{\alpha^{2}}{4} + c_{1}\\
\Upsilon &= c_{2}e^{-\frac{\alpha^{2}}{4}}.
\end{align*}
Applying the initial condition gives us
\[
\Upsilon = \frac{\sqrt{\pi}}{2}e^{-\frac{\alpha^{2}}{4}}.
\]
Substituting back in gives us a final answer of
\begin{align*}
\mu(x,t) &= \frac{2}{\sqrt{\pi}}\cos{x}\Upsilon(\sqrt{4kt})\\
&= \cos{x}e^{-kt},
\end{align*}
and we are done.


\newpage
\indent Now we will move on to the Cauchy wave equation. We will be presented with a similar problem: a wave equation on an unbounded domain. Consider the following problem:
\noindent \\\\
\noindent \textbf{\textit{Ex:}} Solve the following PDE
\[
\begin{cases}
u_{tt} - c^{2}u_{xx} = 0\\
u(x,0) = \sin{x}\\
u_{t}(x,0) = \cos{x}
\end{cases}
\]
\indent \textbf{\textit{Solution:}} Just as with the heat equation, we have a formal solution for the homogeneous wave equation. We will do the derivation, but I will tell you that the formal solution is
\[
u(x,t) = \frac{1}{2}\left[f(x+ct) + f(x-ct)\right] + \frac{1}{2c}\int_{x-ct}^{x+ct}g(s)\mathop{ds}.
\]
Now let's derive the formal solution. We want a solution in the form of
\[
u(x,t) = F(x+ct) + G(x-ct).
\]
Now we can say that $u_{t} = cF'(x+ct) - cG'(x-ct)$, and $u_{t}(x,0) = cF'(x) - cG'(x)$. This means
\begin{align*}
F(x) + G(x) &= f(x)\\
F(x) - G(x) &= \frac{1}{c}\int_{0}^{x}g(s)\mathop{ds} + c_{1}.
\end{align*}
Solving the system yields
\begin{align*}
u(x,t) &= \frac{1}{2} f(x+ct) + \frac{1}{2c}\int_{0}^{x+ct}g(s)\mathop{ds} + \frac{c_{1}}{2} + \frac{1}{2}f(x-ct) -\frac{1}{2c}\int_{0}^{x-ct}g(s)\mathop{ds} - \frac{c_{1}}{2}\\
&= \frac{1}{2}\left[f(x+ct) + f(x-ct)\right] + \frac{1}{2c}\int_{x-ct}^{x+ct}g(s)\mathop{ds},
\end{align*}
and that is the derivation. This problem is simply plugging into the formal solution. We have the following:
\begin{align*}
u(x,t) &= \frac{1}{2}\left[\sin{(x+ct)} + \sin{(x-ct)}\right] + \frac{1}{2c}\int_{x-ct}^{x+ct}\cos{(s)}\mathop{ds}\\
&= \frac{\sin{(x+ct)} + \sin{(x-ct)}}{2} + \frac{\cos{(x)}\sin{(ct)}}{c}\\
&=  \frac{\cos{(x)}\sin{(ct)}}{c} + \cos{(ct)}\sin{(x)},
\end{align*}
and we are done.


\newpage
\indent So far, we have only dealt with homogeneous Cauchy problems. We will now look at the case where we have a non-homogeneous problem. The first variation of this problem is the \textit{half-line} problem. In this problem, we will have a bound on our equation, and we will generalize the solution so that we are dealing with an unbounded Cauchy problem. Let's look at an example.
\noindent\\\\ \textbf{\textit{Ex:}} Solve the following PDE
\[
\begin{cases}
u_{t} - ku_{xx} = 0\\
u(x,0) = \phi(x)\\
u(0,t) = 0
\end{cases}
\]
\indent \textbf{\textit{Solution:}} We have both an initial condition and a boundary condition here. This problem is saying that at $x=0$, there is an impassable boundary. This forces our domain to be $[0,\infty)$. To solve a Cauchy problem, however, we want a domain of $(-\infty,\infty)$. We can force this domain by extending our initial condition, $\phi(x)$, over the boundary. We have the following rule:
\[
\phi_{0}(x) =
\begin{cases}
\phi(x)\quad\quad x>0\\
-\phi(-x)\ x<0.
\end{cases}
\]
Now we are dealing with the system
\[
\begin{cases}
u_{t} - ku_{xx} = 0\\
u(x,0) = \phi_{0}(x),
\end{cases}
\]
which we can solve using the formal solution to the heat equation. Recall that
\[
u(x,t) = \frac{1}{\sqrt{4k\pi t}}\int_{-\infty}^{\infty}e^{-\frac{(x-y)^{2}}{4kt}}\phi_{0}(y)\mathop{dy}.
\]
Substituting into the equation gives us
\begin{align*}
u(x,t) &= \frac{1}{\sqrt{4k\pi t}}\left[\int_{0}^{\infty}e^{-\frac{(x-y)^{2}}{4kt}}\phi(y)\mathop{dy} + \int_{-\infty}^{0}e^{-\frac{(x-y)^{2}}{4kt}}(-\phi(-y))\mathop{dy} \right]\\
&= \frac{1}{\sqrt{4k\pi t}}
\end{align*}
