\section{High Order Partial Differential Equations}
\hrule
\noindent \\\\
This section is concerned with high order PDE's. We will be focusing on PDE's that are still linear,
but contain third, fourth, and higher derivatives. Before we continue, we will need to discuss Green's
theorem.

\subsection{Green's Theorem}
Green's theorem stems from the harmonic function that we know as Laplace's equation. We have seen the
operator $\nabla$, which by definition means
\[
\nabla f = <f_{x},f_{y},f_{z}>
\]
When we dealt with Laplace's equation, we saw the operator used as
\[
\nabla^{2}u = u_{xx} + u_{yy} = 0
\]
We ignored the $u_{zz}$ term because we were only concerned with the 2D plane. Green's function provides
a universal solution to harmonic equations in higher orders than the Laplace equation which we are used to
dealing with. Before we continue, we need to discuss Green's Identities.
\subsubsection{Green's First Identity}
Green's first identity is represented by the equation
\[
\int_{D}\nabla\phi\cdot\nabla\psi\mathop{dV} + \int_{D}\phi\nabla^{2}\psi\mathop{dV} =
\int_{C}\phi\nabla\psi\cdot\mathbf{n}\mathop{dS}
\]
This result follows from the \textit{Divergence Theorem} and the product rule for partial derivatives.
We have $D$, which is the closed surface over which we are integrating, and two function $\phi$ and
$\psi$. Green's first identity provides us with a way to relate $n$ integrals to $n - 1$ integrals of the
same function (it can be extended to show that the identity works for $n$ and $n-1$ integrals).
% We can consider the Neumann style problem as an application of Green's first identity. If we let
% $\psi = 1$, Green's first identity becomes
% \[
% \iint_{\partial D}\frac{\partial\phi}{\partial n}\mathop{dS} = \iiint_{D}\nabla^{2}\phi\mathop{dx}
% \]
\subsubsection{Green's Second Identity}
We can derive Green's second identity from the first identity realtively simply. We simply need to
interchange $\psi$ and $\phi$, then subtract across. This yields
% G = \phi
% u = \psi
% dA = dV
\[
\int_{D} \psi\nabla^{2}\phi\mathop{dV} - \int_{D}\phi\nabla^{2}\psi\mathop{dV} =
\int_{C}\psi\nabla\phi\cdot\mathbf{n}\mathop{dS} - \int_{C}\phi\nabla\psi\cdot\mathbf{n}\mathop{dS}
\]
Again, this identity relates an $n$ dimensional surface integral to an $n-1$ dimesional surface integral
of the same functions.
\subsubsection{The Delta Function}
In addition to Green's identities, we also need to discuss the delta function, $\delta(x)$. We have
seen this function before (recall $\Four[1]$). Looking at the function in depth, we have the following
definition:
\[
\delta(x)=
\begin{cases*}
0,\quad x\neq0\\
\infty\quad x=0
\end{cases*}
\]
We also have the following properties of the delta function:
\begin{gather*}
\int_{-\infty}^{\infty}\delta(x)\mathop{dx} = 1\\
\int_{-\infty}^{\infty}f(x)\delta(x-a)\mathop{dx} = f(a)
\end{gather*}
The second property, called the sifting property, is of particular interest, as we will see soon.
The delta function can also be extended to the 2D case:
\[
\delta(x,y) =
\begin{cases*}
0,\qquad(x,y)\neq0\\
\infty,\qquad(x,y)=0
\end{cases*}
\]
The sifting property from the 1D case becomes
\[
\iint f(x,y)\delta(x-a,y-b)\mathop{dA} = f(a,b)
\]
where $A$ is a surface over which we are integrating.
\subsubsection{Green's Functions}
Finally, we need to discuss the notion of a Green's Function. A Green Function is formally defined
as
\[
G(x,s) = \mathcal{L}^{-1}\delta(x-s)
\]
For some linear differential operator $\mathcal{L}$. If we take a function $f(x)$ and multiply and
integrate, we get the following identity:
\[
u(x) = \int G(x,s)f(s)\mathop{ds}
\]
Green's Functions can be used to solve many different types of equations;
For any general ODE, we require two constraints to be fulfilled:
\begin{enumerate}
\item A solution to the ODE exists, and
\item the solutions are linearly independant.
\end{enumerate}
Recall that we can check for linear independance of solution via the Wronskian, $\mathcal{W}$.
We can derive a Green's Function for a differential equation by
\[
G(x,s) =
\begin{cases*}
\frac{y_{1}(s)y_{2}(x)}{\mathcal{W}(y_{1},y_{2})(s)},\qquad a\leq s\leq x\leq b\\
\frac{y_{1}(x)y_{2}(s)}{\mathcal{W}(y_{1},y_{2})(s)},\qquad a\leq x\leq s\leq b
\end{cases*}
\]
which yields the solution
\[
y(x) = \int_{a}^{b}G(x,s)f(s)\mathop{ds}
\]
Let's consider an example.\\

\noindent\textbf{\textit{Ex:}} Solve the following Boundary Value Problem:
\[
\begin{cases*}
y''(x) = x^{2}\\
y(0) = 0\\
y(1) = 0
\end{cases*}
\]
\indent\textbf{\textit{Solution:}} To solve this problem, we need to first find the Green Function
associated with the problem. Let's consider the homogeneous ODE $y'' = 0$. This has the solution
\[
y(x) = c_{1} + c_{2}x
\]
We can take any arbitrary $c$ as long as our choice does not yield a trivial solution and it satisfies
the boundary conditions. Let's consider
\[
y_{1}(x) = x\qquad\qquad\text{and}\qquad\qquad y_{2}(x) = 1 - x
\]
We can see that this choice satisfies the boundary conditions, and that
\[
\left|
\begin{array}{c c}
0 & 1\\
1 & 1
\end{array}\right|
= -1 \neq 0
\]
which implies that the choice we made is linearly independant. We can now derive a Green's function
for our ODE as follows:
\begin{align*}
G(x,s) &=
\begin{cases*}
\frac{s(1-x)}{-1}\\
\frac{x(1-s)}{-1}
\end{cases*}\\
&=
\begin{cases*}
x-s\\
s-x
\end{cases*}
\end{align*}
We are bounded by $[0,1]$ in this case, so we really have
\[
G(x,s) =
\begin{cases*}
x-s,\qquad 0\leq s\leq x\\
s-x,\qquad x\leq s\leq 1
\end{cases*}
\]
We said earlier that the solution $y(x)$ can be found by
\begin{align*}
y(x) &= \int_{a}^{b}G(x,s)f(s)\mathop{ds}\\
&= \int_{0}^{x}(x-s)s^{2}\mathop{ds} + \int_{x}^{1}(s-x)(s^{2})\mathop{ds}\\
&= \frac{1}{12}(x^{x} - x)
\end{align*}
And we are done.
\newpage

Green's Functions are useful for more than ODE's. We want to extend the idea to PDE's, which can
be doe simply enough. Before we continue, we will mention Poisson's Equation, which is a
generalized form of Laplace's Equation:
\[
\nabla^{2}\psi = -4\pi\rho
\]
If we take $\rho = 0$, we get back Laplace's Equation. Let's consider the following example:\\


\noindent\textbf{\textit{Ex:}} Solve the Poisson equation
\[
\begin{cases*}
\nabla^{2}u = -f(x,y)\qquad \text{on }\Omega = \{(x,y)|0<x<\pi,0<y<\pi\}\\
u = 0,\qquad\qquad\qquad\text{on } \partial\Omega
\end{cases*}
\]
\indent\textbf{\textit{Solution:}} We know the formal solution to this PDE:
\[
u(x,y) = \sum_{n=1}^{\infty}b_{n}(y)\sin{(nx)}
\]
